% Options for packages loaded elsewhere
\PassOptionsToPackage{unicode}{hyperref}
\PassOptionsToPackage{hyphens}{url}
\PassOptionsToPackage{dvipsnames,svgnames,x11names}{xcolor}
%
\documentclass[
]{interact}

\usepackage{amsmath,amssymb}
\usepackage{iftex}
\ifPDFTeX
  \usepackage[T1]{fontenc}
  \usepackage[utf8]{inputenc}
  \usepackage{textcomp} % provide euro and other symbols
\else % if luatex or xetex
  \usepackage{unicode-math}
  \defaultfontfeatures{Scale=MatchLowercase}
  \defaultfontfeatures[\rmfamily]{Ligatures=TeX,Scale=1}
\fi
\usepackage{lmodern}
\ifPDFTeX\else  
    % xetex/luatex font selection
\fi
% Use upquote if available, for straight quotes in verbatim environments
\IfFileExists{upquote.sty}{\usepackage{upquote}}{}
\IfFileExists{microtype.sty}{% use microtype if available
  \usepackage[]{microtype}
  \UseMicrotypeSet[protrusion]{basicmath} % disable protrusion for tt fonts
}{}
\makeatletter
\@ifundefined{KOMAClassName}{% if non-KOMA class
  \IfFileExists{parskip.sty}{%
    \usepackage{parskip}
  }{% else
    \setlength{\parindent}{0pt}
    \setlength{\parskip}{6pt plus 2pt minus 1pt}}
}{% if KOMA class
  \KOMAoptions{parskip=half}}
\makeatother
\usepackage{xcolor}
\setlength{\emergencystretch}{3em} % prevent overfull lines
\setcounter{secnumdepth}{5}
% Make \paragraph and \subparagraph free-standing
\ifx\paragraph\undefined\else
  \let\oldparagraph\paragraph
  \renewcommand{\paragraph}[1]{\oldparagraph{#1}\mbox{}}
\fi
\ifx\subparagraph\undefined\else
  \let\oldsubparagraph\subparagraph
  \renewcommand{\subparagraph}[1]{\oldsubparagraph{#1}\mbox{}}
\fi


\providecommand{\tightlist}{%
  \setlength{\itemsep}{0pt}\setlength{\parskip}{0pt}}\usepackage{longtable,booktabs,array}
\usepackage{calc} % for calculating minipage widths
% Correct order of tables after \paragraph or \subparagraph
\usepackage{etoolbox}
\makeatletter
\patchcmd\longtable{\par}{\if@noskipsec\mbox{}\fi\par}{}{}
\makeatother
% Allow footnotes in longtable head/foot
\IfFileExists{footnotehyper.sty}{\usepackage{footnotehyper}}{\usepackage{footnote}}
\makesavenoteenv{longtable}
\usepackage{graphicx}
\makeatletter
\def\maxwidth{\ifdim\Gin@nat@width>\linewidth\linewidth\else\Gin@nat@width\fi}
\def\maxheight{\ifdim\Gin@nat@height>\textheight\textheight\else\Gin@nat@height\fi}
\makeatother
% Scale images if necessary, so that they will not overflow the page
% margins by default, and it is still possible to overwrite the defaults
% using explicit options in \includegraphics[width, height, ...]{}
\setkeys{Gin}{width=\maxwidth,height=\maxheight,keepaspectratio}
% Set default figure placement to htbp
\makeatletter
\def\fps@figure{htbp}
\makeatother
% definitions for citeproc citations
\NewDocumentCommand\citeproctext{}{}
\NewDocumentCommand\citeproc{mm}{%
  \begingroup\def\citeproctext{#2}\cite{#1}\endgroup}
\makeatletter
 % allow citations to break across lines
 \let\@cite@ofmt\@firstofone
 % avoid brackets around text for \cite:
 \def\@biblabel#1{}
 \def\@cite#1#2{{#1\if@tempswa , #2\fi}}
\makeatother
\newlength{\cslhangindent}
\setlength{\cslhangindent}{1.5em}
\newlength{\csllabelwidth}
\setlength{\csllabelwidth}{3em}
\newenvironment{CSLReferences}[2] % #1 hanging-indent, #2 entry-spacing
 {\begin{list}{}{%
  \setlength{\itemindent}{0pt}
  \setlength{\leftmargin}{0pt}
  \setlength{\parsep}{0pt}
  % turn on hanging indent if param 1 is 1
  \ifodd #1
   \setlength{\leftmargin}{\cslhangindent}
   \setlength{\itemindent}{-1\cslhangindent}
  \fi
  % set entry spacing
  \setlength{\itemsep}{#2\baselineskip}}}
 {\end{list}}
\usepackage{calc}
\newcommand{\CSLBlock}[1]{\hfill\break\parbox[t]{\linewidth}{\strut\ignorespaces#1\strut}}
\newcommand{\CSLLeftMargin}[1]{\parbox[t]{\csllabelwidth}{\strut#1\strut}}
\newcommand{\CSLRightInline}[1]{\parbox[t]{\linewidth - \csllabelwidth}{\strut#1\strut}}
\newcommand{\CSLIndent}[1]{\hspace{\cslhangindent}#1}

\usepackage{booktabs}
\usepackage{caption}
\usepackage{longtable}
\usepackage{colortbl}
\usepackage{array}
\usepackage{lscape}
\newcommand{\blandscape}{\begin{landscape}}
\newcommand{\elandscape}{\end{landscape}}
\usepackage{orcidlink}
\makeatletter
\@ifpackageloaded{caption}{}{\usepackage{caption}}
\AtBeginDocument{%
\ifdefined\contentsname
  \renewcommand*\contentsname{Table of contents}
\else
  \newcommand\contentsname{Table of contents}
\fi
\ifdefined\listfigurename
  \renewcommand*\listfigurename{List of Figures}
\else
  \newcommand\listfigurename{List of Figures}
\fi
\ifdefined\listtablename
  \renewcommand*\listtablename{List of Tables}
\else
  \newcommand\listtablename{List of Tables}
\fi
\ifdefined\figurename
  \renewcommand*\figurename{Figure}
\else
  \newcommand\figurename{Figure}
\fi
\ifdefined\tablename
  \renewcommand*\tablename{Table}
\else
  \newcommand\tablename{Table}
\fi
}
\@ifpackageloaded{float}{}{\usepackage{float}}
\floatstyle{ruled}
\@ifundefined{c@chapter}{\newfloat{codelisting}{h}{lop}}{\newfloat{codelisting}{h}{lop}[chapter]}
\floatname{codelisting}{Listing}
\newcommand*\listoflistings{\listof{codelisting}{List of Listings}}
\makeatother
\makeatletter
\makeatother
\makeatletter
\@ifpackageloaded{caption}{}{\usepackage{caption}}
\@ifpackageloaded{subcaption}{}{\usepackage{subcaption}}
\makeatother
\ifLuaTeX
  \usepackage{selnolig}  % disable illegal ligatures
\fi
\usepackage{bookmark}

\IfFileExists{xurl.sty}{\usepackage{xurl}}{} % add URL line breaks if available
\urlstyle{same} % disable monospaced font for URLs
\hypersetup{
  pdftitle={The detection and accurate estimation of frequency attenuation effects in masked repetition priming: A large scale web browser-based study},
  pdfauthor={Roberto Petrosino; Diogo Almeida},
  pdfkeywords={masked repetition priming, frequency attenuation
effect, online browser-based experiment, power analysis},
  colorlinks=true,
  linkcolor={blue},
  filecolor={Maroon},
  citecolor={Blue},
  urlcolor={Blue},
  pdfcreator={LaTeX via pandoc}}

\title{The detection and accurate estimation of frequency attenuation
effects in masked repetition priming: A large scale web browser-based
study}
\author{Roberto
Petrosino$\textsuperscript{1}$~\orcidlink{0000-0002-8502-3070}, Diogo
Almeida$\textsuperscript{1}$~\orcidlink{0000-0003-4674-8092}}

\thanks{CONTACT: Roberto
Petrosino. Email: \href{mailto:roberto.petrosino@nyu.edu}{\nolinkurl{roberto.petrosino@nyu.edu}}. Diogo
Almeida. Email: \href{mailto:diogo@nyu.edu}{\nolinkurl{diogo@nyu.edu}}. }
\begin{document}
\captionsetup{labelsep=space}
\maketitle
\textsuperscript{1} Psychology Program, Division of Science, New York
University Abu Dhabi, Abu Dhabi, United Arab Emirates
\begin{abstract}
This study investigates the controversy surrounding the sensitivity of
masked repetition priming to word frequency: while unmasked priming
exhibits a frequency attenuation effect, wherein high frequency words
yield smaller repetition effects, this phenomenon has been
inconsistently reported in masked priming. We conducted two large online
experiments with rigorously validated frequency databases to reconcile
past discrepancies. The first experiment confirmed the viability of
conducting masked priming experiments in web browser-based settings. The
pre-registered second study, designed for high statistical power and
precision, identified a 10-ms attenuation effect under masked priming.
This result suggests that the repetition effect in masked priming is
less qualitatively distinct from unmasked priming than previously
assumed. This finding has implications for masked priming experimental
design and theoretical consequences for models of priming. Crucially,
models that predict either the presence or absence of frequency
attenuation under masked conditions need to account for a small but
reliable effect.
\end{abstract}
\begin{keywords}
\def\sep{;\ }
masked repetition priming\sep frequency attenuation effect\sep online
browser-based experiment\sep 
power analysis
\end{keywords}

\section{Introduction}\label{sec-intro}

The masked priming technique has been an invaluable tool in visual word
recognition research. It has allowed researchers to study the conditions
under which orthographic, phonological, morphological, and semantic
information impact access to visual word forms while mitigating
strategic effects and minimizing the influence of controlled processes
(Forster 1998). First introduced in its traditional form by Forster and
Davis (1984; see also Evett and Humphreys 1981), this technique involves
a forward mask (i.e., usually a string of hashes, \#\#\#\#\#), followed
by a prime string presented for very short time (\(SOA < 60\)
ms),\footnote{\emph{SOA: Stimulus Onset Asynchrony}, i.e.~the time
  between the start of one stimulus (in our case, the prime stimulus)
  and the start of another stimulus (the target stimulus). In the
  standard repetition priming design, no backward mask occurs between
  the prime and the target, and therefore SOA equals the duration of
  prime presentation.} and a target string presented immediately after.
Because the prime presentation is so brief and masked by preceding and
subsequent stimuli, most participants report not being aware that a
prime string has been presented, and can at most report a screen flicker
just before the target presentation (Forster, Mohan, and Hector 2003).

Among possible manipulations of prime-target relatedness, masked
repetition priming (in which the same word is presented as both the
prime and target within the same trial: e.g., \emph{love-LOVE}) has been
well studied, because its response seems to be qualitatively different
from the unmasked counterpart (\(SOA > 60 ms\)): while high-frequency
words benefit less from repetition than low-frequency words in the
unmasked design (\emph{frequency attenuation effect}, henceforth FAE;
Scarborough, Cortese, and Scarborough 1977), this does not seem to be
the case when the prime is masked (Forster and Davis 1984; Forster et
al. 1987; Segui and Grainger 1990; Sereno 1991; Forster and Davis 1991;
Rajaram and Neely 1992; Bodner and Masson 1997; Forster, Mohan, and
Hector 2003; Nievas 2010).

This asymmetry in sensitivity to lexical frequency between the masked
and unmasked repetition priming responses has been important in
distinguishing among different models of priming in visual word
recognition. More specifically, \emph{interactive activation models}
(McClelland and Rumelhart 1981; Grainger and Jacobs 1996; Coltheart et
al. 2001) conceive of priming as a ``head start'' in processing due to
the pre-activation of the target word due to the presentation of the
prime. Thus, according to \emph{interactive activation} models, priming
is ultimately caused by a single mechanism, making the qualitatively
different profiles for repetition priming in masked and unmasked
conditions a difficult empirical finding to explain.

Similarly, episodic models (e.g., Jacoby and Dallas 1981; Jacoby 1983)
posit a different single mechanism for priming effects: the
activation/retrieval of the episodic memory trace of the encounter with
the prime word. These models therefore encounter the same type of
difficulty in accounting for qualitatively different patterns of
repetition priming effects in masked and unmasked conditions. A similar
type of model, called the \emph{memory recruitment model} makes very
similar predictions to the episodic memory models, positing a
non-lexical source for priming effects (Bodner and Masson 1997; Masson
and Bodner 2003; Bodner and Masson 2014). Repetition priming effects
under this view stem from the exploitation, strategically or
automatically, of a memory resource created by the encounter with the
prime word. The frequency attenuation effect, under episodic and memory
recruitment models alike, is predicted on the basis that low frequency
primes, being more distinctive stimuli, create a more potent and
effective memory resource compared to high frequency primes.

In contrast, other models appear to successfuly sidestep the problem
posed by the qualitatively different repetition priming profiles
observed in masked and unmasked conditions. One such model is the
\emph{entry-opening model} (also known as the \emph{bin model}; Forster
and Davis 1984). According to this model, when the visual stimulus is
presented, lexical entries are assigned to specific bins based on
orthographic similarity. In the first stage (fast search stage), a fast,
frequency-ordered search goes through the entries within a given bin,
and compares each one with the the input stimulus, assigning to each
entry a goodness-of-fit score. This comparison is fast and crude, and
sorts entries into (a) perfect (i.e., no difference is detected between
the input and the entry), (b) close (i.e., small differences are
detected), and (c) irrelevant matches (i.e., substantial difference are
detected). Any entry of type (a) or (b) is opened, so that the entry can
be further analyzed and compared to the input in the subsequent
verification stage. Under a masked presentation, the entry of the prime
word is opened at the fast search stage, but the short duration of the
stimulus prevents it from reaching the evaluation stage. Crucially, the
entry is nonetheless left open. Upon the presentation of the target
stimulus, the access procedure will follow its two stage course, with a
frequency-sensitive fast search and a subsequent entry opening for
evaluation/verification. In this view, the fast search for the target
word proceeds normally, but the evaluation/verification procedure starts
and ends sooner than it otherwise would, because the target entry has
already been left open after the brief processing of the prime. Thus,
the \emph{entry-opening model} explains the masked repetition priming as
the benefit from having the entry of the target word already open by the
time the second stage of recognition starts. Crucially, this occurs
\emph{after} the target word is initially accessed, which happens in
order of frequency. Put differently, according to the
\emph{entry-opening model}, masked repetition priming occurs because of
the time savings from not having to open the entry, which is a
frequency-insensitive process (i.e., every entry takes the same time to
be opened), but \emph{after} the frequency-sensitive first access stage.
As a consequence, the \emph{entry-opening model} predicts a
frequency-insensitive masked repetition priming effect, which is what
has been traditionally reported in the literature (see
Table~\ref{tbl-litReview}). In addition, it also (correctly) predicts
that pseudowords should not benefit from masked repetition priming, as
they have no entries in the mental lexicon to be left open after the
brief processing of the prime.

However, as Table~\ref{tbl-litReview} shows, there are nonetheless a few
studies that do report significant FAEs in masked repetition priming
Nievas (2010). Bodner and Masson (2001) report that when stimuli are
presented in alternating case (e.g., \emph{pHoNe}), this increases the
lexical decision difficulty and therefore generates an extra incentive
to draw on the memory resource created by the brief processing of the
prime. Under such conditions, they were able to observe a statistically
significant FAE. In the same vein, Kinoshita (2006) noticed that in
earlier studies the low frequency words often had very high error rates,
and suggested that perhaps many participants did not know them. If
participants treated a substantial number of low frequency words as
nonwords, and nonwords do not exhibit repetition priming under masked
conditions, it could artificially depress the repetition priming effect
for the low frequency condition alone, which could make any existing FAE
harder to detect. In two separate experiments, Kinoshita (2006) showed
that larger repetition priming effects for low frequency words were only
obtained when the low frequency words were vetted to make sure the
participants knew them prior to the experiment. Following up on
Kinoshita (2006), Norris and Kinoshita (2008) were also able to find an
interaction between lexical frequency and repetition in masked
repetition priming.

Finally, as Table~\ref{tbl-litReview} shows, it is noteworthy that 15
out of 18 previous studies showed numerically larger masked priming
effects for low frequency words as opposed to high frequency words,
irrespective of statistical significance. Similarly, the average
repetition effect for low frequency words in the studies reviewed in
Table~\ref{tbl-litReview} is 13 ms larger when compared to that of high
frequency words. These results are not in line with the predictions
dictated by the \emph{entry opening model}, and seem to align better
with the predictions made by \emph{interactive activation models} and
\emph{memory recruitment} models.

\blandscape

\setlength{\LTpost}{0mm}

\begin{longtable}{l|lrlrrrrll}

\caption{\label{tbl-litReview}Summary of the masked repetition priming
effects as a function of word frequency reported in the literature. The
power range estimates were calculated by simulating 10,000 datasets with
the corresponding sample size (N) and FAE = 15 ms and 30 ms.}

\tabularnewline

\toprule
\multicolumn{1}{l}{} &  &  &  & \multicolumn{2}{c}{MOP (ms)} & \multicolumn{2}{c}{FAE (ms)} & \multicolumn{2}{c}{Power range [min max]} \\ 
\cmidrule(lr){5-6} \cmidrule(lr){7-8} \cmidrule(lr){9-10}
\multicolumn{1}{l}{Study} & Language & N & SOA & HF & LF & ES & \emph{p}\textless{}.05? & FAE=15ms & FAE=30ms \\ 
\midrule\addlinespace[2.5pt]
Forster, Davis, Schoknecht, \& Carter (1987), exp. 1 & English & 16 & 60 & 61 & 66 & 5 &   & [0.02 0.24] & [0.04 0.84] \\ 
Norris, Kinoshita, Hall, \& Henson (2018) & English & 16 & 50 & 38 & 51 & 13 &   & [0.02 0.24] & [0.04 0.84] \\ 
Sereno (1991), exp. 1 & English & 20 & 60 & 40 & 64 & 24 &   & [0.02 0.33] & [0.04 0.92] \\ 
Forster \& Davis (1991), exp. 5 & English & 24 & 60 & 54 & 72 & 18 &   & [0.02 0.4] & [0.05 0.96] \\ 
Bodner \& Masson (1997), exp. 1 & English & 24 & 60 & 29 & 45 & 16 &   & [0.02 0.4] & [0.05 0.96] \\ 
Bodner \& Masson (1997), exp. 3 & English & 24 & 60 & 36 & 50 & 14 &   & [0.02 0.4] & [0.05 0.96] \\ 
Forster, Mohan, \& Hector (2003), exp. 1 & English & 24 & 60 & 63 & 60 & -3 &   & [0.02 0.4] & [0.05 0.96] \\ 
Kinoshita (2006), exp. 1 & English & 24 & 53 & 32 & 38 & 6 &   & [0.02 0.4] & [0.05 0.96] \\ 
Kinoshita (2006), exp. 2 & English & 24 & 53 & 29 & 59 & 30 & * & [0.02 0.4] & [0.05 0.96] \\ 
Norris \& Kinoshita (2008), exp. 1 & English & 24 & 53 & 35 & 66 & 31 & * & [0.02 0.4] & [0.05 0.96] \\ 
Forster, Davis, Schoknecht, \& Carter (1987), exp. 4 & English & 27 & 60 & 34 & 25 & -9 &   & [0.03 0.46] & [0.05 0.98] \\ 
Forster \& Davis (1984), exp. 1 & English & 28 & 60 & 45 & 38 & -7 &   & [0.03 0.48] & [0.06 0.98] \\ 
Nievas (2010), exp. 1b & Spanish & 30 & 50 & 44 & 65 & 21 & * & [0.03 0.52] & [0.06 0.99] \\ 
Nievas (2010), exp. 2a & Spanish & 30 & 50 or 33\textsuperscript{\textit{1}} & 51 & 58 & 7 &   & [0.03 0.52] & [0.06 0.99] \\ 
Segui \& Grainger (1990), exp. 4 & French & 36 & 60 & 42 & 45 & 3 &   & [0.03 0.63] & [0.07 1] \\ 
Bodner \& Masson (2001), exps. 2A, 2B, 3, \& 6 (average)\textsuperscript{\textit{2}} & English & 40 & 60 & 37 & 69 & 32 & * & [0.03 0.68] & [0.08 1] \\ 
Rajaram \& Neely (1992), exp. 1 & English & 48 & 50 & 30 & 37 & 7 &   & [0.04 0.76] & [0.09 1] \\ 
Rajaram \& Neely (1992), exp. 2 & English & 48 & 50 & 45 & 78 & 33 &   & [0.04 0.76] & [0.09 1] \\ 
\midrule 
\midrule 
Mean &   &   &   & 41 & 55 & 13 &   &   &   \\ 
SD &   &   &   & 10 & 14 & 13 &   &   &   \\ 
Correlation &   &   &   &   &   & 0.46 &   &   &   \\ 
\bottomrule

\end{longtable}

\begin{minipage}{\linewidth}
\textsuperscript{\textit{1}}SOA for each subject determined by pre-test\\
\textsuperscript{\textit{2}}Reported in Masson \& Bodner (2003)\\
\end{minipage}

\elandscape

\section{The present study}\label{sec-study}

It is somewhat surprising that the status of the FAE in masked priming
remains largely unresolved in the literature, given its non-negligible
magnitude and its theoretical significance in elucidating the underlying
cognitive processes of masked priming.

One possible interpretation of the conflicting past findings revolves
around the fact that only 4 out of 18 studies demonstrate a
statistically significant FAE in masked repetition priming. Notably,
this number potentially diminishes further when considering that, among
these four studies, the FAE is detected only through the pooling of data
across multiple studies employing a unique alternating-case stimulus
presentation (Bodner and Masson 2001; Masson and Bodner 2003). This line
of reasoning suggests a qualitatively distinct profile between masked
and unmasked repetition priming, with the FAE more firmly established in
the latter.

Conversely, one could argue that 15 out of 18 studies exhibit
numerically larger repetition effect sizes for low-frequency words
compared to high-frequency words ---- a pattern that is challenging to
reconcile with a genuine absence of interaction between frequency and
masked repetition. Additionally, the average FAE across all studies
stands at 13 ms, a modest yet non-negligible effect size. In fact, the
naïve assumption that the two conditions are similar enough across
experiments could justify the use of a \emph{t}-test with statistically
significant results: \emph{M\_FAE} = 13, CI\_95\% = {[}7, 20{]}),
\emph{t}(17) = 4.24, \(p=.0005\). These considerations suggest that a
genuine FAE may exist in masked priming but might be smaller than the
thresholds detectable by most previous experiments. This interpretation
is supported by the results from Adelman et al. (2014) in a large scale,
multi-site lab-based study on orthographic priming. They report a small
but reliable FAE, but only when the frequency counts of Brysbaert and
New (2009) are used, and caution that this effect could simply be an
orthographic neighborhood effect masquerading as a frequency effect, due
to the high correlations between the two variables.

Compounding this complexity, another potential contributor to past
discrepancies is the reliance on the dated Kucera and Francis (1967)
word frequency database, which 15 out of 18 studies have depended on.
This poses a potential problem, as this frequency database has
consistently demonstrated inferior predictive performance, particularly
with low-frequency words, compared to more contemporary databases
(Burgess and Livesay 1998; Zevin and Seidenberg 2002; Balota et al.
2004; Brysbaert and New 2009; Yap and Balota 2009; Brysbaert and Cortese
2011; Gimenes and New 2016; Herdağdelen and Marelli 2017; Brysbaert,
Mandera, and Keuleers 2018). Both of these issues are addressed in the
subsequent sections.

\subsection{Issues with frequency databases}\label{sec-study-freq}

Due to the well-documented concerns over the reliability of the Kučera
and Francis (1967) frequency database for psycholinguistic experiments
(Burgess and Livesay 1998; Zevin and Seidenberg 2002; Balota et al.
2004; Brysbaert and New 2009; Yap and Balota 2009; Brysbaert and Cortese
2011; Gimenes and New 2016; Herdağdelen and Marelli 2017; Brysbaert,
Mandera, and Keuleers 2018), our studies exclusively sourced materials
from the HAL (Lund and Burgess 1996) and SUBTLEX\(_{US}\) (Brysbaert and
New 2009) databases, which reflect more recent linguistic usage and
offer better validation in behavioral experiments (e.g., Balota et al.
2004; Brysbaert and New 2009; Yap and Balota 2009; Brysbaert and Cortese
2011; Gimenes and New 2016; Herdağdelen and Marelli 2017). While these
databases outperform Kučera and Francis (1967) in predicting
psycholinguistic task outcomes, it is important to note potential
discrepancies in individual frequency counts, particularly in the low
and mid-frequency ranges. This variation, attributable to the primary
genre of their sources (USENET groups for HAL and movie subtitles for
SUBTLEX\(_{US}\)),\footnote{A separate, though relevant issue which
  cannot be addressed here is to how to mitigate the discrepancies
  across the databases available, but see Yap and Balota (2009), and
  Brysbaert and Cortese (2011) for proposals about combining the
  frequency counts from different corpora.\label{fn-databases}} may have
minimal impact on megastudies with large word samples (e.g., Balota et
al. 2004; Brysbaert and New 2009; Yap and Balota 2009; Brysbaert and
Cortese 2011; Gimenes and New 2016; Herdağdelen and Marelli 2017).
However, corpus-specific frequency skew can become significant when
dealing with smaller samples of words, as is the case in most masked
priming studies (cf. Adelman et al. (2014)). Table~\ref{tbl-exFreqSkew}
illustrates the potential discrepancy in considering words as high or
low frequency based on the different aforementioned databases.

\begin{longtable}{lrrr}

\caption{\label{tbl-exFreqSkew}Example of frequency count imbalances (in
occurrences per million) across the frequency norms of Kucera \& Francis
(KF), HAL and SUBTLEX\textsubscript{US} for 4 to 6 letter words.}

\tabularnewline

\toprule
Word & KF & HAL & SUBTLEX\textsubscript{US} \\ 
\midrule\addlinespace[2.5pt]
\multicolumn{4}{l}{\emph{Skew in KF}} \\ 
\midrule\addlinespace[2.5pt]
negro & $104$ & $3$ & $5$ \\ 
poet & $99$ & $9$ & $9$ \\ 
mercer & $71$ & $4$ & $2$ \\ 
swung & $48$ & $3$ & $2$ \\ 
mantle & $48$ & $8$ & $2$ \\ 
\midrule\addlinespace[2.5pt]
\multicolumn{4}{l}{\emph{Skew in HAL}} \\ 
\midrule\addlinespace[2.5pt]
web & $6$ & $351$ & $9$ \\ 
user & $4$ & $297$ & $2$ \\ 
mint & $7$ & $211$ & $5$ \\ 
format & $9$ & $198$ & $1$ \\ 
warp & $4$ & $125$ & $5$ \\ 
\midrule\addlinespace[2.5pt]
\multicolumn{4}{l}{\emph{Skew in SUBTLEX\textsubscript{US}}} \\ 
\midrule\addlinespace[2.5pt]
daddy & $4$ & $16$ & $185$ \\ 
bitch & $6$ & $24$ & $169$ \\ 
cute & $5$ & $28$ & $88$ \\ 
pardon & $8$ & $12$ & $65$ \\ 
steal & $5$ & $28$ & $53$ \\ 
\bottomrule

\end{longtable}

\subsection{Issues with statistical power}\label{sec-study-power}

The inconsistency of past findings regarding the FAE in masked priming
has been linked to a potential lack of statistical power in previous
research (Bodner and Masson 1997, 2001; Masson and Bodner 2003; Adelman
et al. 2014). This is a plausible concern, as interactions like the FAE
often require larger sample sizes for statistical detection (Potvin and
Schutz 2000; Brysbaert and Stevens 2018) compared to main effects. We
outline below three ways in which neglecting statistical power might
frustrate our understanding of FAE in masked repetition priming.

First, our literature review revealed crucial gaps in the reporting of
relevant statistical information, which impedes the assessment of the
statistical power attained by past experiments. The inconsistent
reporting of each conditions' standard deviations (only 7 out of 18
studies) and the complete absence of reporting of the correlation
structure between conditions complicates power assessments. Researchers
are thus forced to explore a range of plausible values for standard
deviations and correlation structures on their own.

Table~\ref{tbl-litReview} details our attempt to conduct power
simulations for two hypothesized frequency attenuation effect sizes: 15
ms (close to the averaged FAE of 13 ms) and 30 ms (close to the only
three observed statistically significant FAE in English). Standard
deviations (ranging between 60 ms and 180 ms, in 10 ms increments) and
correlation between conditions (uniformly set to range between 0.6 and
0.9, with 0.1 unit increments) were simulated for each study's sample
size, with 10,000 replications for each simulation. These range of
values were derived from our literature review and previous in lab and
online experiments (Petrosino 2020; Petrosino, Sprouse, and Almeida
2023). For each simulated dataset, a paired \emph{t}-test was performed
comparing the repetition effect for high frequency words and low
frequency words. This calculation is mathematically identical to the
interaction term in a 2x2 factorial within-subjects design (the squared
\emph{t} value is equal to the \emph{F} value for the interaction
calculated in the 2x2 rmANOVA), but it is less computationally expensive
to perform in large scale simulations. Power to detect this interaction
was then calculated as the proportion of significant tests obtained
across replications. All else being equal, standard deviations and
correlations between conditions have opposite effects on statistical
power: increases in standard deviations lead to less power, while
increases in correlation between conditions lead to more power.

The results reported in Table~\ref{tbl-litReview} reveal a wide range of
possible statistical power attained by previous studies, depending
solely on the combination of plausible standard deviation and
correlation across conditions. For instance, the study with the smallest
sample size (Forster et al. 1987, \(N=16\)) had a 2\% to 24\% chance of
detecting a 15 ms frequency attenuation effect and a 4\% to 84\% chance
to detect a 30 ms effect. Similarly, the study with the largest sample
size (Rajaram and Neely 1992, \(N=48\)) exhibited a range of 4\% to 76\%
for a 15 ms frequency attenuation effect and 9\% to 100\% for a 30 ms
effect. As a consequence of the limited reporting of relevant
statistical information in past studies, it is nearly impossible to
determine if any of them were adequately powered to detect the effect of
interest.

A second concern arising from the ambiguity surrounding statistical
power in the literature is the potential impact of a prevalence of
low-powered experiments on the scientific record. An excess of such
experiments increases the risk of observed statistically significant
effects being spurious (Button et al. 2013). As highlighted in
Table~\ref{tbl-litReview}, only 4 out of 18 studies demonstrate a
statistically significant FAE. The absence of clarity regarding the
statistical power of previous research poses challenges in assessing the
likelihood of these significant findings being spurious.

Finally, it is widely acknowledged that experiments with approximately
50\% power are akin to a coin toss in their ability to detect a true
effect (Cohen 1992). A less-appreciated fact is that, in the presence of
even lower power (\textless25\%), statistically significant results can
substantially overestimate the effect size -- a type-M error (Gelman and
Carlin 2014). When power drops to levels below 10\%, a statistically
significant result may occur even when the observed effect goes in the
opposite direction of the true effect -- a type-S error (Gelman and
Carlin 2014). Our power simulations for within-subjects data revealed a
similar relationship between statistical power, type-M, and type-S
errors in line with the observations detailed by Gelman and Carlin
(2014) for the independent samples \(t\)-test. For instance, at 10\%
power (a possibility for virtually all previous studies, as indicated in
Table~\ref{tbl-litReview}), a statistically significant result could
indicate an overestimation of the magnitude of the frequency attenuation
effect by a factor between 2 and 5, with a 5\% chance of incorrectly
determining the direction of the effect.

\begin{figure}[H]

\centering{

\includegraphics{index_files/figure-latex/notebooks-intro_lit-review_power-analysis-fig-power-output-1.png}

}

\caption{\label{fig-power}Power simulations for a FAE = 10 ms, for all
combinations of standard deviation (sd), correlation (cor), and sample
size. The red line identifies the threshold of 80\% power.}

\end{figure}%

The two studies reported here were designed to mitigate these two
confounding issues: the overreliance on the Kučera and Francis (1967)
frequency data as well as a potential lack of statistical power observed
in previous research. As a large increase in statistical power requires
a large sample size, Experiment 1 aimed to assess the suitability of
using \emph{Labvanced} (Finger et al. 2017), an online platform for
running web browser-based experiments, for running masked priming
studies online.

\section{Experiment 1}\label{sec-exp1}

As evident in Table~\ref{tbl-litReview}, conducting a properly powered
experiment for a FAE close to the averaged value calculated from
previous studies will require sample sizes that would be impractical to
pursue in standard university research settings, typically quiet lab
rooms with limited research computers. In response to this challenge,
our study was exclusively conducted online, leveraging the growing trend
in online behavioral research facilitated by HTML5 capabilities and the
availability of advanced web software such as \emph{jsPsych} (de Leeuw
2014), \emph{PsychoJS} (the JavaScript counterpart of \emph{PsychoPy},
Peirce et al. (2019)), \emph{Gorilla} (Anwyl-Irvine et al. 2020), and
\emph{Labvanced} (Finger et al. 2017).

Notably, three recent studies have already demonstrated the viability of
conducting masked priming experiments online, employing different
software tools: Angele et al. (2023) with \emph{PsychoJS}, Cayado, Wray,
and Stockall (2023) with \emph{Gorilla} and Petrosino, Sprouse, and
Almeida (2023) with \emph{Labvanced}. In this study, we opted for
\emph{Labvanced} (Finger et al. 2017), given our previous successful
experience with it (Petrosino, Sprouse, and Almeida 2023). Similar to
\emph{Gorilla}, \emph{Labvanced} eliminates local installation issues,
ensuring cross-platform consistency and simplifying experimental design
without necessitating proficiency in additional programming languages.

\subsection{Methods}\label{sec-exp1-methods}

\subsubsection{Participants}\label{sec-exp1-methods-participants}

Three hundred participants (145 females; \emph{mean age} = 38; \emph{sd}
= 12) were recruited on the Prolific online platform
(\url{https://www.prolific.com}). Several criteria were selected to
ensure recruitment of native speakers of U.S. English. Participants had
to be born in the Unites States of America, speak English as their first
and only language, and have no self-reported language-related disorder.
We encouraged participants to avoid any sort of distraction throughout
the experiment, and to close any program that may be running in the
background. Because the experiment was run online, participants could
not be monitored during data collection. Finally, to further reduce
variability across participants' devices, we restricted the experiment
to be run on Google Chrome only, which is the most used browser
worldwide (W3 Counter 2023), and reportedly performs better than any
other across operating systems (likely thanks to the \emph{Blink}
engine; see Lukács and Gartus 2023).

\subsubsection{Design}\label{sec-exp1-methods-design}

The masked priming procedured relied on a lexical decision task (LDT),
in which a 2 (Frequency: \emph{high} vs \emph{low}) x 2 (prime type:
\emph{repetition} vs \emph{unrelated}) factorial design was used. Both
factors were manipulated within-subjects. The dependent variables were
lexical decision latency (in miliseconds) and error rate (in
percentages).

\subsubsection{Materials}\label{sec-exp1-methods-materials}

Two hundred five-letter English words were selected from the English
Lexicon Project (ELP; Balota et al. 2007), in which 100 words were
selected from an upper and a lower frequency range,
respectively.\footnote{The experiment also included an even lower
  frequency condition (range: {[}3.0 5.01{]}; mean: 4.39, SD: 0.50),
  thus summing up to six hundred trials being presented in the
  experiment. However, the average error rate for this condition was
  44\% and 33 (out of the 50) target words used in the same condition
  had a error rate higher than 30\%. This suggested that they might have
  not known these words (see Kinoshita (2006)). For this reason, this
  condition was completely removed from analysis.} It was not possible
to identify two frequency ranges that were well separated from one
another for both the HAL (Lund and Burgess 1996) and the
SUBTLEX\(_{US}\) (Brysbaert and New 2009) frequency databases. As
Table~\ref{tbl-words_exp1} shows, we managed to do this only for the
former, whereas some overlap was present in the latter. This is expected
given the different source of the two databases (see above, and fn.
\ref{fn-databases}). The two word subsets corresponded to the two word
frequency conditions being tested: the high-frequency, and low-frequency
conditions. In each condition, fifty words were randomly chosen to be
presented as targets and related primes (for the related prime type
condition), and the remaining fifty were presented as unrelated primes
(for the unrelated prime type condition).

\begin{longtable}{crrrrrrrrr}

\caption{\label{tbl-words_exp1}Experiment 1. Descriptive statistics of
the word item used. For both frequency databases, the word frequencies
were converted to per-million count to ensure cross-comparison.}

\tabularnewline

\toprule
 &  & \multicolumn{4}{c}{\textbf{HAL}} & \multicolumn{4}{c}{\textbf{SUBTLEX\textsubscript{US}}} \\ 
\cmidrule(lr){3-6} \cmidrule(lr){7-10}
frequency & N & min & max & mean & SD & min & max & mean & SD \\ 
\midrule\addlinespace[2.5pt]
high & 100 & 169 & 1212 & 482 & 292 & 2.00 & 1168 & 129 & 201 \\ 
low & 100 & 3 & 23 & 9 & 5 & 0.12 & 13 & 3 & 3 \\ 
\bottomrule

\end{longtable}

Two-hundred five-letter phonotactically legal nonwords were randomly
selected from the ELP database as well. Half of them were randomly
selected to be presented as targets; the other half was instead used as
unrelated nonword primes.

\subsubsection{Procedure}\label{sec-exp1-methods-proc}

Each recruited participant was assigned one of two word lists, which
differed only in the relatedness of the prime with respect to the
target; otherwise, the two lists presented the same set of target words
and nonwords (300 items in total). In one list, the three conditions
(high-frequency, low-frequency word conditions, and the non-word
condition) had 25 target items being preceded by themselves (the
\emph{related} condition) and the remaining 25 target items being
preceded by one of the unrelated primes belonging to the same frequency
bin (the \emph{unrelated} condition). In the other list, these
assignments were reversed. The order of stimulus presentation was
randomized for each participant.

After being recruited in the \emph{Prolific} online platform,
participants were asked to click on a link redirecting them to the
Labvanced online service. During the experiment, they were asked to
perform a lexical decision task by pressing either the `J' (for word) or
`F' (for non-word) keys on their keyboard. Each trial consisted of three
different stimuli appearing at the center of the screen: a series of
hashes (\#\#\#\#\#) presented for 500 ms, followed by a prime word
presented for 33 ms, and finally the target word; the target word
disappeared from the screen as soon as a decision was made. The
motivation for the choice of a very short prime duration (as compared to
the literature, in which it is usually between 50 and 60 ms; see
Table~\ref{tbl-litReview}) is threefold. First, previous experiments on
\emph{Labvanced} (e.g., Petrosino, Sprouse, and Almeida 2023) showed
that, due to the inherent difficulties in presenting stimuli for very
short set durations in the browser, a longer set duration would increase
the number of trials in which the prime duration would rise above the
subliminal threshold (usually set at 60 ms) due to timing inaccuracies
and missing screen refreshes, which could trigger the adoption of
experiment-wide strategies in the task, and ultimately contaminate the
masked priming response (Zimmerman and Gomez 2012). Second, Angele et
al. (2023), Cayado, Wray, and Stockall (2023) and Petrosino, Sprouse,
and Almeida (2023) have demonstrated that a 33 ms priming duration is
sufficient to elicit repetition priming effects in online experiments.
Finally, setting such a short prime duration prevents virtually everyone
from consciously perceiving the prime word Nievas (2010), and thus
presents a less contaminated estimate of early putatively automatic
processes in word recognition.

Participants were given 5 breaks throughout the experiment. When the
experiment was over, the participants were then redirected to Prolific
in order to validate their submission. The median time to finish the
experiment was 11 minutes. Each participant was paid with a standard
rate of GBP 9/hour.

\subsection{Data analysis}\label{sec-exp1-analysis}

Analysis scripts and an abridged version of the data collected can be
found on online (\url{https://osf.io/ej8dh}). We performed three
different steps of analyses (in sequential order), with the goal of only
keeping data that pass a set of stringent inclusion criteria (77,359
observations in total). After removing participants and items with high
error rates, we inspected the durations of prime stimuli and removed
those that did not fell within our desired range. Finally, we removed RT
outliers.

\subsubsection{Step 1: subject and item
performance}\label{sec-exp1-analysis-performance}

Item and subject error rates were calculated, with a cutoff of 30\%.
Only 3 low-frequency words (\emph{carte, parse, posit}), 5 non-words
(**), and 8 participants were removed, with 291 participants remaining.

\subsubsection{Step 2: prime
durations}\label{sec-exp1-analysis-primeTime}

During the experiment, the duration of presentation of the prime word
was recorded for every trial. Both the mean (mean = 37.88 ms) and the
median (median = 35 ms) of prime durations were slightly larger than the
intended value (33 ms). This distribution suggests some imprecision in
prime duration during the experiment. This was expected and likely due
to the inherent difficulty with timing precision of visual presentations
in web browsers and the great variation of computer hardware and
internet connections used by the participants. Both of these issues may
be impossible to control, at least at the current state of browser
development. However, in masked priming, in which the duration of the
prime is an essential part of the design, such fluctuations may indeed
hinder proper elicitation of the priming response. As a way to
counteract the influence that such fluctuations might have on the
priming response, we only kept trials whose prime durations were within
a pre-set range from the intended prime duration of 33 ms. Taking a
standard 60-Hz monitor as reference, the lower and the upper bounds were
set respectively at 25 ms (i.e., the intended prime duration minus half
of a full refresh cycle: \(33-8~ ms\); noting that Angele et al. (2023)
already showed that no repetition priming effects are obtained with a
16.7ms prime duration) and 60 ms (i.e., the commonly accepted upper
threshold of subliminal processing), in an attempt to remove any trial
that could have been consciously perceived by participants. Only 4\% of
the trials were out of this duration range. We take this as evidence
that \emph{Labvanced} is able to consistently present stimuli at short
durations. Prime duration fluctuations were however observed, and they
were likely due to external factors outside of experimenter control
(such as computer hardware, internet connection speed, and number of
active operations in the background). The out-of-range trial removal was
performed on the data after the error rate removal procedure. A total of
291 participants and 67,209 observations were included in the next steps
of analysis.

\subsubsection{Step 3: RT distribution}\label{sec-exp1-analysis-RT}

Finally, individual trials were excluded if their RT was below 200 ms
and 1800 ms. 602 observations were excluded at this stage of analysis
(i.e., -98.1\% of the dataset). After removing incorrect trials, to
ensure more accurate estimates, we also made sure that each condition
for each each participant contained at least half of the total number of
trials presented (i.e., 12). A total of 61,449 observations and 282
subjects were included in the statistical analysis below.

\subsection{Results}\label{sec-exp1-results}

For each frequency bin, priming effects were calculated for each subject
by subtracting the subject's mean RT to the related condition from the
subject's mean RT to the unrelated condition. Unstandardized (in ms) and
standardized effect sizes (i.e., Cohen's \emph{d}) were then calculated
for each condition. Table~\ref{tbl-exp1-statsResults} below reports the
descriptive and inferential statistics of the experiment. Both frequency
conditions show statistically significant repetition priming effects
(\emph{MOP\_HF} = 23, CI\_95\% = {[}19, 27{]}, \emph{t}(281) = 10.4,
\(p<.0001\); \emph{MOP\_LF} = 30, CI\_95\% = {[}24, 36{]}, \emph{t}(281)
= 9.75, \(p<.0001\)). Non-word repetition priming effects were
inhibitory and were marginally statistically significant (\emph{MOP\_} =
-4, CI\_95\% = {[}-8, 0{]}, \emph{t}(281) = -1.91, \(p=0.057\)).
Finally, the low-frequency repetition priming effect was 7-ms larger
than that of the high-frequency words, but this FAE effect was only
marginally statistically significant (\emph{M\_FAE} = 7, CI\_95\% =
{[}-1, 15{]}), \emph{t}(281) = 1.88, \(p=0.06\)). As for the error
analysis, we found a significant priming effect in all conditions (high:
\emph{t}(281)=2.51, \(p<.0001\); low: \emph{t}(281)=6.39, \(p<.0001\);
non-word: \emph{t}(281)=-2.24, \(p<.0001\)).

\blandscape

\begin{longtable}{lrrrrrrrrlrrrrl}

\caption{\label{tbl-exp1-statsResults}Experiment 1. Summary of the word
priming results. \emph{Legend.} MOP: magnitude of priming.}

\tabularnewline

\toprule
 & \multicolumn{3}{c}{unrelated RT} & \multicolumn{3}{c}{repetition RT} &  & \multicolumn{4}{c}{priming effects} & \multicolumn{3}{c}{\emph{t}-test} \\ 
\cmidrule(lr){2-4} \cmidrule(lr){5-7} \cmidrule(lr){9-12} \cmidrule(lr){13-15}
factor & mean & SD & Error (\%) & mean & SD & Error (\%) & cor & MOP & 95\% CI & SD\textsubscript{p} & ES & \emph{t} & df & \emph{p} \\ 
\midrule\addlinespace[2.5pt]
high & 619 & 77 & 2 & 596 & 80 & 1 & 0.89 & 23 & [19 27] & 37 & 0.62 & 10.4 & 281 & 8.78e-22 \\ 
low & 699 & 93 & 10 & 669 & 91 & 7 & 0.84 & 30 & [24 36] & 52 & 0.58 & 9.75 & 281 & 1.51e-19 \\ 
non-word & 712 & 110 & 6 & 716 & 110 & 6 & 0.96 & -4 & [-8 0] & 31 & -0.11 & -1.91 & 281 & 0.0567 \\ 
frequency:primetype &   &   &   &   &   &   & -0.01 & 7 & [-1 15] & 64 & 0.11 & 1.88 & 281 & 0.0616 \\ 
\bottomrule

\end{longtable}

\elandscape

\subsection{Discussion}\label{sec-exp1-discussion}

The primary objective of Experiment 1 was to evaluate whether web
browser-based stimulus delivery programs such as \emph{Labvanced} can
yield data comparable in quality to traditional lab-based experiments.
The results indicate that this is indeed possible, but careful
inspection of prime durations is nonetheless necessary.

Robust repetition priming was observed in both frequency conditions. The
non-word condition triggered a small inhibitory repetition effect, in
line with the previous literature Forster (1999), but this was not
statistically significant. Crucially, we observed a 7 ms FAE effect that
was marginally statistically significant. As noted elsewhere (Potvin and
Schutz 2000), the absence of a significant interaction effect may easily
arise due to low statistical power.

The 95\% CI for the FAE was between -1 ms and 15 ms. This interval
suggests that the actual FAE is possibly a positive value that can be as
large as 15 ms. This is in line with the results from previous
literature, with the caveat that the majority of previous experiments
used \textasciitilde50 ms prime durations, while experiment 1 used a 33
ms prime duration. Prime durations have been suggested to be an upper
bound on the size of the masked repetition priming effect (Forster
1998), and thus it is not entirely clear how much the FAE should vary as
a function of the prime duration.

To address the concerns about the lack of statistical power and the
substantial imprecision in the estimated FAE size observed in experiment
1, experiment 2 was designed to have a sample size that ensures
acceptable statistical power to detect the an interaction between
priming and frequency, as well as a sample size that reduces the width
of the resulting confidence interval compared to experiment 1.

\section{Experiment 2}\label{experiment-2}

The findings from Experiment 1, as well as those reported by Angele et
al. (2023), Cayado, Wray, and Stockall (2023) and Petrosino, Sprouse,
and Almeida (2023), establish the feasibility of obtaining masked
repetition priming in online experiments with a 33 ms prime duration.
However, a crucial question remains: can we reliably detect the FAE in
web browser-based settings? Experiment 2 directly addresses concerns
about the potential statistical power limitations observed in Experiment
1 and much of the prior literature. Specifically targeting what we
construe as the smallest theoretically interesting FAE (5ms), we
recruited a larger sample size, as determined by a power analysis. We
simulated 10,000 datasets for each of the combinations of two
statistical parameters: standard deviation and the correlation between
conditions. The latter were kept equal across conditions to simplify the
simulations. Based on our own pilot studies and previous published work
(Petrosino 2020; Petrosino, Sprouse, and Almeida 2023), the simulations
involved standard deviations ranging from 80 to 120 ms (with 10 ms
increments), while the correlation between conditions ranged from 0.7 to
0.9 (with 0.1 increments). The sample size varied between 200 and 3,000
participants (with 150 unit increments). Three different FAE sizes were
simulated: 15 ms, 10 ms and 5 ms. The first effect size (15 ms) is about
half of the ones observed in previous studies that statistically
detected the FAE (\textasciitilde30 ms). The second effect size (10 ms)
is close to the size of the average frequency attenuation effect found
in the literature (13 ms). The last effect size (5 ms) is our
lower-bound estimate of a theoretically interesting effect size. The
code used for the power simulations, along with the simulated datasets
are available online (\url{https://osf.io/r7d2q/}).

Our analysis identified a sample size of 1,250 participants as optimal,
ensuring robust statistical power (\textgreater{} 80\%) across various
parameter combinations (Figure~\ref{fig-power-1250}), especially for raw
FAEs equal to or exceeding 10 ms ---- a value closely aligned with the
average FAE calculated from previous studies (refer to
Table~\ref{tbl-litReview}). In light of the observed limitations in the
temporal accuracy and precision of current online stimulus delivery
programs (discussed in Section~\ref{sec-exp1-analysis-primeTime}), which
necessitated substantial subject and data exclusion in Experiment 1, we
aimed for an intended sample size of 2,600. This decision was made to
enhance the likelihood of obtaining a sample size of at least 1,250
participants after applying all the necessary exclusion criteria to the
data. In addition, sample sizes exceeding 1,250 can only help increase
the precision of the estimated effect size.

\phantomsection\label{cell-fig-power-1250}
\begin{figure}[H]

\centering{

\includegraphics{index_files/figure-pdf/fig-power-1250-1.pdf}

}

\caption{\label{fig-power-1250}Power simulations with a sample size of
1,250, for all combinations of standard deviation (sd), pairwise
correlation (cor), and interaction effect size. The red line identifies
the threshold of 80\% power.}

\end{figure}%

\subsection{Methods}\label{sec-exp2-methods}

\subsubsection{Preregistration}\label{sec-exp2-prereg}

We preregistered the results of the power analysis, the goals, the
design and analysis plan for experiment 2 prior to data collection. The
preregistration, detailing the experimental hypotheses, the desired
sample size as well as the planned analyses is available online
(\url{https://doi.org/10.17605/OSF.IO/3NFQP}).

\subsubsection{Participants}\label{sec-exp2-methods-participants}

Two thousand and six hundred participants (1445 females; \emph{mean age}
= 42, \emph{sd age} = 14) were recruited on Prolific
(\url{https://www.prolific.com}) with the same criteria specified for
experiment 1 (Section~\ref{sec-exp1-methods-participants}).

\subsubsection{Design}\label{sec-exp2-methods-design}

The experimental design was identical to experiment 1.

\subsubsection{Materials}\label{sec-exp2-methods-materials}

One-hundred and four five-letter words, half of low frequency (between 7
and 24 in the SUBTLEX\(_{US}\) frequency per million) and half of high
frequency (between 57 and 2,961 in the SUBTLEX\(_{US}\) frequency per
million) were sampled from ELP (Balota et al. 2007), but this time based
on the SUBTLEX\(_{US}\) frequency counts rather than HAL (which was used
in experiment 1). Table~\ref{tbl-words_exp2} shows that although the
SUBTLEX\(_{US}\) frequency ranges of the two conditions were very far
from one another (similarly to what was done in Experiment 1;
Section~\ref{sec-exp1-methods-materials}), they still show some overlap
when HAL frequencies are used. As mentioned before, this seems to be a
general problem when jointly considering different frequency databases
for a smaller set of stimuli that need to be manipulated and controlled
in different ways (see also fn. \ref{fn-databases} and Adelman et al.
(2014)). From each condition, fifty words were randomly chosen to be
presented as targets and related primes (the \emph{related} condition),
and the remaining fifty were presented as unrelated primes (the
\emph{unrelated} condition). All words used were monomorphemic nouns,
adjectives, or verbs, thus excluding particles, prepositions, and
derived or inflected forms.

\begin{longtable}{lrrrrrrrrr}

\caption{\label{tbl-words_exp2}Experiment 2. Descriptive statistics of
the word items used. For both frequency databases, the word frequencies
were converted to per-million count to ensure cross-comparison.}

\tabularnewline

\toprule
 &  & \multicolumn{4}{c}{\textbf{HAL}} & \multicolumn{4}{c}{\textbf{SUBTLEX\textsubscript{US}}} \\ 
\cmidrule(lr){3-6} \cmidrule(lr){7-10}
frequency & N & min & max & mean & SD & min & max & mean & SD \\ 
\midrule\addlinespace[2.5pt]
high & 52 & 45 & 4984 & 573 & 808 & 57 & 2691 & 210 & 388 \\ 
low & 52 & 6 & 570 & 64 & 93 & 7 & 24 & 13 & 5 \\ 
\bottomrule

\end{longtable}

One-hundred and four five-letter, phonotactically legal nonwords were
randomly selected from the ELP database as well. Half of them were
randomly selected to be presented as targets; the other half was instead
used as unrelated nonword primes. None of the nonwords contained any
existing English morpheme. Both the words and non-words used in the
experiments are reported in the appendix below.

\subsubsection{Procedure}\label{sec-exp2-methods-proc}

Experiment 2 followed the same procedures as experiment 1 (see
Section~\ref{sec-exp1-methods-proc}). The median time to finish the
experiment was 5 minutes.

\subsection{Data analysis}\label{sec-exp2-analysis}

Analysis scripts and an abridged version of the data collected can be
found online (\url{https://osf.io/vn3r2}), and consisted of 297,598
observations in total. We performed the same three steps of analysis
described for experiment 1 (Section~\ref{sec-exp1-analysis}).

\subsubsection{Step 1: subject and item
performance}\label{sec-exp2-analysis-performance}

Item and subject error rates were calculated. The item error rate was
never below above 14\%, so no item was excluded from analysis. 19
subjects were removed because their error rate was above 30\%. Thus, a
total of 269,652 observations and 2,593 participants were included in
further analyses.

\subsubsection{Step 2: prime
durations}\label{sec-exp2-analysis-primeTime}

Prime fluctuations were dealt with in the same way as in experiment 1
(Section~\ref{sec-exp1-analysis-primeTime}). The mean (mean = 32.32 ms,
sd = 15) and the median (median = 33 ms) prime durations were closer to
the intended value (33 ms). The same prime duration cut-off set for
experiment 1 (i.e., any trial whose prime duration was out of the
25-60ms range) removed 13 \% of the trials. No participant was excluded,
for a total of 237,287 observations.

\subsubsection{Step 3: RT distribution}\label{sec-exp2-analysis-RT}

After removing the incorrect responses, similarly to what we did for
experiment 1 (Section~\ref{sec-exp1-analysis-RT}), 0.51\% of the trials
were excluded if the relative RT was below 200 ms and above 1800 ms.
Finally, 249 subjects were removed because the number of trials within
the same condition was less than 7 (i.e., about half of the total number
of trials being presented within the same condition, i.e.~13). A total
of 210,889 observations and 2,341 subjects were included in the
statistical analysis below.

\subsection{Results}\label{sec-exp2-results}

For each frequency condition, priming effects were calculated in the
same way as experiment 1. Table~\ref{tbl-exp2-statsResults} below report
the descriptive statistics of the experiment. All three conditions
showed statistically significant repetition priming effects
(\emph{MOP\_HF} = 18, CI\_95\% = {[}16 20{]}, t(2340) = 19.7,
\(p<.0001\); \emph{MOP\_LF} = 28, CI\_95\% = {[}26 30{]}, \emph{t}(2340)
= 27.8, \(p<.0001\); \emph{MOP\_NW} = -2, CI\_95\% = {[}-4 0{]}, t(2340)
= -2.33, \(p<.0001\)). The low-frequency word repetition priming effect
was 10 ms larger than the high-frequency word repetition priming effect,
and this FAE effect was statistically significant (\emph{M\_FAE} = 10,
CI\_95\% = {[}7 13{]}), \emph{t}(2340) = 7.24, \(p<.0001\)). As for the
word error analysis, we found significant priming effects in the word
conditions (high: \emph{t}(2340)=9.95, \emph{p}\textless.0001; low:
\emph{t}(2340)=16.9, \emph{p}\textless.0001), as well as in the non-word
condition (non-word: \emph{t}(2340)=-3.27, \emph{p}=.001).

\blandscape

\begin{longtable}{lrrrrrrrrlrrrrl}

\caption{\label{tbl-exp2-statsResults}Experiment 2. Summary of the word
priming results. \emph{Legend.} MOP: magnitude of priming.}

\tabularnewline

\toprule
 & \multicolumn{3}{c}{unrelated RT} & \multicolumn{3}{c}{repetition RT} &  & \multicolumn{4}{c}{priming effects} & \multicolumn{3}{c}{\emph{t}-test} \\ 
\cmidrule(lr){2-4} \cmidrule(lr){5-7} \cmidrule(lr){9-12} \cmidrule(lr){13-15}
factor & mean & SD & Error (\%) & mean & SD & Error (\%) & cor & MOP & 95\% CI & SD\textsubscript{p} & ES & \emph{t} & df & \emph{p} \\ 
\midrule\addlinespace[2.5pt]
high & 573 & 83 & 3 & 555 & 85 & 2 & 0.860 & 18 & [16 20] & 45 & 0.41 & 19.7 & 2340 & 2.88e-80 \\ 
low & 605 & 88 & 6 & 577 & 88 & 3 & 0.850 & 28 & [26 30] & 49 & 0.58 & 27.8 & 2340 & 1.52e-147 \\ 
non-word & 623 & 103 & 4 & 625 & 103 & 4 & 0.910 & -2 & [-4 0] & 43 & -0.05 & -2.33 & 2340 & 0.0197 \\ 
frequency:primetype &   &   &   &   &   &   & 0.029 & 10 & [7 13] & 66 & 0.15 & 7.24 & 2340 & 5.86e-13 \\ 
\bottomrule

\end{longtable}

\elandscape

\subsection{Discussion}\label{sec-exp2-discussion}

Experiment 2 was designed to investigate whether Frequency Attenuation
Effects (FAE) can be detected under masked priming conditions (with SOA
\textless{} 60 ms, here 33 ms). We employed a robust sample size to
ensure adequate statistical power for detecting small to medium effect
sizes. Our results not only replicated Experiment 1 in revealing
statistically significant main effects of repetition for high and low
frequency words alike, but also detected a statistically significant
interaction: the low-frequency condition yielded priming effects that
were 10 ms larger than the high-frequency condition. This value is
within the 95\% CI from experiment 1, making it a successful replication
of that result. The 95\% CI of experiment 2 ranges from 7 ms to 13 ms.
This is notable because it includes the effect size of experiment 1, but
is also quite narrow (a halfwidth of 3 ms). This indicates that, for the
frequency ranges investigated in experiment 2, the FAE is unlikely to be
smaller than 7 ms or larger than 13 ms when the prime duration is 33 ms.
In contrast, experiment 1 had a halfwidth almost three times as large (8
ms).

The absence of a robust non-word masked priming response has been used
as an additional piece of evidence supporting the view that the masked
priming response is devoid of episodic influences (e.g., Forster 1999).
The results of experiment 2 align with the previous evidence (including
that of experiment 1) in showing at best very small inhibitory masked
repetition priming for non-words, with very high precision: the 95\% CI
indicates the plausible range for the masked repetition priming effect
for non-words to be between -4 and 0 ms when prime duration is 33 ms.

\section{General discussion}\label{sec-discussion}

The repetition priming response stands as a cornerstone in
psycholinguistic investigations, offering insights into the mechanisms
governing word recognition. An ongoing debate surrounds the
interpretation of these effects, particularly concerning their source in
the memory system. On the one hand, \emph{interactive activation models}
(McClelland and Rumelhart 1981; Grainger and Jacobs 1996; Coltheart et
al. 2001) posit a lexical source for repetition priming effects, either
in terms of temporarily raised resting activation levels for lexical
nodes in unmasked priming, or as a head start in the retrieval process
in masked priming. \emph{Episodic} and \emph{memory recruitment models}
(Jacoby and Dallas 1981; Jacoby 1983; Bodner and Masson 1997; Masson and
Bodner 2003; Bodner and Masson 2014) on the other hand, invoke a
non-lexical source for the repetition effect, namely an episodic or
episodic-like memory resource formed upon brief exposure to the prime
word that can be recruited during the processing of the target item.
Crucially, both models predict a single mechanism underlying masked and
unmasked priming. Differential mechanisms between unmasked and masked
repetition priming, however, are predicted by the \emph{entry-opening
model} (Forster and Davis 1984), which propose both lexical and episodic
sources of priming effects.

Thus, the existence of qualitatively distinct outcomes in masked and
unmasked priming presented a direct challenge to some, but not all of
these models. One such finding is the \emph{Frequency Attenuation
Effect} (FAE), in which higher frequency words exhibit smaller
repetition effects compared to lower frequency words. The FAE has been
described as observable only in unmasked priming since the work of
Forster and Davis (1984), who demonstrated that when the prime word is
presented very briefly (SOA \(<\) 60 ms), it becomes masked by the
target word, and this is hypothesized to prevent the conscious encoding
of the prime. Under such conditions, the FAE purportedly disappears.
Forster and Davis (1984) argued that this potentially shows that the FAE
is subserved by a different type of memory source (perhaps episodic)
than the masked repetition priming response. This conclusion, however,
is the source of ongoing debates (see Table~\ref{tbl-litReview} for
review of past findings), which the two experiments reported here were
meant to address.

Within this research landscape, our experiments targeting the frequency
sensitivity of the repetition effect under masked conditions contribute
methodological and theoretical insights. Methodologically, our results
help establish the viability and reliability of online data collection
for the masked priming paradigm. Building on the work of Angele et al.
(2023), Cayado, Wray, and Stockall (2023) and Petrosino, Sprouse, and
Almeida (2023), we addressed pitfalls in implementing and analyzing
masked priming data collected online, and by doing so offered a solution
to the longstanding problem of low statistical power and lack of
estimation precision when it comes to investigating phenomena with
effect sizes that are harder to detect statistically, like interactions
in factorial designs. However, this newfound opportunity necessitates
careful data scrutiny.

In the same vein, the FAEs observed in experiments 1 and 2 have
important theoretical ramifications. The historical belief in the
non-observability of FAE in masked priming primarily arose from a lack
of statistically significant results, possibly rooted in outdated
frequency corpora or inadequate statistical power. Our design addressed
these concerns, yielding statistically significant FAE results aligning
with the literature's average effect (see Table~\ref{tbl-litReview}; the
95\% CI implies that the FAE is unlikely to be larger than 13 ms with a
33 ms prime duration). These results challenge the supposed qualitative
distinction between masked and unmasked repetition priming cleaved by
the FAE, complicating the rejection of single-mechanism theories, and
suggesting that \emph{interactive-activation models} and \emph{memory
recruitment models} may yet offer unifying explanations for masked and
unmasked priming.

Similarly, our results also challenge the entry-opening model's
prediction of the absence of FAE in masked priming. One potential way of
dealing with this in the \emph{entry opening model} is to claim that
masked priming severely reduces, but does not entirely eliminate, the
use of sources other than lexical memory (see Forster 1998; Forster,
Mohan, and Hector 2003, for proposals along this line). Alternatively,
within the entry-opening model, the results of experiment 2 may be
explained by the frequency-based mechanism occurring in the fast search
stage. A potential mechanism in this direction was already hinted at by
Forster and Davis (1984) themselves, and consists of a procedure,
whereby during the fast search stage, the entry of a prime word is
promoted to the top position of the search list. As a consequence,
low-frequency words (which are fairly low in the search list) will
benefit from such promotion procedure more than high-frequency words
(which are instead already in higher positions), thus ultimately giving
rise to the FAE.

While our findings present a compelling case for the presence of FAE in
masked priming that is seemingly parallel to the unmasked case,
questions about potential mechanistic differences persist. The larger
sample size needed for masked FAEs raises intriguing considerations
about the influence of memory sources and warrants further
investigation. For example, there is independent evidence for different
mechanisms in masked and unmasked repetition priming from RT
distributional analyses (cf. Gomez, Perea, and Ratcliff (2013)) that
suggests that repetition priming under masked conditions affect
primarily the encoding stage of the stimulus. Given that frequency is
often associated with facilitation of encoding, our results could help
support this view. Additionally, the trivially small inhibitory effect
sizes of non-word masked repetition priming in experiments 1 and 2 align
with the trend (overwhelmingly shown in the literature) that
facilitatory effect may be exclusive to unmasked designs (Forster 1998;
Forster, Mohan, and Hector 2003; but see Masson and Bodner 2003), and
suggests avenues for future exploration on large-scale.

Finally, the finding that the FAE occurs under masked priming conditions
may impact our understanding of masked morphological priming. In this
literature, there is a unresolved question about the ability of affixes
to elicit masked morphological priming results (for a review, Amenta and
Crepaldi 2012). In English, the evidence seems to indicate that only
stems, but not affixes, have the ability to prime entries across the
lexicon. This finding can and has been used to support models in which
affixes are initially stripped before stems are accessed in the lexicon
(Taft and Forster 1975; Forster and Azuma 2000; Stockall and Marantz
2006). However, stems and affixes do also have a large frequency
imbalance, with most affixes being substantially more frequent that most
stems. The observation of FAE under masked priming can provide an
alternative reason for why masked stem morphological priming is well
attested but masked affix morphological priming is not: the latter could
be due to a ceiling frequency attenuation effect. This is an intriguing
possibility that must be left for future work to explore.

In summary, our study successfully replicated and expanded upon the work
of Angele et al. (2023), Cayado, Wray, and Stockall (2023) and
Petrosino, Sprouse, and Almeida (2023), confirming the viability of
observing repetition priming effects in masked priming experiments
conducted online with a brief Stimulus Onset Asynchrony (SOA) of 33 ms.
Notably, we addressed a lingering question in the literature by
establishing the presence of the Frequency Attenuation Effect (FAE)
under masked conditions. The use of large online samples proved
instrumental in overcoming the longstanding challenge of insufficient
statistical power to detect interactions in factorial designs, which we
believe had impeded previous investigations into detecting the FAE in
masked priming.

These results not only contribute to our understanding of masked priming
but also open up intriguing avenues for further research. The ability to
harness extensive online samples provides a valuable opportunity to
explore and illuminate unresolved issues across various domains where
masked priming is a crucial research tool, underscoring the potential
for online experimentation to advance our knowledge and resolve
long-standing questions in the field.

\section*{References}\label{references}
\addcontentsline{toc}{section}{References}

\phantomsection\label{refs}
\begin{CSLReferences}{1}{0}
\bibitem[\citeproctext]{ref-Adelman2014}
Adelman, James S, Rebecca L Johnson, Samantha F McCormick, Meredith
McKague, Sachiko Kinoshita, Jeffrey S Bowers, Jason R Perry, et al.
2014. {``A Behavioral Database for Masked Form Priming.''}
\emph{Behavior Research Methods} 46: 1052--67.

\bibitem[\citeproctext]{ref-AmentaCrepaldi2012}
Amenta, Simona, and Davide Crepaldi. 2012. {``Morphological Processing
as We Know It: An Analytical Review of Morphological Effects in Visual
Word Identification.''} \emph{Frontiers in Psychology} 3: 232.

\bibitem[\citeproctext]{ref-Angele2023}
Angele, Bernhard, Ana Baciero, Pablo Gómez, and Manuel Perea. 2023.
{``Does Online Masked Priming Pass the Test? The Effects of Prime
Exposure Duration on Masked Identity Priming.''} \emph{Behavior Research
Methods} 55 (1): 151--67.
\url{https://doi.org/10.3758/s13428-021-01742-y}.

\bibitem[\citeproctext]{ref-Anwyl2020}
Anwyl-Irvine, Alexander L, Jessica Massonnié, Adam Flitton, Natasha
Kirkham, and Jo K Evershed. 2020. {``Gorilla in Our Midst: An Online
Behavioral Experiment Builder.''} \emph{Behavior Research Methods} 52:
388--407.

\bibitem[\citeproctext]{ref-Balota2004}
Balota, David A., Michael J. Cortese, Susan D. Sergent-Marshall, Daniel
H. Spieler, and Melvin J. Yap. 2004. {``Visual Word Recognition of
Single-Syllable Words.''} \emph{Journal of Experimental Psychology:
General} 133 (2): 283.

\bibitem[\citeproctext]{ref-balota2007}
Balota, David A., Melvin J. Yap, Keith A. Hutchison, Michael J. Cortese,
Brett Kessler, Bjorn Loftis, James H. Neely, Douglas L. Nelson, Greg B.
Simpson, and Rebecca Treiman. 2007. {``The English Lexicon Project.''}
\emph{Behavior Research Methods} 39 (3): 445--59.
\url{https://doi.org/10.3758/bf03193014}.

\bibitem[\citeproctext]{ref-Bodner2014}
Bodner, Glen E., and Michael E. J Masson. 2014. {``Memory Recruitment: A
Backward Idea about Masked Priming.''} In \emph{Psychology of Learning
and Motivation}, 61:179--213. Elsevier.

\bibitem[\citeproctext]{ref-BodnerMasson1997}
Bodner, Glen E., and Michael E. J. Masson. 1997. {``Masked Repetition
Priming of Words and Nonwords: Evidence for a Nonlexical Basis for
Priming.''} \emph{Journal of Memory and Language} 37 (2): 268--93.
\url{https://doi.org/10.1006/jmla.1996.2507}.

\bibitem[\citeproctext]{ref-BodnerMasson2001}
---------. 2001. {``Prime Validity Affects Masked Repetition Priming:
Evidence for an Episodic Resource Account of Priming.''} \emph{Journal
of Memory and Language} 45 (4): 616--47.
\url{https://doi.org/10.1006/jmla.2001.2791}.

\bibitem[\citeproctext]{ref-Brysbaert2011}
Brysbaert, Marc, and Michael J Cortese. 2011. {``Do the Effects of
Subjective Frequency and Age of Acquisition Survive Better Word
Frequency Norms?''} \emph{Quarterly Journal of Experimental Psychology}
64 (3): 545--59.

\bibitem[\citeproctext]{ref-Brysbaert2018}
Brysbaert, Marc, Paweł Mandera, and Emmanuel Keuleers. 2018. {``The Word
Frequency Effect in Word Processing: An Updated Review.''} \emph{Current
Directions in Psychological Science} 27 (1): 45--50.

\bibitem[\citeproctext]{ref-BrysbaertNew2009}
Brysbaert, Marc, and Boris New. 2009. {``Moving Beyond Ku{č}era and
Francis: A Critical Evaluation of Current Word Frequency Norms and the
Introduction of a New and Improved Word Frequency Measure for American
English.''} \emph{Behavior Research Methods} 41 (4): 977--90.
\url{https://doi.org/10.3758/brm.41.4.977}.

\bibitem[\citeproctext]{ref-BrysbaertStevens2018}
Brysbaert, Marc, and Michaël Stevens. 2018. {``Power Analysis and Effect
Size in Mixed Effects Models: A Tutorial.''} \emph{Journal of Cognition}
1 (1). \url{https://doi.org/10.5334/joc.10}.

\bibitem[\citeproctext]{ref-Burgess1998}
Burgess, Curt, and Kay Livesay. 1998. {``The Effect of Corpus Size in
Predicting Reaction Time in a Basic Word Recognition Task: Moving on
from Ku{č}era and Francis.''} \emph{Behavior Research Methods,
Instruments, \& Computers} 30 (2): 272--77.

\bibitem[\citeproctext]{ref-Button2013}
Button, Katherine S, John PA Ioannidis, Claire Mokrysz, Brian A Nosek,
Jonathan Flint, Emma SJ Robinson, and Marcus R Munafò. 2013. {``Power
Failure: Why Small Sample Size Undermines the Reliability of
Neuroscience.''} \emph{Nature Reviews Neuroscience} 14 (5): 365--76.

\bibitem[\citeproctext]{ref-Cayado2023}
Cayado, Dave Kenneth Tayao, Samantha Wray, and Linnaea Stockall. 2023.
{``Does Linear Position Matter for Morphological Processing? Evidence
from a Tagalog Masked Priming Experiment.''} \emph{Language, Cognition
and Neuroscience}, 1--16.

\bibitem[\citeproctext]{ref-Cohen1992}
Cohen, Jacob. 1992. {``A Power Primer.''} \emph{Psychological Bulletin}
112 (1): 155--59. \url{https://doi.org/10.1037/0033-2909.112.1.155}.

\bibitem[\citeproctext]{ref-ColtheartEtal2001}
Coltheart, Max, Kathleen Rastle, Conrad Perry, Robyn Langdon, and
Johannes Ziegler. 2001. {``DRC: A Dual Route Cascaded Model of Visual
Word Recognition and Reading Aloud.''} \emph{Psychological Review} 108
(1): 204--56.

\bibitem[\citeproctext]{ref-deLeeuw2014}
de Leeuw, Joshua R. 2014. {``jsPsych: A JavaScript Library for Creating
Behavioral Experiments in a Web Browser.''} \emph{Behavior Research
Methods} 47 (1): 1--12. \url{https://doi.org/10.3758/s13428-014-0458-y}.

\bibitem[\citeproctext]{ref-EvettHumphreys1981}
Evett, Lindsay J., and Glyn W. Humphreys. 1981. {``The Use of Abstract
Graphemic Information in Lexical Access.''} \emph{The Quarterly Journal
of Experimental Psychology} 33 (4): 325--50.

\bibitem[\citeproctext]{ref-Labvanced2017}
Finger, Holger, Caspar Goeke, Dorena Diekamp, Kai Standvoß, and Peter
König. 2017. {``LabVanced: A Unified JavaScript Framework for Online
Studies.''} In \emph{2017 International Conference on Computational
Social Science}. Cologne, Germany.

\bibitem[\citeproctext]{ref-Forster1998}
Forster, Kenneth I. 1998. {``The Pros and Cons of Masked Priming.''}
\emph{Journal of Psycholinguistic Research} 27 (2): 203--33.

\bibitem[\citeproctext]{ref-Forster1999}
---------. 1999. {``Microgenesis of Priming Effects in Lexical
Access.''} \emph{Brain and Language} 68: 5--15.

\bibitem[\citeproctext]{ref-ForsterAzuma2000}
Forster, Kenneth I., and Tamiko Azuma. 2000. {``Masked Priming for
Prefixed Words with Bound Stems: Does Submit Prime Permit?''}
\emph{Language and Cognitive Processes} 15 (4-5): 539--61.

\bibitem[\citeproctext]{ref-ForsterDavis1984}
Forster, Kenneth I., and Chris Davis. 1984. {``Repetition Priming and
Frequency Attenuation in Lexical Access.''} \emph{Journal of
Experimental Psychology: Learning, Memory, and Cognition} 10 (4): 680.

\bibitem[\citeproctext]{ref-ForsterDavis1991}
Forster, Kenneth I., and Christopher Davis. 1991. {``The Density
Constraint on Form-Priming in the Naming Task: Interference Effects from
a Masked Prime.''} \emph{Journal of Memory and Language} 30 (1): 1--25.
\url{https://doi.org/10.1016/0749-596x(91)90008-8}.

\bibitem[\citeproctext]{ref-ForsterEtal1987}
Forster, Kenneth I., C. Davis, C. Schoknecht, and R. Carter. 1987.
{``Masked Priming with Graphemically Related Forms: Repetition or
Partial Activation?''} \emph{The Quarterly Journal of Experimental
Psychology Section A} 39 (2): 211--51.
\url{https://doi.org/10.1080/14640748708401785}.

\bibitem[\citeproctext]{ref-ForsterEtal2003}
Forster, Kenneth I., Kathleen Mohan, and Jo Hector. 2003. {``The
Mechanics of Masked Priming.''} In \emph{Masked Priming: The State of
the Art}, edited by Sachiko Kinoshita and Stephen J. Lupker, 3--37. New
York, NY/Hove, UK: Psychology Press.

\bibitem[\citeproctext]{ref-GelmanCarlin2014}
Gelman, Andrew, and John Carlin. 2014. {``Beyond Power Calculations.''}
\emph{Perspectives on Psychological Science} 9 (6): 641--51.
\url{https://doi.org/10.1177/1745691614551642}.

\bibitem[\citeproctext]{ref-Gimenes2016}
Gimenes, Manuel, and Boris New. 2016. {``Worldlex: Twitter and Blog Word
Frequencies for 66 Languages.''} \emph{Behavior Research Methods} 48:
963--72.

\bibitem[\citeproctext]{ref-Gomez2013}
Gomez, Pablo, Manuel Perea, and Roger Ratcliff. 2013. {``A Diffusion
Model Account of Masked Versus Unmasked Priming: Are They Qualitatively
Different?''} \emph{Journal of Experimental Psychology: Human Perception
and Performance} 39 (6): 1731.

\bibitem[\citeproctext]{ref-GraingerJacobs1996}
Grainger, Jonathan, and Arthur M. Jacobs. 1996. {``Orthographic
Processing in Visual Word Recognition: A Multiple Read-Out Model.''}
\emph{Psychological Review} 103 (3): 518.

\bibitem[\citeproctext]{ref-Herdaugdelen2017}
Herdağdelen, Amaç, and Marco Marelli. 2017. {``Social Media and Language
Processing: How Facebook and Twitter Provide the Best Frequency
Estimates for Studying Word Recognition.''} \emph{Cognitive Science} 41
(4): 976--95.

\bibitem[\citeproctext]{ref-Jacoby1983}
Jacoby, Larry L. 1983. {``Remembering the Data: Analyzing Interactive
Processes in Reading.''} \emph{Journal of Verbal Learning and Verbal
Behavior} 22 (5): 485--508.

\bibitem[\citeproctext]{ref-Jacoby1981}
Jacoby, Larry L, and Mark Dallas. 1981. {``On the Relationship Between
Autobiographical Memory and Perceptual Learning.''} \emph{Journal of
Experimental Psychology: General} 110 (3): 306.

\bibitem[\citeproctext]{ref-Kinoshita2006}
Kinoshita, Sachiko. 2006. {``Additive and Interactive Effects of Word
Frequency and Masked Repetition in the Lexical Decision Task.''}
\emph{Psychonomic Bulletin \& Review} 13 (4): 668--73.
\url{https://doi.org/10.3758/bf03193979}.

\bibitem[\citeproctext]{ref-KuceraFrancis1967}
Kučera, J., and W. N. Francis. 1967. \emph{Computational Analysis of
Present Day American English}. Providence, RI: Brown University Press.

\bibitem[\citeproctext]{ref-LukacsGaspar2023}
Lukács, Gáspár, and Andreas Gartus. 2023. {``Precise Display Time
Measurement in JavaScript for Web-Based Experiments.''} \emph{Behavior
Research Methods} 55 (3): 1079--93.
\url{https://doi.org/10.3758/s13428-022-01835-2}.

\bibitem[\citeproctext]{ref-LundBurgess1996}
Lund, Kevin, and Curt Burgess. 1996. {``Producing High-Dimensional
Semantic Spaces from Lexical Co-Occurrence.''} \emph{Behavior Research
Methods, Instruments, {\&} Computers} 28 (2): 203--8.

\bibitem[\citeproctext]{ref-MassonBodner2003}
Masson, Michael E. J., and Glen E. Bodner. 2003. {``A Retrospective View
of Masked Priming: Toward a Unified Account of Masked and Long-Term
Repetition Priming.''} \emph{Masked Priming: The State of the Art},
57--94.

\bibitem[\citeproctext]{ref-McClellandRumelhart1981}
McClelland, James L., and David E. Rumelhart. 1981. {``An Interactive
Activation Model of Context Effects in Letter Perception: Part i. An
Account of Basic Findings.''} \emph{Psychological Review} 88 (5):
375--407.

\bibitem[\citeproctext]{ref-Nievas2010}
Nievas, Francisco. 2010. {``The Frequency Attenuation Effect in Identity
and Associative Priming.''} \emph{The Spanish Journal of Psychology} 13
(1): 30--62. \url{https://doi.org/10.1017/s1138741600003668}.

\bibitem[\citeproctext]{ref-NorrisKinoshita2008}
Norris, Dennis, and Sachiko Kinoshita. 2008. {``Perception as Evidence
Accumulation and Bayesian Inference: Insights from Masked Priming.''}
\emph{Journal of Experimental Psychology: General} 137 (3): 434--55.
\url{https://doi.org/10.1037/a0012799}.

\bibitem[\citeproctext]{ref-PeirceEtal2019}
Peirce, Jonathan, Jeremy R. Gray, Sol Simpson, Michael MacAskill,
Richard Höchenberger, Hiroyuki Sogo, Erik Kastman, and Jonas Kristoffer
Lindeløv. 2019. {``PsychoPy2: Experiments in Behavior Made Easy.''}
\emph{Behavior Research Methods} 51 (1): 195--203.

\bibitem[\citeproctext]{ref-Petrosino2020}
Petrosino, Roberto. 2020. {``More Than Islands of Regularity: An
Investigation of the Sensitivity of Morphological Decomposition to
Higher-Level Linguistic Properties.''} PhD thesis, University of
Connecticut.

\bibitem[\citeproctext]{ref-PetrosinoEtal2023}
Petrosino, Roberto, Jon Sprouse, and Diogo Almeida. 2023. {``Asymmetries
in the Stem and Suffix Masked Priming Response in a Large-Scale Online
Study.''} \emph{Quaderni Di Linguistica e Studi Orientali}, no. 49:
177--94. \url{https://doi.org/10.13128/QUL-SO-2421-7220-15154}.

\bibitem[\citeproctext]{ref-PotvinSchtuz2000}
Potvin, Patrick J., and Robert W. Schutz. 2000. {``Statistical Power for
the Two-Factor Repeated Measures ANOVA.''} \emph{Behavior Research
Methods, Instruments, \& Computers} 32 (2): 347--56.
\url{https://doi.org/10.3758/bf03207805}.

\bibitem[\citeproctext]{ref-RajaramNeely1992}
Rajaram, Suparna, and James H Neely. 1992. {``Dissociative Masked
Repetition Priming and Word Frequency Effects in Lexical Decision and
Episodic Recognition Tasks.''} \emph{Journal of Memory and Language} 31
(2): 152--82. \url{https://doi.org/10.1016/0749-596x(92)90009-m}.

\bibitem[\citeproctext]{ref-ScarboroughEtal1977}
Scarborough, Don L., Charles Cortese, and Hollis S. Scarborough. 1977.
{``Frequency and Repetition Effects in Lexical Memory.''} \emph{Journal
of Experimental Psychology: Human Perception and Performance} 3 (1):
1--17. \url{https://doi.org/10.1037/0096-1523.3.1.1}.

\bibitem[\citeproctext]{ref-SeguiGrainger1990}
Segui, Juan, and Jonathan Grainger. 1990. {``Priming Word Recognition
with Orthographic Neighbors: Effects of Relative Prime-Target
Frequency.''} \emph{Journal of Experimental Psychology: Human Perception
and Performance} 16 (1): 65--76.
\url{https://doi.org/10.1037/0096-1523.16.1.65}.

\bibitem[\citeproctext]{ref-Sereno1991}
Sereno, Joan A. 1991. {``Graphemic, Associative, and Syntactic Priming
Effects at a Brief Stimulus Onset Asynchrony in Lexical Decision and
Naming.''} \emph{Journal of Experimental Psychology: Learning, Memory,
and Cognition} 17 (3): 459--77.
\url{https://doi.org/10.1037/0278-7393.17.3.459}.

\bibitem[\citeproctext]{ref-StockallMarantz2006}
Stockall, Linnaea, and Alec Marantz. 2006. {``A Single Route, Full
Decomposition Model of Morphological Complexity: MEG Evidence.''}
\emph{The Mental Lexicon} 1 (1): 85--123.
https://doi.org/\url{https://doi.org/10.1075/ml.1.1.07sto}.

\bibitem[\citeproctext]{ref-TaftForster1975}
Taft, Marcus, and Kenneth I. Forster. 1975. {``Lexical Storage and
Retrieval of Prefixed Words.''} \emph{Journal of Verbal Learning and
Verbal Behavior} 14 (6): 638--47.

\bibitem[\citeproctext]{ref-w3counterGlobalStats}
W3 Counter. 2023. {``Browser \& Platform Market Share - November
2023.''}
\url{https://www.w3counter.com/globalstats.php?year=2023&month=11}.

\bibitem[\citeproctext]{ref-Yap2009}
Yap, Melvin J., and David A. Balota. 2009. {``Visual Word Recognition of
Multisyllabic Words.''} \emph{Journal of Memory and Language} 60 (4):
502--29.

\bibitem[\citeproctext]{ref-Zevin2002}
Zevin, Jason D, and Mark S Seidenberg. 2002. {``Age of Acquisition
Effects in Word Reading and Other Tasks.''} \emph{Journal of Memory and
Language} 47 (1): 1--29.

\bibitem[\citeproctext]{ref-Zimmerman2012}
Zimmerman, Robert, and Pablo Gomez. 2012. {``Drawing Attention to Primes
Increases Inhibitory Word Priming Effects.''} \emph{The Mental Lexicon}
7 (2): 119--46.

\end{CSLReferences}

\newpage

\section*{Wordlists}\label{wordlists}
\addcontentsline{toc}{section}{Wordlists}

\subsubsection*{Experiment 1}\label{experiment-1}
\addcontentsline{toc}{subsubsection}{Experiment 1}

\begin{longtable*}{llrrrr}
\toprule
 &  & \multicolumn{2}{c}{RT (to repetition)} & \multicolumn{2}{c}{RT (to unrelated)} \\ 
\cmidrule(lr){3-4} \cmidrule(lr){5-6}
unrelated prime & word & mean & SD & mean & SD \\ 
\midrule\addlinespace[2.5pt]
\multicolumn{6}{l}{\emph{low frequency condition}} \\ 
\midrule\addlinespace[2.5pt]
smash & chasm & 714 & 216 & 831 & 250 \\ 
manna & oxide & 719 & 198 & 715 & 156 \\ 
legit & vowel & 655 & 139 & 694 & 152 \\ 
blunt & clerk & 617 & 157 & 635 & 133 \\ 
slope & bleed & 609 & 171 & 621 & 176 \\ 
nasal & decor & 654 & 140 & 694 & 204 \\ 
forte & quirk & 689 & 204 & 688 & 155 \\ 
aloud & speck & 732 & 208 & 739 & 187 \\ 
nymph & stash & 638 & 175 & 657 & 142 \\ 
crass & ditch & 671 & 173 & 678 & 157 \\ 
squid & snare & 684 & 168 & 722 & 164 \\ 
swirl & budge & 672 & 200 & 732 & 207 \\ 
grunt & slack & 608 & 129 & 664 & 157 \\ 
taunt & sedan & 711 & 197 & 705 & 122 \\ 
cigar & tally & 667 & 131 & 720 & 176 \\ 
lunge & posit & – & – & – & – \\ 
negro & flock & 654 & 141 & 716 & 166 \\ 
exert & scorn & 670 & 159 & 651 & 146 \\ 
lathe & grail & 697 & 206 & 718 & 171 \\ 
viola & bloat & 663 & 185 & 698 & 181 \\ 
rival & tumor & 627 & 159 & 651 & 152 \\ 
dizzy & acute & 662 & 174 & 660 & 142 \\ 
hertz & sauna & 652 & 132 & 706 & 154 \\ 
haste & elect & 640 & 162 & 650 & 144 \\ 
poppy & spoof & 706 & 185 & 759 & 201 \\ 
clove & plush & 615 & 138 & 669 & 175 \\ 
guise & fiend & 785 & 209 & 846 & 185 \\ 
magma & knelt & 744 & 213 & 814 & 225 \\ 
lotto & privy & 733 & 182 & 777 & 219 \\ 
kayak & sigma & 798 & 258 & 796 & 205 \\ 
taint & parse & – & – & – & – \\ 
fanny & carte & – & – & – & – \\ 
rouge & verge & 664 & 168 & 672 & 171 \\ 
vitro & mourn & 665 & 171 & 682 & 186 \\ 
floss & shrug & 687 & 175 & 682 & 132 \\ 
tempt & clasp & 658 & 128 & 701 & 178 \\ 
flirt & bathe & 659 & 159 & 701 & 197 \\ 
fluff & linen & 620 & 91 & 650 & 133 \\ 
butch & stare & 617 & 126 & 632 & 144 \\ 
bowel & medic & 637 & 166 & 663 & 218 \\ 
aspen & weave & 614 & 128 & 649 & 128 \\ 
chime & flint & 681 & 140 & 718 & 191 \\ 
crust & flank & 689 & 176 & 740 & 177 \\ 
spunk & scrub & 645 & 172 & 670 & 167 \\ 
stoke & hoist & 686 & 168 & 724 & 190 \\ 
dairy & stout & 667 & 148 & 707 & 166 \\ 
stale & cough & 588 & 147 & 629 & 157 \\ 
gypsy & annex & 744 & 197 & 798 & 169 \\ 
gloss & plume & 730 & 195 & 775 & 194 \\ 
topaz & quart & 662 & 159 & 715 & 205 \\ 
\midrule\addlinespace[2.5pt]
\multicolumn{6}{l}{\emph{high frequency condition}} \\ 
\midrule\addlinespace[2.5pt]
shoot & proof & 576 & 119 & 617 & 129 \\ 
usual & clear & 598 & 169 & 588 & 120 \\ 
teach & audio & 589 & 141 & 632 & 112 \\ 
adult & apply & 592 & 154 & 632 & 130 \\ 
allow & phone & 573 & 143 & 588 & 89 \\ 
forum & class & 656 & 162 & 682 & 197 \\ 
whole & raise & 611 & 154 & 598 & 116 \\ 
often & civil & 580 & 107 & 623 & 120 \\ 
issue & match & 590 & 119 & 619 & 169 \\ 
style & local & 589 & 141 & 580 & 113 \\ 
coast & minor & 600 & 137 & 632 & 157 \\ 
reach & below & 611 & 143 & 618 & 90 \\ 
smith & extra & 599 & 146 & 609 & 141 \\ 
speed & court & 585 & 115 & 638 & 141 \\ 
sense & exact & 592 & 127 & 590 & 113 \\ 
write & bunch & 647 & 140 & 646 & 130 \\ 
trust & quick & 554 & 104 & 616 & 134 \\ 
sleep & birth & 619 & 165 & 609 & 156 \\ 
reply & truth & 579 & 140 & 611 & 150 \\ 
track & serve & 611 & 136 & 649 & 168 \\ 
dream & trade & 606 & 185 & 602 & 106 \\ 
image & heart & 592 & 159 & 602 & 113 \\ 
white & index & 606 & 111 & 625 & 146 \\ 
flame & cable & 583 & 119 & 626 & 130 \\ 
value & break & 605 & 163 & 601 & 133 \\ 
avoid & woman & 576 & 119 & 609 & 153 \\ 
short & front & 587 & 138 & 619 & 140 \\ 
aware & voice & 562 & 127 & 585 & 116 \\ 
large & stock & 596 & 148 & 661 & 216 \\ 
prove & seven & 583 & 130 & 653 & 193 \\ 
brand & blood & 568 & 109 & 598 & 109 \\ 
river & plain & 596 & 115 & 617 & 123 \\ 
guess & solid & 643 & 158 & 612 & 140 \\ 
month & limit & 603 & 122 & 658 & 136 \\ 
heard & scale & 632 & 144 & 639 & 176 \\ 
space & stuff & 623 & 133 & 642 & 154 \\ 
leave & major & 599 & 139 & 585 & 123 \\ 
agree & brown & 591 & 120 & 632 & 167 \\ 
metal & house & 552 & 121 & 603 & 137 \\ 
along & stage & 590 & 138 & 619 & 160 \\ 
print & built & 628 & 155 & 664 & 166 \\ 
worst & video & 570 & 113 & 650 & 157 \\ 
sound & story & 594 & 129 & 614 & 176 \\ 
faith & march & 607 & 134 & 630 & 191 \\ 
quote & clean & 553 & 93 & 585 & 135 \\ 
train & price & 599 & 141 & 624 & 189 \\ 
small & event & 583 & 127 & 623 & 166 \\ 
night & thank & 656 & 190 & 607 & 128 \\ 
shell & radio & 577 & 131 & 604 & 162 \\ 
alone & sorry & 592 & 155 & 609 & 140 \\ 
\midrule\addlinespace[2.5pt]
\multicolumn{6}{l}{non-word} \\ 
\midrule\addlinespace[2.5pt]
strat & inurt & 726 & 259 & 712 & 215 \\ 
gleat & shawt & 760 & 270 & 672 & 154 \\ 
dolio & delax & 758 & 182 & 767 & 195 \\ 
cutch & thelp & 745 & 242 & 687 & 199 \\ 
greaf & isapt & 645 & 181 & 628 & 160 \\ 
broot & fopaz & 660 & 196 & 628 & 125 \\ 
lubic & fuxom & 676 & 234 & 601 & 126 \\ 
drirk & bloot & 761 & 190 & 744 & 172 \\ 
cooch & scart & 768 & 220 & 726 & 162 \\ 
motem & frint & 720 & 203 & 685 & 148 \\ 
abapt & ahuck & 673 & 207 & 633 & 153 \\ 
nigit & netro & 734 & 217 & 721 & 169 \\ 
hilac & moust & 744 & 186 & 798 & 174 \\ 
cojex & barsh & 731 & 216 & 706 & 183 \\ 
prilt & avort & 710 & 196 & 725 & 199 \\ 
whirp & venem & – & – & – & – \\ 
shino & grack & 743 & 209 & 728 & 182 \\ 
nelch & ranth & 681 & 174 & 654 & 135 \\ 
exulk & frick & – & – & – & – \\ 
morex & nohew & 683 & 197 & 656 & 165 \\ 
tamek & pramp & 745 & 239 & 696 & 200 \\ 
miant & altep & 664 & 179 & 654 & 159 \\ 
bloth & scrib & 788 & 243 & 749 & 230 \\ 
bumbo & tumph & 785 & 204 & 768 & 210 \\ 
occut & dorst & 686 & 168 & 674 & 184 \\ 
topec & thint & 754 & 205 & 748 & 153 \\ 
shoof & rourt & 691 & 192 & 688 & 194 \\ 
spack & smout & 759 & 195 & 736 & 184 \\ 
blenk & kayuk & 823 & 289 & 772 & 237 \\ 
silaf & drick & 727 & 189 & 678 & 131 \\ 
crunk & smoop & 710 & 185 & 684 & 154 \\ 
fluck & deirm & 649 & 161 & 657 & 178 \\ 
ghisk & ephic & 787 & 223 & 751 & 212 \\ 
chrik & glurp & 731 & 209 & 727 & 236 \\ 
cetup & blumb & 746 & 183 & 733 & 220 \\ 
firch & eicht & 725 & 226 & 718 & 205 \\ 
vasem & forim & 736 & 214 & 690 & 185 \\ 
earch & slent & 840 & 207 & 773 & 178 \\ 
blont & lepot & 693 & 203 & 659 & 162 \\ 
ecret & plock & 763 & 222 & 734 & 195 \\ 
wateb & ocheb & 643 & 168 & 620 & 130 \\ 
trook & febut & 659 & 166 & 632 & 156 \\ 
ruzak & coreb & 656 & 169 & 643 & 133 \\ 
theet & frath & 738 & 193 & 699 & 148 \\ 
blamp & eggem & 705 & 190 & 681 & 160 \\ 
lambo & gredo & 700 & 217 & 689 & 182 \\ 
aliom & brost & 728 & 204 & 690 & 170 \\ 
brust & ganic & 712 & 178 & 660 & 117 \\ 
cleot & polep & 714 & 236 & 641 & 174 \\ 
lindo & snock & 766 & 194 & 776 & 206 \\ 
driff & fomit & 711 & 187 & 633 & 147 \\ 
wrast & sholf & 665 & 157 & 642 & 111 \\ 
lidst & racef & 668 & 167 & 658 & 171 \\ 
huirk & thamp & 711 & 188 & 708 & 226 \\ 
pumbo & purso & 702 & 196 & 665 & 168 \\ 
whilo & glarm & 765 & 210 & 748 & 184 \\ 
murkt & fingo & 707 & 164 & 683 & 179 \\ 
steck & gotch & – & – & – & – \\ 
molax & spuff & 745 & 198 & 692 & 151 \\ 
ronch & schew & 811 & 294 & 756 & 265 \\ 
guesh & humot & 690 & 175 & 674 & 149 \\ 
snump & sgrew & 706 & 212 & 724 & 175 \\ 
fleak & fadio & 713 & 175 & 678 & 141 \\ 
recup & plint & 768 & 246 & 735 & 225 \\ 
loast & pheek & 696 & 181 & 676 & 192 \\ 
smalt & blasm & 785 & 226 & 755 & 175 \\ 
swimp & reash & 780 & 187 & 754 & 181 \\ 
tymph & chank & 798 & 229 & 774 & 221 \\ 
laget & septh & 721 & 196 & 688 & 193 \\ 
gluck & feeth & 756 & 191 & 720 & 156 \\ 
gatob & tosit & 683 & 184 & 668 & 210 \\ 
sauto & exuct & 767 & 232 & 693 & 191 \\ 
crunt & ethym & 724 & 211 & 700 & 213 \\ 
pranc & feght & 718 & 187 & 723 & 203 \\ 
twank & stoff & 709 & 165 & 688 & 155 \\ 
letap & cruck & 742 & 197 & 812 & 229 \\ 
alash & fatho & 643 & 146 & 660 & 184 \\ 
sharf & firsh & 717 & 168 & 717 & 196 \\ 
frimp & paltz & 688 & 211 & 719 & 227 \\ 
lumpo & thark & 683 & 134 & 714 & 205 \\ 
huilt & aufit & 638 & 146 & 649 & 184 \\ 
brosk & hinup & 636 & 126 & 653 & 142 \\ 
dulch & jongo & 681 & 181 & 705 & 202 \\ 
dealf & guast & 670 & 178 & 687 & 210 \\ 
drash & sunch & 697 & 196 & 692 & 190 \\ 
prock & cleak & 766 & 177 & 819 & 214 \\ 
spaft & stram & 720 & 157 & 726 & 155 \\ 
criex & etuip & 620 & 138 & 635 & 177 \\ 
phumb & opert & 750 & 225 & 791 & 255 \\ 
denet & keach & 670 & 176 & 700 & 189 \\ 
bluck & umarm & 719 & 213 & 756 & 239 \\ 
racet & tooch & 739 & 213 & 741 & 234 \\ 
phrap & chuth & 682 & 152 & 726 & 208 \\ 
wight & tedic & 695 & 196 & 704 & 199 \\ 
lorro & mutch & 796 & 257 & 811 & 279 \\ 
oorph & hilth & 682 & 195 & 711 & 213 \\ 
praph & pluff & – & – & – & – \\ 
aboot & widet & 799 & 222 & 818 & 251 \\ 
hoest & scook & 721 & 168 & 749 & 201 \\ 
polic & fisco & 797 & 261 & 797 & 271 \\ 
glunk & gamit & 751 & 257 & 725 & 243 \\ 
letch & phasm & – & – & – & – \\ 
spink & sondo & 679 & 168 & 672 & 182 \\ 
dippo & vuint & 634 & 137 & 616 & 130 \\ 
astef & rynic & 629 & 123 & 658 & 161 \\ 
tatch & waget & 736 & 205 & 747 & 211 \\ 
shoop & vooch & 671 & 158 & 691 & 169 \\ 
isloo & guilm & 675 & 179 & 719 & 210 \\ 
scack & elsom & 686 & 195 & 704 & 248 \\ 
bliff & crost & 718 & 190 & 731 & 199 \\ 
cempo & alept & 754 & 186 & 780 & 216 \\ 
glaim & robit & 741 & 206 & 783 & 220 \\ 
thunt & noast & 658 & 122 & 688 & 160 \\ 
plesh & bealm & 740 & 175 & 759 & 204 \\ 
thoop & hyrup & 703 & 125 & 741 & 191 \\ 
louth & chost & 752 & 192 & 778 & 209 \\ 
preak & borif & 617 & 111 & 616 & 130 \\ 
creck & starp & 751 & 215 & 744 & 208 \\ 
realp & valif & 656 & 178 & 678 & 190 \\ 
ferit & raceb & 674 & 203 & 687 & 174 \\ 
theep & dacit & 642 & 171 & 649 & 179 \\ 
murch & abert & 733 & 190 & 765 & 233 \\ 
blomp & paith & 703 & 161 & 724 & 170 \\ 
sloup & mough & 710 & 145 & 719 & 145 \\ 
strit & plick & 768 & 218 & 793 & 232 \\ 
skinp & toost & 763 & 198 & 786 & 218 \\ 
phock & tacao & 751 & 217 & 778 & 285 \\ 
cyrrh & kneak & 790 & 212 & 826 & 228 \\ 
ahack & vitch & 682 & 155 & 717 & 238 \\ 
saist & paxim & 647 & 152 & 667 & 185 \\ 
pheep & kingo & 734 & 167 & 738 & 175 \\ 
ehert & truff & 767 & 218 & 771 & 246 \\ 
spuck & fundt & 655 & 162 & 700 & 183 \\ 
antuc & bloam & 719 & 155 & 741 & 191 \\ 
shish & quilp & 726 & 217 & 718 & 233 \\ 
gijou & fotch & 658 & 127 & 661 & 132 \\ 
drarp & broup & 674 & 161 & 690 & 213 \\ 
stilp & krauf & 683 & 183 & 683 & 200 \\ 
doint & swaft & 826 & 286 & 821 & 253 \\ 
owlut & adoof & 726 & 176 & 724 & 191 \\ 
swant & meash & 722 & 195 & 776 & 230 \\ 
vepot & afent & 660 & 151 & 651 & 180 \\ 
ploic & setip & 705 & 198 & 710 & 203 \\ 
glick & linew & 769 & 207 & 794 & 242 \\ 
hatex & corax & 696 & 162 & 755 & 218 \\ 
framo & scock & 811 & 233 & 807 & 232 \\ 
praft & quast & 733 & 193 & 763 & 211 \\ 
minch & ipept & 685 & 201 & 691 & 209 \\ 
ragic & gonet & 658 & 172 & 692 & 203 \\ 
stabt & lertz & 629 & 153 & 652 & 155 \\ 
\bottomrule
\end{longtable*}

\subsubsection*{Experiment 2}\label{experiment-2-1}
\addcontentsline{toc}{subsubsection}{Experiment 2}

\begin{longtable*}{lllrrrr}
\toprule
 &  &  & \multicolumn{2}{c}{RT (to repetition)} & \multicolumn{2}{c}{RT (to unrelated)} \\ 
\cmidrule(lr){4-5} \cmidrule(lr){6-7}
related & unrelated prime & word & mean & SD & mean & SD \\ 
\midrule\addlinespace[2.5pt]
\multicolumn{7}{l}{\emph{low frequency condition}} \\ 
\midrule\addlinespace[2.5pt]
arrow & hunch & arrow & 590 & 130 & 587 & 124 \\ 
pitch & sneak & pitch & 576 & 126 & 612 & 122 \\ 
hatch & widow & hatch & 621 & 151 & 639 & 148 \\ 
shark & brief & shark & 573 & 125 & 590 & 138 \\ 
tooth & sharp & tooth & 536 & 125 & 565 & 116 \\ 
booth & grief & booth & 572 & 136 & 627 & 157 \\ 
pound & sting & pound & 551 & 127 & 572 & 127 \\ 
weigh & thief & weigh & 593 & 167 & 636 & 164 \\ 
blank & avoid & blank & 571 & 139 & 596 & 124 \\ 
crush & award & crush & 554 & 128 & 592 & 136 \\ 
bench & smack & bench & 573 & 132 & 601 & 129 \\ 
fetch & brand & fetch & 622 & 156 & 658 & 146 \\ 
cheek & salad & cheek & 561 & 141 & 602 & 142 \\ 
brush & swamp & brush & 564 & 130 & 600 & 128 \\ 
march & depth & march & 559 & 125 & 580 & 123 \\ 
bleed & flesh & bleed & 560 & 148 & 577 & 146 \\ 
cliff & harsh & cliff & 602 & 130 & 645 & 137 \\ 
fraud & creep & fraud & 621 & 147 & 628 & 132 \\ 
cloud & plead & cloud & 536 & 115 & 551 & 101 \\ 
fluid & thumb & fluid & 605 & 140 & 678 & 162 \\ 
trash & creek & trash & 554 & 127 & 560 & 128 \\ 
flush & blond & flush & 576 & 123 & 617 & 140 \\ 
porch & stink & porch & 587 & 136 & 620 & 160 \\ 
stiff & patch & stiff & 626 & 154 & 678 & 156 \\ 
cough & sweep & cough & 564 & 142 & 601 & 141 \\ 
smash & squad & smash & 570 & 129 & 587 & 126 \\ 
\midrule\addlinespace[2.5pt]
\multicolumn{7}{l}{\emph{high frequency condition}} \\ 
\midrule\addlinespace[2.5pt]
blood & chief & blood & 541 & 130 & 551 & 104 \\ 
bunch & child & bunch & 585 & 148 & 617 & 145 \\ 
catch & board & catch & 545 & 116 & 562 & 130 \\ 
stuff & tough & stuff & 555 & 119 & 585 & 137 \\ 
break & stand & break & 545 & 107 & 561 & 124 \\ 
speak & beach & speak & 545 & 131 & 573 & 129 \\ 
stick & hotel & stick & 562 & 128 & 598 & 138 \\ 
sleep & angel & sleep & 538 & 113 & 559 & 119 \\ 
wrong & truth & wrong & 563 & 143 & 565 & 132 \\ 
grand & quick & grand & 571 & 127 & 582 & 143 \\ 
mouth & world & mouth & 543 & 125 & 556 & 119 \\ 
knock & extra & knock & 560 & 134 & 631 & 136 \\ 
guard & think & guard & 580 & 132 & 590 & 134 \\ 
small & thing & small & 557 & 130 & 577 & 125 \\ 
check & round & check & 558 & 135 & 562 & 121 \\ 
watch & proud & watch & 541 & 128 & 546 & 110 \\ 
group & smell & group & 559 & 127 & 576 & 142 \\ 
month & earth & month & 555 & 120 & 572 & 123 \\ 
south & relax & south & 575 & 139 & 611 & 133 \\ 
lunch & truck & lunch & 547 & 119 & 557 & 125 \\ 
clock & throw & clock & 548 & 132 & 574 & 124 \\ 
sound & death & sound & 538 & 127 & 552 & 103 \\ 
drink & north & drink & 559 & 129 & 556 & 122 \\ 
touch & young & touch & 541 & 122 & 573 & 121 \\ 
laugh & weird & laugh & 546 & 119 & 568 & 121 \\ 
black & reach & black & 553 & 131 & 563 & 114 \\ 
\midrule\addlinespace[2.5pt]
\multicolumn{7}{l}{non-word} \\ 
\midrule\addlinespace[2.5pt]
alkew & grack & alkew & 599 & 153 & 591 & 140 \\ 
agink & furob & agink & 626 & 148 & 614 & 141 \\ 
ruzak & begro & ruzak & 577 & 130 & 584 & 142 \\ 
sondo & labok & sondo & 625 & 142 & 612 & 149 \\ 
guesh & gazzo & guesh & 702 & 184 & 721 & 194 \\ 
fadio & criam & fadio & 618 & 149 & 604 & 146 \\ 
plich & coreb & plich & 650 & 162 & 640 & 159 \\ 
sgrew & docab & sgrew & 626 & 182 & 638 & 182 \\ 
sceak & colob & sceak & 675 & 154 & 683 & 171 \\ 
ghisk & isloo & ghisk & 588 & 139 & 593 & 139 \\ 
deirm & ahuck & deirm & 589 & 142 & 596 & 139 \\ 
villo & flurb & villo & 632 & 182 & 615 & 181 \\ 
tidow & pikto & tidow & 648 & 167 & 624 & 160 \\ 
drick & aliom & drick & 684 & 168 & 681 & 172 \\ 
phick & purso & phick & 643 & 160 & 637 & 165 \\ 
nello & borno & nello & 625 & 156 & 612 & 151 \\ 
feach & pacaw & feach & 730 & 201 & 720 & 192 \\ 
tello & rilth & tello & 651 & 175 & 644 & 171 \\ 
dolio & caveb & dolio & 602 & 148 & 610 & 165 \\ 
gorgo & swysh & gorgo & 643 & 164 & 619 & 170 \\ 
whilo & lanjo & whilo & 612 & 137 & 604 & 150 \\ 
stanf & drief & stanf & 611 & 134 & 617 & 133 \\ 
crulk & ocheb & crulk & 671 & 162 & 665 & 169 \\ 
phumb & tunch & phumb & 645 & 160 & 633 & 148 \\ 
sirth & steaf & sirth & 612 & 141 & 618 & 145 \\ 
slerk & nohew & slerk & 640 & 153 & 634 & 163 \\ 
vitbo & nualm & vitbo & 593 & 151 & 596 & 154 \\ 
sunch & ofium & sunch & 665 & 165 & 665 & 161 \\ 
soeth & croik & soeth & 589 & 141 & 589 & 130 \\ 
eltow & valuo & eltow & 628 & 171 & 606 & 158 \\ 
framo & sorgo & framo & 617 & 146 & 618 & 146 \\ 
lumpo & shavo & lumpo & 630 & 162 & 635 & 172 \\ 
spuff & oceab & spuff & 672 & 169 & 667 & 183 \\ 
gatob & tolio & gatob & 599 & 139 & 606 & 155 \\ 
nosom & theck & nosom & 598 & 155 & 604 & 139 \\ 
gezzo & tooch & gezzo & 592 & 136 & 586 & 131 \\ 
afoub & slonk & afoub & 582 & 133 & 589 & 128 \\ 
wateb & salch & wateb & 633 & 151 & 619 & 133 \\ 
nelch & raceb & nelch & 601 & 144 & 594 & 145 \\ 
dahoo & ahack & dahoo & 598 & 132 & 595 & 146 \\ 
driek & fideo & driek & 606 & 145 & 606 & 143 \\ 
gnask & fluko & gnask & 612 & 171 & 604 & 153 \\ 
brosk & cyrrh & brosk & 629 & 159 & 647 & 175 \\ 
duvez & revuo & duvez & 580 & 152 & 580 & 155 \\ 
fielm & cempo & fielm & 609 & 146 & 611 & 151 \\ 
pumph & exulk & pumph & 669 & 162 & 685 & 176 \\ 
gerif & kleck & gerif & 584 & 137 & 588 & 149 \\ 
racef & bonth & racef & 618 & 151 & 622 & 156 \\ 
pheek & scook & pheek & 640 & 155 & 644 & 176 \\ 
pruaw & slork & pruaw & 593 & 133 & 592 & 135 \\ 
guilm & whilf & guilm & 603 & 142 & 598 & 142 \\ 
lairf & drosh & lairf & 587 & 144 & 600 & 150 \\ 
\bottomrule
\end{longtable*}



\end{document}
