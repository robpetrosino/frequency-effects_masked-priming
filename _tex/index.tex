% Options for packages loaded elsewhere
\PassOptionsToPackage{unicode}{hyperref}
\PassOptionsToPackage{hyphens}{url}
\PassOptionsToPackage{dvipsnames,svgnames,x11names}{xcolor}
%
\documentclass[
]{interact}

\usepackage{amsmath,amssymb}
\usepackage{iftex}
\ifPDFTeX
  \usepackage[T1]{fontenc}
  \usepackage[utf8]{inputenc}
  \usepackage{textcomp} % provide euro and other symbols
\else % if luatex or xetex
  \usepackage{unicode-math}
  \defaultfontfeatures{Scale=MatchLowercase}
  \defaultfontfeatures[\rmfamily]{Ligatures=TeX,Scale=1}
\fi
\usepackage{lmodern}
\ifPDFTeX\else  
    % xetex/luatex font selection
\fi
% Use upquote if available, for straight quotes in verbatim environments
\IfFileExists{upquote.sty}{\usepackage{upquote}}{}
\IfFileExists{microtype.sty}{% use microtype if available
  \usepackage[]{microtype}
  \UseMicrotypeSet[protrusion]{basicmath} % disable protrusion for tt fonts
}{}
\makeatletter
\@ifundefined{KOMAClassName}{% if non-KOMA class
  \IfFileExists{parskip.sty}{%
    \usepackage{parskip}
  }{% else
    \setlength{\parindent}{0pt}
    \setlength{\parskip}{6pt plus 2pt minus 1pt}}
}{% if KOMA class
  \KOMAoptions{parskip=half}}
\makeatother
\usepackage{xcolor}
\setlength{\emergencystretch}{3em} % prevent overfull lines
\setcounter{secnumdepth}{5}
% Make \paragraph and \subparagraph free-standing
\makeatletter
\ifx\paragraph\undefined\else
  \let\oldparagraph\paragraph
  \renewcommand{\paragraph}{
    \@ifstar
      \xxxParagraphStar
      \xxxParagraphNoStar
  }
  \newcommand{\xxxParagraphStar}[1]{\oldparagraph*{#1}\mbox{}}
  \newcommand{\xxxParagraphNoStar}[1]{\oldparagraph{#1}\mbox{}}
\fi
\ifx\subparagraph\undefined\else
  \let\oldsubparagraph\subparagraph
  \renewcommand{\subparagraph}{
    \@ifstar
      \xxxSubParagraphStar
      \xxxSubParagraphNoStar
  }
  \newcommand{\xxxSubParagraphStar}[1]{\oldsubparagraph*{#1}\mbox{}}
  \newcommand{\xxxSubParagraphNoStar}[1]{\oldsubparagraph{#1}\mbox{}}
\fi
\makeatother


\providecommand{\tightlist}{%
  \setlength{\itemsep}{0pt}\setlength{\parskip}{0pt}}\usepackage{longtable,booktabs,array}
\usepackage{calc} % for calculating minipage widths
% Correct order of tables after \paragraph or \subparagraph
\usepackage{etoolbox}
\makeatletter
\patchcmd\longtable{\par}{\if@noskipsec\mbox{}\fi\par}{}{}
\makeatother
% Allow footnotes in longtable head/foot
\IfFileExists{footnotehyper.sty}{\usepackage{footnotehyper}}{\usepackage{footnote}}
\makesavenoteenv{longtable}
\usepackage{graphicx}
\makeatletter
\def\maxwidth{\ifdim\Gin@nat@width>\linewidth\linewidth\else\Gin@nat@width\fi}
\def\maxheight{\ifdim\Gin@nat@height>\textheight\textheight\else\Gin@nat@height\fi}
\makeatother
% Scale images if necessary, so that they will not overflow the page
% margins by default, and it is still possible to overwrite the defaults
% using explicit options in \includegraphics[width, height, ...]{}
\setkeys{Gin}{width=\maxwidth,height=\maxheight,keepaspectratio}
% Set default figure placement to htbp
\makeatletter
\def\fps@figure{htbp}
\makeatother
% definitions for citeproc citations
\NewDocumentCommand\citeproctext{}{}
\NewDocumentCommand\citeproc{mm}{%
  \begingroup\def\citeproctext{#2}\cite{#1}\endgroup}
\makeatletter
 % allow citations to break across lines
 \let\@cite@ofmt\@firstofone
 % avoid brackets around text for \cite:
 \def\@biblabel#1{}
 \def\@cite#1#2{{#1\if@tempswa , #2\fi}}
\makeatother
\newlength{\cslhangindent}
\setlength{\cslhangindent}{1.5em}
\newlength{\csllabelwidth}
\setlength{\csllabelwidth}{3em}
\newenvironment{CSLReferences}[2] % #1 hanging-indent, #2 entry-spacing
 {\begin{list}{}{%
  \setlength{\itemindent}{0pt}
  \setlength{\leftmargin}{0pt}
  \setlength{\parsep}{0pt}
  % turn on hanging indent if param 1 is 1
  \ifodd #1
   \setlength{\leftmargin}{\cslhangindent}
   \setlength{\itemindent}{-1\cslhangindent}
  \fi
  % set entry spacing
  \setlength{\itemsep}{#2\baselineskip}}}
 {\end{list}}
\usepackage{calc}
\newcommand{\CSLBlock}[1]{\hfill\break\parbox[t]{\linewidth}{\strut\ignorespaces#1\strut}}
\newcommand{\CSLLeftMargin}[1]{\parbox[t]{\csllabelwidth}{\strut#1\strut}}
\newcommand{\CSLRightInline}[1]{\parbox[t]{\linewidth - \csllabelwidth}{\strut#1\strut}}
\newcommand{\CSLIndent}[1]{\hspace{\cslhangindent}#1}

\usepackage{booktabs}
\usepackage{caption}
\usepackage{longtable}
\usepackage{colortbl}
\usepackage{array}
\usepackage{lscape}
\newcommand{\blandscape}{\begin{landscape}}
\newcommand{\elandscape}{\end{landscape}}
\usepackage{orcidlink}
\makeatletter
\@ifpackageloaded{caption}{}{\usepackage{caption}}
\AtBeginDocument{%
\ifdefined\contentsname
  \renewcommand*\contentsname{Table of contents}
\else
  \newcommand\contentsname{Table of contents}
\fi
\ifdefined\listfigurename
  \renewcommand*\listfigurename{List of Figures}
\else
  \newcommand\listfigurename{List of Figures}
\fi
\ifdefined\listtablename
  \renewcommand*\listtablename{List of Tables}
\else
  \newcommand\listtablename{List of Tables}
\fi
\ifdefined\figurename
  \renewcommand*\figurename{Figure}
\else
  \newcommand\figurename{Figure}
\fi
\ifdefined\tablename
  \renewcommand*\tablename{Table}
\else
  \newcommand\tablename{Table}
\fi
}
\@ifpackageloaded{float}{}{\usepackage{float}}
\floatstyle{ruled}
\@ifundefined{c@chapter}{\newfloat{codelisting}{h}{lop}}{\newfloat{codelisting}{h}{lop}[chapter]}
\floatname{codelisting}{Listing}
\newcommand*\listoflistings{\listof{codelisting}{List of Listings}}
\makeatother
\makeatletter
\makeatother
\makeatletter
\@ifpackageloaded{caption}{}{\usepackage{caption}}
\@ifpackageloaded{subcaption}{}{\usepackage{subcaption}}
\makeatother

\ifLuaTeX
  \usepackage{selnolig}  % disable illegal ligatures
\fi
\usepackage{bookmark}

\IfFileExists{xurl.sty}{\usepackage{xurl}}{} % add URL line breaks if available
\urlstyle{same} % disable monospaced font for URLs
\hypersetup{
  pdftitle={The detection and accurate estimation of frequency attenuation effects in masked repetition priming: A large scale web browser-based study},
  pdfauthor={Roberto Petrosino; Diogo Almeida},
  pdfkeywords={masked repetition priming, frequency attenuation
effect, online browser-based experiment, power analysis},
  colorlinks=true,
  linkcolor={blue},
  filecolor={Maroon},
  citecolor={Blue},
  urlcolor={Blue},
  pdfcreator={LaTeX via pandoc}}


\title{The detection and accurate estimation of frequency attenuation
effects in masked repetition priming: A large scale web browser-based
study}
\author{Roberto
Petrosino$\textsuperscript{1}$~\orcidlink{0000-0002-8502-3070}, Diogo
Almeida$\textsuperscript{1}$~\orcidlink{0000-0003-4674-8092}}

\thanks{CONTACT: Roberto
Petrosino. Email: \href{mailto:roberto.petrosino@nyu.edu}{\nolinkurl{roberto.petrosino@nyu.edu}}. Diogo
Almeida. Email: \href{mailto:diogo@nyu.edu}{\nolinkurl{diogo@nyu.edu}}. }
\begin{document}
\captionsetup{labelsep=space}
\maketitle
\textsuperscript{1} Psychology Program, Division of Science, New York
University Abu Dhabi, Abu Dhabi, United Arab Emirates
\begin{abstract}
This study investigates the controversy surrounding the sensitivity of
masked repetition priming to word frequency: while unmasked priming
exhibits a frequency attenuation effect, wherein high frequency words
yield smaller repetition effects, this phenomenon has been
inconsistently reported in masked priming. We conducted two large online
experiments with rigorously validated frequency databases to reconcile
past discrepancies. The first experiment confirmed the viability of
conducting masked priming experiments in web browser-based settings. The
pre-registered second study, designed for high statistical power and
precision, identified a 10-ms attenuation effect under masked priming.
This result suggests that the repetition effect in masked priming is
less qualitatively distinct from unmasked priming than previously
assumed. This finding has implications for masked priming experimental
design and theoretical consequences for models of priming. Crucially,
models that predict either the presence or absence of frequency
attenuation under masked conditions need to account for a small but
reliable effect.
\end{abstract}
\begin{keywords}
\def\sep{;\ }
masked repetition priming\sep frequency attenuation effect\sep online
browser-based experiment\sep 
power analysis
\end{keywords}


\section{Introduction}\label{sec-intro}

The masked priming technique has been an invaluable tool in visual word
recognition research. It has allowed researchers to study the conditions
under which orthographic, phonological, morphological, and semantic
information impact access to visual word forms while mitigating
strategic effects and minimizing the influence of controlled processes
(Forster 1998). First introduced in its traditional form by Forster and
Davis (1984; see also Evett and Humphreys 1981), this technique involves
a forward mask (i.e., usually a string of hashes, \#\#\#\#\#), followed
by a prime string presented for very short time (\(SOA < 60\) ms), and a
target string presented immediately after. Because the prime
presentation is so brief and masked by preceding and subsequent stimuli,
most participants report not being aware that a prime string has been
presented, and can at most report a screen flicker just before the
target presentation (Forster, Mohan, and Hector 2003).

Among possible manipulations of prime-target relatedness, masked
repetition priming (in which the same word is presented as both the
prime and target within the same trial: e.g., \emph{love-LOVE}) has been
well studied, because its response seems to be qualitatively different
from the unmasked counterpart (\(SOA > 60 ms\)): while high-frequency
words benefit less from repetition than low-frequency words in the
unmasked design (\emph{frequency attenuation effect}, henceforth FAE;
Scarborough, Cortese, and Scarborough 1977), this does not seem to be
the case when the prime is masked (Forster and Davis 1984; Forster et
al. 1987; Segui and Grainger 1990; Sereno 1991; Forster and Davis 1991;
Rajaram and Neely 1992; Bodner and Masson 1997; Forster, Mohan, and
Hector 2003; Nievas 2010).

This asymmetry in sensitivity to lexical frequency between the masked
and unmasked repetition priming responses has been important in
distinguishing among different models of priming in visual word
recognition. More specifically, \emph{interactive activation models}
(McClelland and Rumelhart 1981; Grainger and Jacobs 1996; Coltheart et
al. 2001) conceive of priming as a ``head start'' in processing due to
the pre-activation of the target word due to the presentation of the
prime. Thus, according to \emph{interactive activation} models, priming
is ultimately caused by a single mechanism, making the qualitatively
different profiles for repetition priming in masked and unmasked
conditions a difficult empirical finding to explain.

Similarly, episodic models (e.g., Jacoby and Dallas 1981; Jacoby 1983)
posit a different single mechanism for priming effects: the
activation/retrieval of the episodic memory trace of the encounter with
the prime word. These models therefore encounter the same type of
difficulty in accounting for qualitatively different patterns of
repetition priming effects in masked and unmasked conditions. A similar
type of model, called the \emph{memory recruitment model} makes very
similar predictions to the episodic memory models, positing a
non-lexical source for priming effects (Bodner and Masson 1997; Masson
and Bodner 2003; Bodner and Masson 2014). Repetition priming effects
under this view stem from the exploitation, strategically or
automatically, of a memory resource created by the encounter with the
prime word. The frequency attenuation effect, under episodic and memory
recruitment models alike, is predicted on the basis that low frequency
primes, being more distinctive stimuli, create a more potent and
effective memory resource compared to high frequency primes.

In contrast, other models appear to successfuly sidestep the problem
posed by the qualitatively different repetition priming profiles
observed in masked and unmasked conditions. One such model is the
\emph{entry-opening model} (also known as the \emph{bin model}; Forster
and Davis 1984). According to this model, when the visual stimulus is
presented, lexical entries are assigned to specific bins based on
orthographic similarity. In the first stage (fast search stage), a fast,
frequency-ordered search goes through the entries within a given bin,
and compares each one with the the input stimulus, assigning to each
entry a goodness-of-fit score. This comparison is fast and crude, and
sorts entries into (a) perfect (i.e., no difference is detected between
the input and the entry), (b) close (i.e., small differences are
detected), and (c) irrelevant matches (i.e., substantial difference are
detected). Any entry of type (a) or (b) is opened, so that the entry can
be further analyzed and compared to the input in the subsequent
verification stage. Under a masked presentation, the entry of the prime
word is opened at the fast search stage, but the short duration of the
stimulus prevents it from reaching the evaluation stage. Crucially, the
entry is nonetheless left open. Upon the presentation of the target
stimulus, the access procedure will follow its two stage course, with a
frequency-sensitive fast search and a subsequent entry opening for
evaluation/verification. In this view, the fast search for the target
word proceeds normally, but the evaluation/verification procedure starts
and ends sooner than it otherwise would, because the target entry has
already been left open after the brief processing of the prime. Thus,
the \emph{entry-opening model} explains the masked repetition priming as
the benefit from having the entry of the target word already open by the
time the second stage of recognition starts. Crucially, this occurs
\emph{after} the target word is initially accessed, which happens in
order of frequency. Put differently, according to the
\emph{entry-opening model}, masked repetition priming occurs because of
the time savings from not having to open the entry, which is a
frequency-insensitive process (i.e., every entry takes the same time to
be opened), but \emph{after} the frequency-sensitive first access stage.
As a consequence, the \emph{entry-opening model} predicts a
frequency-insensitive masked repetition priming effect, which is what
has been traditionally reported in the literature (see
Table~\ref{tbl-litReview}). In addition, it also (correctly) predicts
that pseudowords should not benefit from masked repetition priming, as
they have no entries in the mental lexicon to be left open after the
brief processing of the prime.

However, as Table~\ref{tbl-litReview} shows, there are nonetheless a few
studies that do report significant FAEs in masked repetition priming
Nievas (2010). Bodner and Masson (2001) argues that when stimuli are
presented in alternating case (e.g., \emph{pHoNe}), this increases the
lexical decision difficulty and therefore generates an extra incentive
to draw on the memory resource created by the brief processing of the
prime. Under such conditions, they were able to observe a statistically
significant FAE.

In the same vein, Kinoshita (2006) noticed that in earlier studies the
low frequency words often had very high error rates, and suggested that
perhaps many participants did not know them. If participants treated a
substantial number of low frequency words as nonwords, and nonwords do
not exhibit repetition priming under masked conditions, it could
artificially depress the repetition priming effect for the low frequency
condition alone, which could make any existing FAE harder to detect. In
two separate experiments, Kinoshita (2006) showed that larger repetition
priming effects for low frequency words were only obtained when the low
frequency words were vetted to make sure the participants knew them
prior to the experiment. Following up on Kinoshita (2006), Norris and
Kinoshita (2008) were also able to find an interaction between lexical
frequency and repetition in masked repetition priming, as was Nievas
(2010) in Spanish (exp. 1B).

Finally, as Table~\ref{tbl-litReview} shows, it is noteworthy that 15
out of 18 previous studies showed numerically larger masked priming
effects for low frequency words as opposed to high frequency words,
irrespective of statistical significance. Similarly, the average
repetition effect for low frequency words in the studies reviewed in
Table~\ref{tbl-litReview} is 13 ms larger when compared to that of high
frequency words. These results are not in line with the predictions
dictated by the \emph{entry opening model}, and seem to align better
with the predictions made by \emph{interactive activation models} and
\emph{memory recruitment} models.

\blandscape
\small

\setlength{\LTpost}{0mm}

\begin{longtable}{l|lrlrrrrll}

\caption{\label{tbl-litReview}Summary of the masked repetition priming
effects as a function of word frequency reported in the literature. The
statistical power range estimates were calculated by simulation with the
corresponding sample size (N) and for two representative FAE magnitudes.
Simulations were performed across a range of correlation values between
conditions (from 0.6 to 0.9, in increments of 0.1) as well as plausible
standard deviations per conditions (from 60 ms to 180 ms, in increments
of 10 ms), with 10,000 simulated datasets for each combination of
parameters.}

\tabularnewline

\toprule
\multicolumn{1}{l}{} &  &  &  & \multicolumn{2}{c}{MOP (ms)} & \multicolumn{2}{c}{FAE (ms)} & \multicolumn{2}{c}{Power range [min max]} \\ 
\cmidrule(lr){5-6} \cmidrule(lr){7-8} \cmidrule(lr){9-10}
\multicolumn{1}{l}{Study} & Language & N & SOA & HF & LF & ES & \emph{p}\textless{}.05? & FAE=15ms & FAE=30ms \\ 
\midrule\addlinespace[2.5pt]
Forster, Davis, Schoknecht, \& Carter (1987), exp. 1 & English & 16 & 60 & 61 & 66 & 5 &   & [0.02 0.24] & [0.04 0.84] \\ 
Norris, Kinoshita, Hall, \& Henson (2018) & English & 16 & 50 & 38 & 51 & 13 &   & [0.02 0.24] & [0.04 0.84] \\ 
Sereno (1991), exp. 1 & English & 20 & 60 & 40 & 64 & 24 &   & [0.02 0.33] & [0.04 0.92] \\ 
Forster \& Davis (1991), exp. 5 & English & 24 & 60 & 54 & 72 & 18 &   & [0.02 0.4] & [0.05 0.96] \\ 
Bodner \& Masson (1997), exp. 1 & English & 24 & 60 & 29 & 45 & 16 &   & [0.02 0.4] & [0.05 0.96] \\ 
Bodner \& Masson (1997), exp. 3 & English & 24 & 60 & 36 & 50 & 14 &   & [0.02 0.4] & [0.05 0.96] \\ 
Forster, Mohan, \& Hector (2003), exp. 1 & English & 24 & 60 & 63 & 60 & -3 &   & [0.02 0.4] & [0.05 0.96] \\ 
Kinoshita (2006), exp. 1 & English & 24 & 53 & 32 & 38 & 6 &   & [0.02 0.4] & [0.05 0.96] \\ 
Kinoshita (2006), exp. 2 & English & 24 & 53 & 29 & 59 & 30 & * & [0.02 0.4] & [0.05 0.96] \\ 
Norris \& Kinoshita (2008), exp. 1 & English & 24 & 53 & 35 & 66 & 31 & * & [0.02 0.4] & [0.05 0.96] \\ 
Forster, Davis, Schoknecht, \& Carter (1987), exp. 4 & English & 27 & 60 & 34 & 25 & -9 &   & [0.03 0.46] & [0.05 0.98] \\ 
Forster \& Davis (1984), exp. 1 & English & 28 & 60 & 45 & 38 & -7 &   & [0.03 0.48] & [0.06 0.98] \\ 
Nievas (2010), exp. 1b & Spanish & 30 & 50 & 44 & 65 & 21 & * & [0.03 0.52] & [0.06 0.99] \\ 
Nievas (2010), exp. 2a & Spanish & 30 & 50 or 33\textsuperscript{\textit{1}} & 51 & 58 & 7 &   & [0.03 0.52] & [0.06 0.99] \\ 
Segui \& Grainger (1990), exp. 4 & French & 36 & 60 & 42 & 45 & 3 &   & [0.03 0.63] & [0.07 1] \\ 
Bodner \& Masson (2001), exps. 2A, 2B, 3, \& 6 (average)\textsuperscript{\textit{2}} & English & 40 & 60 & 37 & 69 & 32 & * & [0.03 0.68] & [0.08 1] \\ 
Rajaram \& Neely (1992), exp. 1 & English & 48 & 50 & 30 & 37 & 7 &   & [0.04 0.76] & [0.09 1] \\ 
Rajaram \& Neely (1992), exp. 2 & English & 48 & 50 & 45 & 78 & 33 &   & [0.04 0.76] & [0.09 1] \\ 
\midrule 
\midrule 
Mean &   &   &   & 41 & 55 & 13 &   &   &   \\ 
SD &   &   &   & 10 & 14 & 13 &   &   &   \\ 
Correlation &   &   &   &   &   & 0.46 &   &   &   \\ 
\bottomrule

\end{longtable}

\begin{minipage}{\linewidth}
\textsuperscript{\textit{1}}SOA for each subject determined by pre-test\\
\textsuperscript{\textit{2}}Reported in Masson \& Bodner (2003)\\
\end{minipage}

\elandscape

\section{The present study}\label{sec-study}

It is somewhat surprising that the status of the FAE in masked priming
remains largely unresolved in the literature, given its non-negligible
average magnitude across studies and its theoretical significance in
elucidating the underlying cognitive processes of masked priming.

One possible interpretation of the conflicting past findings revolves
around the fact that only 4 out of 18 studies demonstrate a
statistically significant FAE. Notably, this number potentially
diminishes further when considering that, among these four studies, the
FAE is detected only through the pooling of data across multiple studies
employing a unique alternating-case stimulus presentation (Bodner and
Masson 2001; Masson and Bodner 2003). This line of reasoning suggests a
qualitatively distinct profile between masked and unmasked repetition
priming, with the FAE more firmly established in the latter.

Conversely, one could argue that 15 out of 18 studies exhibit
numerically larger repetition effect sizes for low-frequency words
compared to high-frequency words ---- a pattern that is challenging to
reconcile with a genuine absence of interaction between frequency and
masked repetition. Additionally, the average FAE across all studies
stands at 13 ms, a modest yet non-negligible effect size. In fact, the
naïve assumption that the two conditions are similar enough across
experiments could justify the use of a \emph{t}-test with statistically
significant results: \emph{M\_FAE} = 13, CI\_95\% = {[}7, 20{]}),
\emph{t}(17) = 4.24, \(p=.0005\). These considerations suggest that a
genuine FAE may exist in masked priming but might be smaller than the
magnitudes that are statistically detectable in most previous
experiments. This interpretation is supported by the results from
Adelman et al. (2014) in a large scale, multi-site lab-based study on
orthographic priming. They report a small but reliable FAE, but caution
this effect could simply be an orthographic neighborhood effect
masquerading as a frequency effect, due to the high correlations between
the two variables.

In addition, another potential contributor to past discrepancies is the
reliance on the dated Kučera and Francis (1967) word frequency database,
which 15 out of 18 studies have depended on. This poses a potential
problem, as this frequency database has consistently demonstrated
inferior predictive performance in psycholinguistic experiments,
particularly with low-frequency words, compared to more contemporary
databases (Burgess and Livesay 1998; Zevin and Seidenberg 2002; Balota
et al. 2004; Brysbaert and New 2009; Yap and Balota 2009; Brysbaert and
Cortese 2011; Gimenes and New 2016; Herdağdelen and Marelli 2017;
Brysbaert, Mandera, and Keuleers 2018). Both of these issues are
addressed in the subsequent sections.

\subsection{Issues with frequency databases}\label{sec-study-freq}

Due to the well-documented concerns over the reliability of the Kučera
and Francis (1967) frequency database for psycholinguistic experiments
(Burgess and Livesay 1998; Zevin and Seidenberg 2002; Balota et al.
2004; Brysbaert and New 2009; Yap and Balota 2009; Brysbaert and Cortese
2011; Gimenes and New 2016; Herdağdelen and Marelli 2017; Brysbaert,
Mandera, and Keuleers 2018), our studies exclusively sourced materials
from the HAL (Lund and Burgess 1996) and SUBTLEX\(_{US}\) (Brysbaert and
New 2009) databases, which reflect more recent linguistic usage and
offer better validation in behavioral experiments (e.g., Balota et al.
2004; Brysbaert and New 2009; Yap and Balota 2009; Brysbaert and Cortese
2011; Gimenes and New 2016; Herdağdelen and Marelli 2017). While these
databases outperform Kučera and Francis (1967) in predicting
psycholinguistic task outcomes, it is important to note potential
discrepancies in individual frequency counts, particularly in the low
and mid-frequency ranges. It is possible that this variation,
attributable to the primary genre of their sources (USENET groups for
HAL and movie subtitles for SUBTLEX\(_{US}\)),\footnote{A separate,
  though relevant issue which cannot be addressed here is to how to
  mitigate the discrepancies across the databases available, but see Yap
  and Balota (2009), and Brysbaert and Cortese (2011) for proposals
  about combining the frequency counts from different
  corpora.\label{fn-databases}} may not have an oversized impact on
megastudies with very large word samples (e.g., Balota et al. 2004;
Brysbaert and New 2009; Yap and Balota 2009; Brysbaert and Cortese 2011;
Gimenes and New 2016; Herdağdelen and Marelli 2017). However,
corpus-specific frequency skew can become significant when dealing with
smaller samples of words, as is the case in most masked priming studies
(cf. Adelman et al. (2014)). Table~\ref{tbl-exFreqSkew} illustrates the
potential discrepancy in considering words as high or low frequency
based on the different aforementioned databases.

\begin{longtable}{lrrr}

\caption{\label{tbl-exFreqSkew}Example of frequency count imbalances (in
occurrences per million) across the frequency norms of Kucera \& Francis
(KF), HAL and SUBTLEX\textsubscript{US} for 4 to 6 letter words.}

\tabularnewline

\toprule
Word & KF & HAL & SUBTLEX\textsubscript{US} \\ 
\midrule\addlinespace[2.5pt]
\multicolumn{4}{l}{\emph{Skew in KF}} \\ 
\midrule\addlinespace[2.5pt]
negro & $104$ & $3$ & $5$ \\ 
poet & $99$ & $9$ & $9$ \\ 
mercer & $71$ & $4$ & $2$ \\ 
swung & $48$ & $3$ & $2$ \\ 
mantle & $48$ & $8$ & $2$ \\ 
\midrule\addlinespace[2.5pt]
\multicolumn{4}{l}{\emph{Skew in HAL}} \\ 
\midrule\addlinespace[2.5pt]
web & $6$ & $351$ & $9$ \\ 
user & $4$ & $297$ & $2$ \\ 
mint & $7$ & $211$ & $5$ \\ 
format & $9$ & $198$ & $1$ \\ 
warp & $4$ & $125$ & $5$ \\ 
\midrule\addlinespace[2.5pt]
\multicolumn{4}{l}{\emph{Skew in SUBTLEX\textsubscript{US}}} \\ 
\midrule\addlinespace[2.5pt]
daddy & $4$ & $16$ & $185$ \\ 
bitch & $6$ & $24$ & $169$ \\ 
cute & $5$ & $28$ & $88$ \\ 
pardon & $8$ & $12$ & $65$ \\ 
steal & $5$ & $28$ & $53$ \\ 
\bottomrule

\end{longtable}

\subsection{Issues with statistical power}\label{sec-study-power}

The inconsistency of past findings regarding the FAE in masked priming
has been linked to a potential lack of statistical power in previous
research (Bodner and Masson 1997, 2001; Masson and Bodner 2003; Adelman
et al. 2014). This is a reasonable concern, as interactions like the FAE
often require larger sample sizes for statistical detection (Potvin and
Schutz 2000; Brysbaert and Stevens 2018) compared to main effects. We
outline below three ways in which neglecting statistical power might
frustrate our understanding of FAE in masked repetition priming.

First, our literature review revealed crucial gaps in the reporting of
relevant statistical information, which impedes the assessment of the
statistical power attained by past experiments. The inconsistent
reporting of each conditions' standard deviations (in only 7 out of 18
studies) and the complete absence of reporting of the correlation
structure between conditions complicates power assessments. Researchers
are thus forced to explore a range of plausible values for standard
deviations and correlation structures on their own.

Table~\ref{tbl-litReview} details our attempt to conduct power
simulations for two hypothesized frequency attenuation effect sizes: 15
ms (close to the averaged FAE of 13 ms) and 30 ms (close to the only
three observed statistically significant FAE in English). Standard
deviations (ranging between 60 ms and 180 ms, in 10 ms increments) and
correlation between conditions (uniformly set to range between 0.6 and
0.9, with 0.1 unit increments) were simulated for each study's sample
size, with 10,000 replications for each simulation. These range of
values were derived from our literature review and previous in lab and
online experiments (Petrosino 2020; Petrosino, Sprouse, and Almeida
2023). For each simulated dataset, a paired \emph{t}-test was performed
comparing the repetition effect for high frequency words and low
frequency words. This calculation is mathematically identical to the
interaction term in a 2x2 factorial repeated-measures design\^{}{[}the
resulting \emph{t} value, when squared, is equal to the \emph{F} value
for the interaction calculated in the 2x2 repeated-measures ANOVA), but
it is less computationally expensive to perform in large scale
simulations. Power to detect this interaction was then calculated as the
proportion of statistically significant tests (\alpha = 5\%) obtained
across replications. All else being equal, standard deviations and
correlations between conditions have opposite effects on statistical
power: increases in standard deviations lead to less power, while
increases in correlation between conditions lead to more power.

The results reported in Table~\ref{tbl-litReview} reveal a wide range of
possible statistical power attained by previous studies, depending
solely on the combination of plausible standard deviation and
correlation across conditions. For instance, the study with the smallest
sample size (Forster et al. 1987, N=16) had a 2\% to 24\% chance of
detecting a 15 ms frequency attenuation effect and a 4\% to 84\% chance
to detect a 30 ms effect. Similarly, the study with the largest sample
size (Rajaram and Neely 1992, N=48) exhibited a range of 4\% to 76\% for
a 15 ms frequency attenuation effect and 9\% to 100\% for a 30 ms
effect. As a consequence of the limited reporting of relevant
statistical information in past studies, it is nearly impossible to
determine if any of them were adequately powered to detect the effect of
interest.

A second concern arising from the ambiguity surrounding statistical
power in the literature is the potential impact of a prevalence of
low-powered experiments on the scientific record. An excess of such
experiments increases the risk of observed statistically significant
effects being spurious (Button et al. 2013). As highlighted in
Table~\ref{tbl-litReview}, only 4 out of 18 studies demonstrate a
statistically significant FAE. The absence of clarity regarding the
statistical power of previous research poses challenges in assessing the
likelihood of these significant findings being spurious.

Finally, it is widely acknowledged that experiments with approximately
50\% power are akin to a coin toss in their ability to detect a true
effect (Cohen 1992). A less-appreciated fact is that, in the presence of
even lower power (\textless25\%), statistically significant results can
substantially overestimate the effect size -- a type-M error (Gelman and
Carlin 2014). When power drops to levels below 10\%, a statistically
significant result may occur even when the observed effect goes in the
opposite direction of the true effect -- a type-S error (Gelman and
Carlin 2014). Our power simulations for within-subjects data revealed a
similar relationship between statistical power, type-M, and type-S
errors in line with the observations detailed by Gelman and Carlin
(2014) for the independent samples \(t\)-test. For instance, at 10\%
power (a possibility for virtually all previous studies, as indicated in
Table~\ref{tbl-litReview}), a statistically significant result could
indicate an overestimation of the magnitude of the frequency attenuation
effect by a factor between 2 and 5, with up to a 5\% chance of
incorrectly determining the direction of the effect.

The two studies reported here were designed to mitigate these two
confounding issues: the overreliance on the Kučera and Francis (1967)
frequency data as well as a potential lack of statistical power observed
in previous research. As a large increase in statistical power requires
a large sample size, Experiment 1 aimed to assess the suitability of
using \emph{Labvanced} (Finger et al. 2017), an online platform for
running web browser-based experiments, for running masked priming
studies online.

\section{Experiment 1}\label{sec-exp1}

As evident in Table~\ref{tbl-litReview}, conducting a properly powered
experiment for a FAE close to the averaged value calculated from
previous studies requires a sample size that would be impractical to
pursue in standard university research settings, i.e.~typically quiet
lab rooms with a small number of research computers. In response to this
challenge, the present study was exclusively conducted online,
leveraging the growing trend in online behavioral research facilitated
by HTML5 capabilities and the availability of advanced web software such
as \emph{jsPsych} (de Leeuw 2014), \emph{PsychoJS} (the JavaScript
counterpart of \emph{PsychoPy}, Peirce et al. (2019)), \emph{Gorilla}
(Anwyl-Irvine et al. 2020), and \emph{Labvanced} (Finger et al. 2017).
Notably, three recent studies have already demonstrated the viability of
conducting masked priming experiments online, employing different
software tools: Angele et al. (2023) with \emph{PsychoJS}, Cayado, Wray,
and Stockall (2023) with \emph{Gorilla} and Petrosino, Sprouse, and
Almeida (2023) with \emph{Labvanced}. In this study, we opted for
\emph{Labvanced} (Finger et al. 2017), given our previous successful
experience with it (Petrosino, Sprouse, and Almeida 2023). Similar to
\emph{Gorilla}, \emph{Labvanced} eliminates local installation issues,
ensuring cross-platform consistency and simplifying experimental design
without necessitating proficiency in additional programming languages.

We determined the sample size large enough to guarantee an acceptable
statistical power (\(>80%
\)) by a full-fledged power analysis specifically targeting what we
construed as the smallest theoretically interesting FAE (i.e., 5ms). The
details of the power analysis is available as supplemental material of
the paper. Similarly, the code used for the power simulations, along
with the simulated datasets are available online
(\url{https://osf.io/r7d2q/}). Our analysis identified a sample size of
1,250 participants as optimal, ensuring robust statistical power
especially for a raw FAE equal to or exceeding 10 ms ---- a value
closely aligned with the average FAE calculated from previous studies
(cf. Table~\ref{tbl-litReview}). In light of the limitations in the
temporal accuracy and precision of current online stimulus delivery
programs (observed in several pilots and previous published studies
conducted in our lab), we aimed for an intended sample size of 2,600.
This decision was made to enhance the likelihood of obtaining a sample
size of at least 1,250 participants after applying all the necessary
exclusion criteria to the data. In addition, sample sizes exceeding
1,250 can only help increase the precision of the estimated effect size.

In this experiment, the prime duration was set at 33 ms. The motivation
for the choice of such a short prime duration (as compared to the
literature, in which it is usually between 50 and 60 ms; see
Table~\ref{tbl-litReview}) is threefold. First, previous experiments on
\emph{Labvanced} (Petrosino, Sprouse, and Almeida 2023) showed that, due
to the inherent difficulties in presenting stimuli for very short set
durations in the browser, a longer set duration would increase the
number of trials in which the prime duration would rise above the
subliminal threshold (usually thought to be around 60 ms) due to timing
inaccuracies and missing screen refreshes, which could trigger the
adoption of experiment-wide strategies in the task, and ultimately
contaminate the masked priming response (Zimmerman and Gomez 2012).
Second, Angele et al. (2023), Cayado, Wray, and Stockall (2023) and
Petrosino, Sprouse, and Almeida (2023) have demonstrated that a 33 ms
priming duration is sufficient to elicit repetition priming effects in
online experiments. Finally, setting such a short prime duration
prevents virtually everyone from consciously perceiving the prime word
Nievas (2010), and thus presents a less contaminated estimate of early
putatively automatic processes in word recognition.

\subsection{Methods}\label{sec-exp1-methods}

\subsubsection{Preregistration}\label{sec-exp1-prereg}

We preregistered the results of the power analysis, the goals, the
design and analysis plan for this experiment prior to data collection.
The preregistration, detailing the experimental hypotheses, the desired
sample size as well as the planned analyses is available online
(\url{https://doi.org/10.17605/OSF.IO/3NFQP}).

\subsubsection{Participants}\label{sec-exp1-methods-participants}

Two thousand and six hundred participants (1,445 females; \emph{mean
age} = 42, \emph{sd age} = 14) were recruited on Prolific
(\url{https://www.prolific.com}). Several criteria were selected to
ensure recruitment of native speakers of U.S. English. Participants had
to be born in the Unites States of America, speak English as their first
and only language, and have no self-reported language-related disorder.
We encouraged participants to avoid any sort of distraction throughout
the experiment, and to close any program that may be running in the
background. Because the experiment was run online, participants could
not be monitored during data collection. Finally, to further reduce
variability across participants' devices, we restricted the experiment
to be run on Google Chrome only, which is the most used browser
worldwide (W3 Counter 2023), and reportedly performs better than any
other across operating systems (likely thanks to the \emph{Blink}
engine; see Lukács and Gartus 2023).

\subsubsection{Design}\label{sec-exp1-methods-design}

The masked priming procedure relied on a lexical decision task (LDT), in
which a 2 (frequency: \emph{high} vs \emph{low}) x 2 (prime type:
\emph{repetition} vs \emph{unrelated}) factorial design was used. Both
factors were manipulated within-subjects. The dependent variables were
lexical decision latency (RT, in milliseconds) and error rate (in
percentages).

\subsubsection{Materials}\label{sec-exp1-methods-materials}

One-hundred and four five-letter words, half of low frequency (between 7
and 24 in the SUBTLEX\(_{US}\) frequency per million) and half of high
frequency (between 57 and 2,961 in the SUBTLEX\(_{US}\) frequency per
million) were sampled from ELP (Balota et al. 2007), but this time based
on the SUBTLEX\(_{US}\) frequency counts rather than HAL.
Table~\ref{tbl-words_exp1} shows that although the SUBTLEX\(_{US}\)
frequency ranges of the two conditions were very far from one another
they still show some overlap when HAL frequencies are used. As mentioned
before, this seems to be a general problem when jointly considering
different frequency databases for a smaller set of stimuli that need to
be manipulated and controlled in different ways (see also fn.
\ref{fn-databases} and Adelman et al. (2014)). From each condition, 26
words were selected to be presented as targets and related primes (the
\emph{repetition} condition), and the remaining 26 were presented as
unrelated primes (the \emph{unrelated} condition). All word items were
also controlled for orthographic neighborhood (i.e., Coltheart's
\emph{N}): \(t \approx 0\). All words used were monomorphemic nouns,
adjectives, or verbs, thus excluding particles, prepositions, and
derived or inflected forms.

\begin{longtable}{lrrrrrrrrrrrrr}

\caption{\label{tbl-words_exp1}Experiment 1. Descriptive statistics of
the word items used. For both frequency databases, the word frequencies
were converted to per-million count to ensure cross-comparison.}

\tabularnewline

\toprule
 &  & \multicolumn{4}{c}{\textbf{HAL}} & \multicolumn{4}{c}{\textbf{SUBTLEX\textsubscript{US}}} & \multicolumn{4}{c}{\textbf{Orthographic \emph{N}}} \\ 
\cmidrule(lr){3-6} \cmidrule(lr){7-10} \cmidrule(lr){11-14}
frequency & N & min & max & mean & SD & min & max & mean & SD & min & max & mean & SD \\ 
\midrule\addlinespace[2.5pt]
high & 52 & 45 & 4984 & 573 & 808 & 57 & 2691 & 210 & 388 & 0 & 10 & 3.98 & 2.60 \\ 
low & 52 & 6 & 570 & 64 & 93 & 7 & 24 & 13 & 5 & 0 & 11 & 3.92 & 2.79 \\ 
\bottomrule

\end{longtable}

One-hundred and four five-letter, phono-orthographically legal non-words
were randomly selected from the ELP database as well. Half of them
(i.e., 52) were randomly selected to be presented as targets; the other
half was instead used as unrelated non-word primes. None of the
non-words contained any existing English morpheme. Both the words and
non-words used in the experiments are reported in the appendix below. In
addition, all items had a reported error rate smaller than 10\%, so to
ensure that they were all clearly distinguishable by participants.

\subsubsection{Procedure}\label{sec-exp1-methods-proc}

Each recruited participant was assigned one of two word lists, which
differed only in the relatedness of the prime with respect to the
target; otherwise, the two lists presented the same set of target words
and nonwords (i.e., 104 pairs for each list). In one list, the three
conditions (the high- and low-frequency word conditions, and the
non-word condition) had half of the target items being preceded by
themselves (the \emph{repetition} condition) and half of the target
items being preceded by one of the unrelated primes belonging to the
same frequency bin (the \emph{unrelated} condition). In the other list,
these assignments were reversed. The order of stimulus presentation was
randomized for each participant.

After being recruited in the \emph{Prolific} online platform,
participants were asked to click on a link redirecting them to the
Labvanced online service. During the experiment, they were asked to
perform a lexical decision task by pressing either the `J' (for word) or
`F' (for non-word) keys on their keyboard. Each trial consisted of three
different stimuli appearing at the center of the screen: a series of
five hashes (\#\#\#\#\#) presented for 500 ms, followed by a prime word
presented for 33 ms, and finally the target word; the target word
disappeared from the screen as soon as a decision was made.

Participants were given 5 breaks throughout the experiment. When the
experiment was over, the participants were then redirected to Prolific
in order to validate their submission. The median time to finish the
experiment was 6 minutes. Each participant was paid with a standard rate
of GBP 9/hour.

\subsection{Data analysis}\label{sec-exp1-analysis}

Analysis scripts and an abridged version of the data collected can be
found online (\url{https://osf.io/vn3r2}), and consisted of 297,598
observations in total. We performed the same three steps of analysis
described for experiment 1 (Section~\ref{sec-exp1-analysis}).

\subsubsection{Step 1: subject and item
performance}\label{sec-exp1-analysis-performance}

Item and subject error rates were calculated. The item error rate was
never above 14\%, so no item was excluded from analysis. 19 subjects
were removed because their error rate was above 30\%. Thus, a total of
269,652 observations and 2,593 participants were included in further
analyses.

\subsubsection{Step 2: prime
durations}\label{sec-exp1-analysis-primeTime}

Prime fluctuations were dealt with in the same way as in experiment 1
(Section~\ref{sec-exp1-analysis-primeTime}). The mean (mean = 32.32 ms,
sd = 15) and the median (median = 33 ms) prime durations were closer to
the intended value (33 ms). The same prime duration cut-off set for
experiment 1 (i.e., any trial whose prime duration was out of the
25-60ms range) removed 13 \% of the trials. No participant was excluded,
for a total of 237,287 observations.

\subsubsection{Step 3: RT analysis}\label{sec-exp1-analysis-RT}

After removing the incorrect responses, similarly to experiment 1
(Section~\ref{sec-exp1-analysis-RT}), 0.51\% of the trials were excluded
if their corresponding RT was below 200 ms or above 1800 ms. Finally,
249 subjects were removed because the number of trials within the same
condition was less than 7 (i.e., about half of the total number of
trials being presented within the same condition, i.e.~13). A total of
210,889 observations and 2,341 subjects were included in the statistical
analysis below.

\subsection{Results}\label{sec-exp1-results}

Table~\ref{tbl-exp1-statsResults} below report the descriptive
statistics of the experiment. For each frequency condition, priming
effects were calculated by subtracting the mean RT to the related
condition to the mean RT from the unrelated condition. We ran two
different analyses, for both RT and error data. First, we ran a 2x2
repeated-measures ANOVA (condition, 3 levels: high vs.~low vs.~non-word;
primetype, 2 levels: unrelated vs.~repetition) on the by-subject
averaged data, paired with planned comparisons between related and
unrelated primetype for each condition. In addition, a Generalized
Linear Mixed Model (GLMM) analysis on the raw RT and the raw accuracy
data (rather the by-subject means or error rates used in the ANOVA
analyses). Unlike linear mixed-effect models, GLMMs do not assume normal
distribution of the data, and are therefore particularly useful for
non-normally distributed data such as RTs and accuracy. For the RT data,
a Gamma distribution was used; for the accuracy data, a binomial
distribution was used instead, both with an identity link between fixed
effects and the dependent variable (Lo and Andrews 2015). To prevent
converge failure, the model was kept as simple as possibly, with
condition, primetype, and their interaction as fixed effects, and
subjects and items as random intercepts. Before fitting the model, the
contrasts were also set to sum-to-zero contrasts (i.e., by using the R
function \texttt{contr.sum()}) to facilitation interpretation of main
and interaction effects. The fitting was performed by using the
\texttt{lme4} R-package (Bates et al. 2015) with the Laplace
approximation technique, using 1 million iterations and the BOBYQA
optimizer to help convergence. Then, we used the function
\texttt{Anova()} from the \texttt{car} R-package (Fox and Weisberg 2019)
to obtain estimates and probability values for fixed effects calculated
for Type-III sums of squares.

In the word analysis, the ANOVA analysis showed significant main effects
(condition: \emph{F}(1, 2340)=1572, \emph{p}\textless.0001; primetype:
\emph{F}(1, 2340)=1113, \emph{p}\textless.0001) and their interaction
\emph{F}(1, 2340)=52.48, \emph{p}\textless.0001). Planned comparisons
confirmed statistically significant repetition priming effects for both
word conditions (\emph{MOP\_HF}=18 ms, \emph{CI\_95\%}={[}16 20{]},
\emph{t}(2340)=19.7, \emph{p}\textless.0001; \emph{MOP\_LF} = 28 ms,
\emph{CI\_95\%}={[}26 30{]}, \emph{t}(2340)=27.8,
\emph{p}\textless.0001), with the low-frequency word repetition priming
effect being 10 ms larger than the high-frequency word repetition
priming effect. This FAE effect was statistically significant
(\emph{M\_FAE}=10 ms, \emph{CI\_95\%}={[}7 13{]}), \emph{t}(2340)=7.24,
\emph{p}\textless.0001l). A very small but statistically significant
inhibitory priming effect was observed in the non-word condition
(\emph{MOP\_NW}=-2 ms, \emph{CI\_95\%}={[}-4 0{]}, \emph{t}(2340)=-2.33,
\emph{p}\textless.0001). Similarly, the GLMM analysis confirmed that
both main effects (condition: \(\chi^2=2978.35, p<.0001\); primetype:
\(\chi^2=1531.12, p<.0001\)) and their interaction
(\(\chi^2=870.27, p<.0001\)) were significant.

In the error analysis, the ANOVA analysis revealed significant effects
for both main effects (condition: \emph{F}(1, 2340)=392.5,
\emph{p}\textless.0001; primetype: \emph{F}(1, 2340)=380.5,
\emph{p}\textless.0001) and interaction \emph{F}(1, 2340)=55.47,
\emph{p}\textless.0001). Planned comparison revealed significant priming
effects in the form of fewer errors in the repeated compared to
unrelated trials in the all conditions (high: \emph{t}(2340)=9.95,
\emph{p}\textless.0001; low: \emph{t}(2340)=16.9,
\emph{p}\textless.0001; non-word: \emph{t}(2340)=-3.27, \emph{p}=.001).
As in the ANOVA error analysis, both main effects (condition:
\(\chi^2=30.45, p<.0001\); primetype: \(\chi^2=108.88, p<.0001\)) and
their interaction (\(\chi^2=307.57, p<.0001\)) were significant.

\blandscape

\begin{longtable}{lrrrrrrrrlrrrrl}

\caption{\label{tbl-exp1-statsResults}Experiment 1. Summary of the word
priming results. \emph{Legend.} MOP: magnitude of priming.}

\tabularnewline

\toprule
 & \multicolumn{3}{c}{unrelated RT} & \multicolumn{3}{c}{repetition RT} &  & \multicolumn{4}{c}{priming effects} & \multicolumn{3}{c}{\emph{t}-test} \\ 
\cmidrule(lr){2-4} \cmidrule(lr){5-7} \cmidrule(lr){9-12} \cmidrule(lr){13-15}
factor & mean & SD & Error (\%) & mean & SD & Error (\%) & cor & MOP & 95\% CI & SD\textsubscript{p} & ES & \emph{t} & df & \emph{p} \\ 
\midrule\addlinespace[2.5pt]
high & 573 & 83 & 3 & 555 & 85 & 2 & 0.860 & 18 & [16 20] & 45 & 0.41 & 19.7 & 2340 & 2.88e-80 \\ 
low & 605 & 88 & 6 & 577 & 88 & 3 & 0.850 & 28 & [26 30] & 49 & 0.58 & 27.8 & 2340 & 1.52e-147 \\ 
non-word & 623 & 103 & 4 & 625 & 103 & 4 & 0.910 & -2 & [-4 0] & 43 & -0.05 & -2.33 & 2340 & 0.0197 \\ 
frequency:primetype &   &   &   &   &   &   & 0.029 & 10 & [7 13] & 66 & 0.15 & 7.24 & 2340 & 5.86e-13 \\ 
\bottomrule

\end{longtable}

\elandscape

\subsection{Discussion}\label{sec-exp1-discussion}

Experiment 1 was designed to investigate whether Frequency Attenuation
Effects (FAE) can be detected under masked priming conditions (with SOA
= 33 ms). We managed to recruit a very large sample size (\(N=2341\)) to
ensure adequate statistical power for detecting even small effect sizes.
Our results not only showed statistically significant main effects of
repetition for high and low frequency words alike, but also detected a
statistically significant interaction: the low-frequency condition
yielded priming effects that were 10 ms larger than the high-frequency
condition with a margin of error as narrow as 5 ms. The absence of a
robust non-word masked priming response has been used as an additional
piece of evidence supporting the view that the masked priming response
stems from lexical memory and is devoid of episodic influences (e.g.,
Forster 1999). The results of this experiment align with the previous
evidence in showing at best very small inhibitory masked repetition
priming for non-words, with very high precision: the 95\% CI indicates
the plausible range for the masked repetition priming effect for
non-words to be between -4 and 0 ms when prime duration is 33 ms.

\section{Experiment 2}\label{sec-exp2}

The results of experiment 1 showed that with the right sample size, the
repetition priming response may indeed be modulated by word frequency.
However, the experiment was carried out with a unusually short prime
duration (i.e., 33 ms). In previous pilot experiments carried out at
longer durations (e.g., 48 ms), we observed that the distribution of the
prime duration tended to be more positively skewed, with a substantial
increase of the number of trials with a prime duration above the
subliminal threshold of 60 ms. While being aware of these methodological
limitations, we carried out a follow-up experiment that elicited the
repetition priming response to the same materials used in experiment 1,
with a longer prime duration (i.e., 48 ms), which is more in line with
the literature on the topic. The aim for this experiment was three-fold.
From the methodological point of view, it may show that a shorter prime
duration in online masked priming experiment may prevent unnecessary
data loss, while still providing reliable results. From the theoretical
point of view, it may provide further evidence on two separate issues.
First, it may inform on the interaction between priming and prime
duration. In particular, we expect the longer prime duration to elicit a
larger facilitation, and therefore a larger magnitude of priming.
Second, and more importantly for the question being asked here, it may
further inform on the size of the interaction between priming and word
frequency.

\subsection{Methods}\label{sec-exp2-methods}

\subsubsection{Participants}\label{sec-exp2-methods-participants}

Two thousand and six hundred participants (1,551 females; \emph{mean
age} = 39, \emph{sd age} = 12) were recruited on Prolific
(\url{https://www.prolific.com}) with the same criteria specified for
experiment 1 (Section~\ref{sec-exp1-methods-participants}). Prior to
recruitment, the participants participating in experiment 1 were
excluded from the pool, so that the participants recruited for
experiment 2 was completely different from the participants recruited
for experiment 1.

\subsubsection{Design}\label{sec-exp2-methods-design}

The experimental design was identical to experiment 1.

\subsubsection{Materials}\label{sec-exp2-methods-materials}

The experimental items were the same as experiment 1.

\subsubsection{Procedure}\label{sec-exp2-methods-proc}

Experiment 2 followed the same procedures as experiment 1 (see
Section~\ref{sec-exp1-methods-proc}). The median time to finish the
experiment was the same as experiment 1 (i.e., about 6 minutes).

\subsection{Data analysis}\label{sec-exp2-analysis}

Analysis scripts and an abridged version of the data collected can be
found online (\url{https://osf.io/k3gpc}), and consisted of 295,940
observations in total. We performed the same three steps of analysis
described for experiment 1 (Section~\ref{sec-exp1-analysis}).

\subsubsection{Step 1: subject and item
performance}\label{sec-exp2-analysis-performance}

Item and subject error rates were calculated. Similarly to experiment 1,
the item error rate was never above 12\%, so no item was excluded from
analysis. 39 subjects were removed because their error rate was above
30\%. Thus, a total of 265,982 observations and 2,558 participants were
included in further analyses.

\subsubsection{Step 2: prime
durations}\label{sec-exp2-analysis-primeTime}

Prime fluctuations were dealt with in the same way as in experiment 1
(Section~\ref{sec-exp1-analysis-primeTime}). The mean (mean = 50.11 ms,
sd = 11.13) and the median (median = 50 ms) prime durations were closer
to the intended value (50 ms). The same prime duration cut-off set for
experiment 1 (i.e., any trial whose prime duration was out of the
25-60ms range) removed 20 \% of the trials. As compared to experiment 1,
experiment 2 had therefore a 7\% larger percentage of removed
out-of-range trials, the majority of which (18\%) were above the
subliminal threshold. This is in sharp contrast with the distribution of
the prime durations in experiment 1, where the trials above the range
amounted to only 0\% of the dataset. As observed in previous studies and
pilot conducted in our lab, this distribution suggests that setting the
prime duration closer to either limit of a given range has the side
effect of allowing for more fluctuations around either limit, thus
potentially leading to greater data loss (see also further below). No
participant was excluded, for a total of 213,078 observations.

\subsubsection{Step 3: RT analysis}\label{sec-exp2-analysis-RT}

After removing the incorrect responses, similarly to experiment 1
(Section~\ref{sec-exp1-analysis-RT}), 0.59\% of the trials were excluded
if their corresponding RT was below 200 ms or above 1800 ms. Finally,
634 subjects were removed because the number of trials within the same
condition was less than 7 (i.e., about half of the total number of
trials being presented within the same condition, i.e.~13). A total of
168,195 observations and 1,924 subjects were included in the statistical
analysis below. The substantial number of subjects being removed at this
final stage was the ultimate side effect of the increased number of
out-of-range trials that were removed during the previous step of
analysis, and provides further evidence of the risks with setting the
prime duration closer to the upper bound of the subliminal range.

\subsection{Results}\label{sec-exp2-results}

Table~\ref{tbl-exp2-statsResults} below report the descriptive
statistics of the experiment. For experiment 2, we ran the same
statistical analyses as experiment 1. In the word analysis, a 2x2
repeated-measures ANOVA revealed significant main effects (condition:
\emph{F}(1, 1923)=987.4, \emph{p}\textless.0001; primetype: \emph{F}(1,
1923)=1447, \emph{p}\textless.0001) and interaction \emph{F}(1,
1923)=36.82, \emph{p}\textless.0001). Planned comparisons confirmed
statistically significant repetition priming effects for both word
conditions (\emph{MOP\_HF}=26 ms, \emph{CI\_95\%}={[}24 28{]},
\emph{t}(1923)=24.6, \emph{p}\textless.0001; \emph{MOP\_LF} = 35 ms,
\emph{CI\_95\%}={[}33 37{]}, \emph{t}(1923)=30.2,
\emph{p}\textless.0001), with the low-frequency word repetition priming
effect being 10 ms larger than the high-frequency word repetition
priming effect. This FAE effect was statistically significant
(\emph{M\_FAE}=9 ms, \emph{CI\_95\%}={[}6 12{]}), \emph{t}(1920)=6.07,
\emph{p}\textless.0001). A very small but statistically significant
inhibitory priming effect was observed in the non-word condition
(\emph{MOP\_NW}=-4 ms, \emph{CI\_95\%}={[}-6 -2{]},
\emph{t}(1923)=-3.53, \emph{p}=.0004). The GLMM analysis confirmed
significant effects for condition (\(\chi^2=2613.3, p<.0001\)),
primetype (\(\chi^2=1700.5, p<.0001\)), and their interaction
(\(\chi^2=1158.9, p<.0001\)).

In the error analysis, the ANOVA showed significant effects for both
main effects (condition: \emph{F}(1, 2340)=392.5,
\emph{p}\textless.0001; primetype: \emph{F}(1, 2340)=380.5,
\emph{p}\textless.0001) and interaction \emph{F}(1, 2340)=55.47,
\emph{p}\textless.0001). Planned comparisons confirmed significant
priming effects in the form of fewer errors in repeated compared to
unrelated trials in the all conditions (high: \emph{t}(1923)=9.30,
\emph{p}\textless.0001; low: \emph{t}(1923)=15.8,
\emph{p}\textless.0001; non-word: \emph{t}(1923)=-2.32, \emph{p}=.002).
Similar results were obtained in the GLMM error analysis (condition:
\(\chi^2=51.50, p<.0001\); primetype: \(\chi^2=97.42, p<.0001\);
interaction: \(\chi^2=283.77, p<.0001\).

\blandscape

\begin{longtable}{lrrrrrrrrlrrrrr}

\caption{\label{tbl-exp2-statsResults}Experiment 2. Summary of the word
priming results. \emph{Legend.} MOP: magnitude of priming.}

\tabularnewline

\toprule
 & \multicolumn{3}{c}{unrelated RT} & \multicolumn{3}{c}{repetition RT} &  & \multicolumn{4}{c}{priming effects} & \multicolumn{3}{c}{\emph{t}-test} \\ 
\cmidrule(lr){2-4} \cmidrule(lr){5-7} \cmidrule(lr){9-12} \cmidrule(lr){13-15}
factor & mean & SD & Error (\%) & mean & SD & Error (\%) & cor & MOP & 95\% CI & SD\textsubscript{p} & ES & \emph{t} & df & \emph{p} \\ 
\midrule\addlinespace[2.5pt]
high & 574 & 83 & 3 & 548 & 83 & 2 & 0.850 & 26 & [24 28] & 46 & 0.56 & 24.595576 & 1923 & 2.27e-116 \\ 
low & 605 & 89 & 6 & 570 & 89 & 3 & 0.840 & 35 & [33 37] & 51 & 0.69 & 30.185474 & 1923 & 3.51e-164 \\ 
non-word & 629 & 108 & 4 & 633 & 110 & 5 & 0.900 & -4 & [-6 -2] & 50 & -0.08 & -3.525726 & 1923 & 4.32e-04 \\ 
frequency:primetype &   &   &   &   &   &   & 0.024 & 9 & [6 12] & 68 & 0.13 & 6.068254 & 1923 & 1.55e-09 \\ 
\bottomrule

\end{longtable}

\elandscape

\subsection{Discussion}\label{sec-exp2-discussion}

Experiment 2 was designed to investigate whether a frequency attenuation
effect akin to the one detected in experiment 1 at SOA of 33 ms can be
detected at the longer, and most commonly used SOA of 48 ms. To this
end, we used the same stimuli and recruited the same sample size as
experiment 1, and only set the prime duration accordingly. First, we
found that the masked repetition priming to both high and low-frequency
words were respectively about 8 ms larger in experiment 2 than in
experiment 1, resulting from the longer prime duration (Forster, Mohan,
and Hector 2003). Nevertheless, the significant interaction effect size
amounted to 9 ms, which is only 1 ms away from the estimate of the
interaction in experiment 1 (10 ms), but well within its 95\% CI (which
ranged from 7 ms to 13 ms). Finally, similarly to experiment 1, the
magnitude of non-word masked repetition priming response in experiment 2
was inhibitory, though slightly larger (-4 ms, as compared to -2 ms of
experiment 1).

At the face value, experiment 2 is very similar to experiment 1 on
several respects. First, they involved the exact the same stimuli and
the same sample size from the same pool (i.e., the Prolific pool;
crucially though, different participants were recruited for each
experiment). Second, they were both analyzed with the exact same
analysis pipeline, and, in particular, with the same criterion to detect
trials with outlying prime durations. Third, the estimates (means, SDs,
error percentages, and correlations) of both experiments are numerically
very close to one another, with the maximal differences almost
exclusively present in the repetition conditions. For these reasons, one
may therefore argue against its methodological and theoretical validity,
and may deem the differences of the magnitude of priming across the two
experiments might have just been due to sampling error having two major
sources. The first source of sampling error is the inherent imprecision
in the presentation of the prime. Regardless of the preset duration of
the prime, there was an inherent inaccuracy in the actual presentation
of the prime in both experiments. The analysis pipeline (and, in
particular, with the criterion to detect trials with outlying prime
durations) described in detail above might have therefore led to two
datasets with a similar distribution of the prime durations.
Figure~\ref{fig-prime-distributions} shows the distributions of the
prime durations for both datasets after the last analysis step. The two
distributions minimally overlap, with most of the trials peaking at
about the relative preset duration (i.e., 33 and 48 ms). The difference
between the two distribution was also statistically significant:
\(t(394431)=-868.47, p<.0001\).

\phantomsection\label{cell-fig-prime-distributions}
\begin{figure}[H]

\centering{

\includegraphics{index_files/figure-pdf/fig-prime-distributions-1.pdf}

}

\caption{\label{fig-prime-distributions}Distribution of the prime
durations across the two experiments.}

\end{figure}%

The second source of sampling error is equally unavoidable and present
in every study: participant selection, and is ensured by the fact that
recruiting sample sizes approaching infinity is virtually impossible.
One way to gauge the extent to which such source of sampling error might
have affected the differences in the effect sizes of the two experiments
is to calculate the \emph{prediction intervals} on the results of
experiment 1, and check if the results of experiment 2 fall within them.
Unlike confidence intervals, the lesser-known prediction intervals are
calculated around the original study's mean (rather than the population
mean), and express the amount of deviation that a future replication of
the study may allow for due to sampling error (Spence and Stanley 2016).
In our case, if the results of experiment 2 falls within the prediction
interval above, it would suggest that the deviation between the two
experiment was just due to sampling error, and therefore experiment 2 is
a replication of experiment 1. Conversely, if the results of experiment
2 falls within the prediction interval above, it would suggest that the
deviation between the two studies could not be due to sampling error,
and therefore experiment 2 is not a replication of experiment 1. The
prediction intervals for the means of the three conditions and the
interaction between the two word condition (i.e., FAE) of experiment 1
were calculated by using the function \texttt{pi.m()} of the R package
\texttt{predictionInterval} (available in the
\href{https://cran.r-project.org/web/packages/predictionInterval/predictionInterval.pdf}{CRAN
repository}), and are reported in Table~\ref{tbl-pis} below. In
experiment 2, while the estimates of the non-word condition and the FAE
fall within the relative prediction interval (\emph{MOP\_NW} = -4 ms,
\emph{PI\_95\%} = {[}-5 1{]}; \emph{FAE} = 9 ms, \emph{PI\_95\%} = {[}6
14{]}), the estimates of both word conditions are respectively 5 and 4
ms above the corresponding PIs (\emph{MOP\_HF} = 26 ms, \emph{PI\_95\%}
= {[}15 21{]}; \emph{MOP\_LF} = 35 ms, \emph{PI\_95\%} = {[}25 31{]}).
Our interpretation of these results is three-fold. First, the fact that
the the word priming effects of experiment 2 (for both the high and
low-frequency conditions) are outside the prediction intervals of
experiment 1 suggest that the effects elicited in experiment 2 may not
be considered as mere replications of the effects elicited in experiment
1. Rather, they corroborate the view that priming may significantly
benefit from a longer prime duration. Second, the masked repetition
non-word priming effect size being replicated in both experiments
further corroborates the commonly-accepted view that the masked priming
response taps into lexical memory, rather than pure orthographic
decoding. Finally, and more importantly for the main purpose of the
paper, the estimate of the FAE of experiment falling within the relative
prediction interval further confirms the presence of an interaction
effect between priming and frequency, while not being contingent on the
duration of the prime.

\begin{longtable}{lrrrrl}

\caption{\label{tbl-pis}Prediction intervals calculated on the means and
standard deviations (SD) of the conditions tested in experiment 1.}

\tabularnewline

\toprule
 &  &  & \multicolumn{2}{c}{sample size} &  \\ 
\cmidrule(lr){4-5}
factor & mean & SD & experiment 1 & experiment 2 & 95\% PI \\ 
\midrule\addlinespace[2.5pt]
high & 18 & 45 & 2341 & 1924 & [15 21] \\ 
low & 28 & 49 & 2341 & 1924 & [25 31] \\ 
non-word & -2 & 43 & 2341 & 1924 & [-5 1] \\ 
frequency:primetype & 10 & 66 & 2341 & 1924 & [6 14] \\ 
\bottomrule

\end{longtable}

\section{General discussion}\label{sec-discussion}

The repetition priming response stands as a cornerstone in
psycholinguistic investigations, offering insights into the mechanisms
governing word recognition. An ongoing debate surrounds the
interpretation of these effects, particularly concerning their source in
the memory system. On the one hand, \emph{interactive activation models}
(McClelland and Rumelhart 1981; Grainger and Jacobs 1996; Coltheart et
al. 2001) posit a lexical source for repetition priming effects, either
in terms of temporarily raised resting activation levels for lexical
nodes in unmasked priming, or as a head start in the retrieval process
in masked priming. \emph{Episodic} and \emph{memory recruitment models}
(Jacoby and Dallas 1981; Jacoby 1983; Bodner and Masson 1997; Masson and
Bodner 2003; Bodner and Masson 2014) on the other hand, invoke a
non-lexical source for the repetition effect, namely an episodic or
episodic-like memory resource formed upon brief exposure to the prime
word that can be recruited during the processing of the target item.
Crucially, both models predict a single mechanism underlying masked and
unmasked priming. Differential mechanisms between unmasked and masked
repetition priming, however, are predicted by the \emph{entry-opening
model} (Forster and Davis 1984), which propose both lexical and episodic
sources of priming effects.

Thus, the existence of qualitatively distinct outcomes in masked and
unmasked priming presented a direct challenge to some, but not all of
these models. One such finding is the \emph{Frequency Attenuation
Effect} (FAE), in which higher frequency words exhibit smaller
repetition effects compared to lower frequency words. The FAE has been
described as observable only in unmasked priming since the work of
Forster and Davis (1984), who demonstrated that when the prime word is
presented very briefly (SOA \(<\) 60 ms), it becomes masked by the
target word, and this is hypothesized to prevent the conscious encoding
of the prime. Under such conditions, the FAE purportedly disappears.
Forster and Davis (1984) argued that this potentially shows that the FAE
is subserved by a different type of memory source (perhaps episodic)
than the masked repetition priming response. This conclusion, however,
is the source of ongoing debates (see Table~\ref{tbl-litReview} for
review of past findings), which the two experiments reported here were
meant to address.

Within this research landscape, our experiments targeting the frequency
sensitivity of the repetition effect under masked conditions contribute
methodological and theoretical insights. Methodologically, our results
help establish the viability and reliability of online data collection
for the masked priming paradigm, building on the work of Angele et al.
(2023), Cayado, Wray, and Stockall (2023) and Petrosino, Sprouse, and
Almeida (2023).

In the same vein, the FAEs observed in experiments 1 and 2 have
important theoretical ramifications. The historical belief that FAE
fails to obtain in masked priming arose from a lack of statistically
significant results. These were possibly rooted in the reliance of
outdated frequency corpora by earlier experiments or inadequate
statistical power to detect plausible effect sizes. Our design addressed
these concerns, yielding statistically significant FAE results aligning
with the literature's average effect (see Table~\ref{tbl-litReview}; the
95\% CI implies that the FAE is unlikely to be larger than 13 ms with a
33 ms prime duration). These results challenge the supposed qualitative
distinction between masked and unmasked repetition priming cleaved by
the FAE, complicating the rejection of single-mechanism theories, and
suggesting that \emph{interactive-activation models} and \emph{memory
recruitment models} may yet offer unifying explanations for masked and
unmasked priming.

Similarly, our results also challenge the entry-opening model's
prediction of the absence of FAE in masked priming. One potential way of
dealing with this in the \emph{entry opening model} is to claim that
masked priming severely reduces, but does not entirely eliminate, the
use of sources other than lexical memory (see Forster 1998; Forster,
Mohan, and Hector 2003, for proposals along this line). Alternatively,
within the entry-opening model, the results of this study may be
explained by the frequency-based mechanism occurring in the fast search
stage. A potential mechanism in this direction was already hinted at by
Forster and Davis (1984) themselves, and consists of a procedure,
whereby during the fast search stage, the entry of a prime word is
promoted to the top position of the search list. As a consequence,
low-frequency words (which are fairly low in the search list) will
benefit from such promotion procedure more than high-frequency words
(which are instead already in higher positions), thus ultimately giving
rise to the FAE.

While our findings present a compelling case for the presence of FAE in
masked priming that is seemingly parallel to the unmasked case,
questions about potential mechanistic differences persist. The larger
sample size needed for masked FAEs raises intriguing considerations
about the influence of memory sources and warrants further
investigation. For example, there is independent evidence for different
mechanisms in masked and unmasked repetition priming from RT
distributional analyses (cf. Gomez, Perea, and Ratcliff (2013)) that
suggests that repetition priming under masked conditions affect
primarily the encoding stage of the stimulus. Given that frequency is
often associated with facilitation of encoding, our results could help
support this view. Additionally, the trivially small inhibitory effect
sizes of non-word masked repetition priming in experiments 1 and 2 align
with the trend (overwhelmingly shown in the literature) that
facilitatory effect may be exclusive to unmasked designs (Forster 1998;
Forster, Mohan, and Hector 2003; but see Masson and Bodner 2003), and
suggests avenues for future exploration.

Finally, the finding that the FAE occurs under masked priming conditions
may impact our understanding of masked morphological priming. In this
literature, there is a unresolved question about the ability of affixes
to elicit masked morphological priming results (for a review, Amenta and
Crepaldi 2012). In English, the evidence seems to indicate that only
stems, but not affixes, have the ability to prime entries across the
lexicon. This finding can and has been used to support models in which
affixes are initially stripped before stems are accessed in the lexicon
(Taft and Forster 1975; Forster and Azuma 2000; Stockall and Marantz
2006). However, stems and affixes do also have a large frequency
imbalance, with most affixes being substantially more frequent that most
stems. The observation of FAE under masked priming can provide an
alternative reason for why masked stem morphological priming is well
attested but masked affix morphological priming is not: the latter could
be due to a ceiling frequency attenuation effect. This is an intriguing
possibility that must be left for future work to explore.

In summary, our study successfully replicated and expanded upon the work
of Angele et al. (2023), Cayado, Wray, and Stockall (2023) and
Petrosino, Sprouse, and Almeida (2023), confirming the viability of
observing repetition priming effects in masked priming experiments
conducted online with a brief SOA of 33 ms. Notably, we addressed a
lingering question in the literature by establishing the presence of the
Frequency Attenuation Effect (FAE) under masked conditions. The use of
large online samples proved instrumental in overcoming the longstanding
challenge of insufficient statistical power to detect interactions in
factorial designs, which we believe had impeded previous investigations
into detecting the FAE in masked priming.

\newpage{}

\section*{References}\label{references}
\addcontentsline{toc}{section}{References}

\phantomsection\label{refs}
\begin{CSLReferences}{1}{0}
\bibitem[\citeproctext]{ref-Adelman2014}
Adelman, James S, Rebecca L Johnson, Samantha F McCormick, Meredith
McKague, Sachiko Kinoshita, Jeffrey S Bowers, Jason R Perry, et al.
2014. {``A Behavioral Database for Masked Form Priming.''}
\emph{Behavior Research Methods} 46: 1052--67.

\bibitem[\citeproctext]{ref-AmentaCrepaldi2012}
Amenta, Simona, and Davide Crepaldi. 2012. {``Morphological Processing
as We Know It: An Analytical Review of Morphological Effects in Visual
Word Identification.''} \emph{Frontiers in Psychology} 3: 232.

\bibitem[\citeproctext]{ref-Angele2023}
Angele, Bernhard, Ana Baciero, Pablo Gómez, and Manuel Perea. 2023.
{``Does Online Masked Priming Pass the Test? The Effects of Prime
Exposure Duration on Masked Identity Priming.''} \emph{Behavior Research
Methods} 55 (1): 151--67.
\url{https://doi.org/10.3758/s13428-021-01742-y}.

\bibitem[\citeproctext]{ref-Anwyl2020}
Anwyl-Irvine, Alexander L, Jessica Massonnié, Adam Flitton, Natasha
Kirkham, and Jo K Evershed. 2020. {``Gorilla in Our Midst: An Online
Behavioral Experiment Builder.''} \emph{Behavior Research Methods} 52:
388--407.

\bibitem[\citeproctext]{ref-Balota2004}
Balota, David A., Michael J. Cortese, Susan D. Sergent-Marshall, Daniel
H. Spieler, and Melvin J. Yap. 2004. {``Visual Word Recognition of
Single-Syllable Words.''} \emph{Journal of Experimental Psychology:
General} 133 (2): 283.

\bibitem[\citeproctext]{ref-balota2007}
Balota, David A., Melvin J. Yap, Keith A. Hutchison, Michael J. Cortese,
Brett Kessler, Bjorn Loftis, James H. Neely, Douglas L. Nelson, Greg B.
Simpson, and Rebecca Treiman. 2007. {``The English Lexicon Project.''}
\emph{Behavior Research Methods} 39 (3): 445--59.
\url{https://doi.org/10.3758/bf03193014}.

\bibitem[\citeproctext]{ref-BatesEtal2015}
Bates, Douglas, Reinhold Kliegl, Shravan Vasishth, and Harald Baayen.
2015. {``Parsimonious Mixed Models.''}
\url{https://arxiv.org/abs/1506.04967}.

\bibitem[\citeproctext]{ref-Bodner2014}
Bodner, Glen E., and Michael E. J Masson. 2014. {``Memory Recruitment: A
Backward Idea about Masked Priming.''} In \emph{Psychology of Learning
and Motivation}, 61:179--213. Elsevier.

\bibitem[\citeproctext]{ref-BodnerMasson1997}
Bodner, Glen E., and Michael E. J. Masson. 1997. {``Masked Repetition
Priming of Words and Nonwords: Evidence for a Nonlexical Basis for
Priming.''} \emph{Journal of Memory and Language} 37 (2): 268--93.
\url{https://doi.org/10.1006/jmla.1996.2507}.

\bibitem[\citeproctext]{ref-BodnerMasson2001}
---------. 2001. {``Prime Validity Affects Masked Repetition Priming:
Evidence for an Episodic Resource Account of Priming.''} \emph{Journal
of Memory and Language} 45 (4): 616--47.
\url{https://doi.org/10.1006/jmla.2001.2791}.

\bibitem[\citeproctext]{ref-Brysbaert2011}
Brysbaert, Marc, and Michael J Cortese. 2011. {``Do the Effects of
Subjective Frequency and Age of Acquisition Survive Better Word
Frequency Norms?''} \emph{Quarterly Journal of Experimental Psychology}
64 (3): 545--59.

\bibitem[\citeproctext]{ref-Brysbaert2018}
Brysbaert, Marc, Paweł Mandera, and Emmanuel Keuleers. 2018. {``The Word
Frequency Effect in Word Processing: An Updated Review.''} \emph{Current
Directions in Psychological Science} 27 (1): 45--50.

\bibitem[\citeproctext]{ref-BrysbaertNew2009}
Brysbaert, Marc, and Boris New. 2009. {``Moving Beyond Ku{č}era and
Francis: A Critical Evaluation of Current Word Frequency Norms and the
Introduction of a New and Improved Word Frequency Measure for American
English.''} \emph{Behavior Research Methods} 41 (4): 977--90.
\url{https://doi.org/10.3758/brm.41.4.977}.

\bibitem[\citeproctext]{ref-BrysbaertStevens2018}
Brysbaert, Marc, and Michaël Stevens. 2018. {``Power Analysis and Effect
Size in Mixed Effects Models: A Tutorial.''} \emph{Journal of Cognition}
1 (1). \url{https://doi.org/10.5334/joc.10}.

\bibitem[\citeproctext]{ref-Burgess1998}
Burgess, Curt, and Kay Livesay. 1998. {``The Effect of Corpus Size in
Predicting Reaction Time in a Basic Word Recognition Task: Moving on
from Ku{č}era and Francis.''} \emph{Behavior Research Methods,
Instruments, \& Computers} 30 (2): 272--77.

\bibitem[\citeproctext]{ref-Button2013}
Button, Katherine S, John PA Ioannidis, Claire Mokrysz, Brian A Nosek,
Jonathan Flint, Emma SJ Robinson, and Marcus R Munafò. 2013. {``Power
Failure: Why Small Sample Size Undermines the Reliability of
Neuroscience.''} \emph{Nature Reviews Neuroscience} 14 (5): 365--76.

\bibitem[\citeproctext]{ref-Cayado2023}
Cayado, Dave Kenneth Tayao, Samantha Wray, and Linnaea Stockall. 2023.
{``Does Linear Position Matter for Morphological Processing? Evidence
from a Tagalog Masked Priming Experiment.''} \emph{Language, Cognition
and Neuroscience}, 1--16.

\bibitem[\citeproctext]{ref-Cohen1992}
Cohen, Jacob. 1992. {``A Power Primer.''} \emph{Psychological Bulletin}
112 (1): 155--59. \url{https://doi.org/10.1037/0033-2909.112.1.155}.

\bibitem[\citeproctext]{ref-ColtheartEtal2001}
Coltheart, Max, Kathleen Rastle, Conrad Perry, Robyn Langdon, and
Johannes Ziegler. 2001. {``DRC: A Dual Route Cascaded Model of Visual
Word Recognition and Reading Aloud.''} \emph{Psychological Review} 108
(1): 204--56.

\bibitem[\citeproctext]{ref-deLeeuw2014}
de Leeuw, Joshua R. 2014. {``jsPsych: A JavaScript Library for Creating
Behavioral Experiments in a Web Browser.''} \emph{Behavior Research
Methods} 47 (1): 1--12. \url{https://doi.org/10.3758/s13428-014-0458-y}.

\bibitem[\citeproctext]{ref-EvettHumphreys1981}
Evett, Lindsay J., and Glyn W. Humphreys. 1981. {``The Use of Abstract
Graphemic Information in Lexical Access.''} \emph{The Quarterly Journal
of Experimental Psychology} 33 (4): 325--50.

\bibitem[\citeproctext]{ref-Labvanced2017}
Finger, Holger, Caspar Goeke, Dorena Diekamp, Kai Standvoß, and Peter
König. 2017. {``LabVanced: A Unified JavaScript Framework for Online
Studies.''} In \emph{2017 International Conference on Computational
Social Science}. Cologne, Germany.

\bibitem[\citeproctext]{ref-Forster1998}
Forster, Kenneth I. 1998. {``The Pros and Cons of Masked Priming.''}
\emph{Journal of Psycholinguistic Research} 27 (2): 203--33.

\bibitem[\citeproctext]{ref-Forster1999}
---------. 1999. {``Microgenesis of Priming Effects in Lexical
Access.''} \emph{Brain and Language} 68: 5--15.

\bibitem[\citeproctext]{ref-ForsterAzuma2000}
Forster, Kenneth I., and Tamiko Azuma. 2000. {``Masked Priming for
Prefixed Words with Bound Stems: Does Submit Prime Permit?''}
\emph{Language and Cognitive Processes} 15 (4-5): 539--61.

\bibitem[\citeproctext]{ref-ForsterDavis1984}
Forster, Kenneth I., and Chris Davis. 1984. {``Repetition Priming and
Frequency Attenuation in Lexical Access.''} \emph{Journal of
Experimental Psychology: Learning, Memory, and Cognition} 10 (4): 680.

\bibitem[\citeproctext]{ref-ForsterDavis1991}
Forster, Kenneth I., and Christopher Davis. 1991. {``The Density
Constraint on Form-Priming in the Naming Task: Interference Effects from
a Masked Prime.''} \emph{Journal of Memory and Language} 30 (1): 1--25.
\url{https://doi.org/10.1016/0749-596x(91)90008-8}.

\bibitem[\citeproctext]{ref-ForsterEtal1987}
Forster, Kenneth I., C. Davis, C. Schoknecht, and R. Carter. 1987.
{``Masked Priming with Graphemically Related Forms: Repetition or
Partial Activation?''} \emph{The Quarterly Journal of Experimental
Psychology Section A} 39 (2): 211--51.
\url{https://doi.org/10.1080/14640748708401785}.

\bibitem[\citeproctext]{ref-ForsterEtal2003}
Forster, Kenneth I., Kathleen Mohan, and Jo Hector. 2003. {``The
Mechanics of Masked Priming.''} In \emph{Masked Priming: The State of
the Art}, edited by Sachiko Kinoshita and Stephen J. Lupker, 3--37. New
York, NY/Hove, UK: Psychology Press.

\bibitem[\citeproctext]{ref-FoxWeisberg2019}
Fox, John, and Sanford Weisberg. 2019. \emph{An {R} Companion to Applied
Regression}. Third. Thousand Oaks {CA}: Sage.
\url{https://socialsciences.mcmaster.ca/jfox/Books/Companion/}.

\bibitem[\citeproctext]{ref-GelmanCarlin2014}
Gelman, Andrew, and John Carlin. 2014. {``Beyond Power Calculations.''}
\emph{Perspectives on Psychological Science} 9 (6): 641--51.
\url{https://doi.org/10.1177/1745691614551642}.

\bibitem[\citeproctext]{ref-Gimenes2016}
Gimenes, Manuel, and Boris New. 2016. {``Worldlex: Twitter and Blog Word
Frequencies for 66 Languages.''} \emph{Behavior Research Methods} 48:
963--72.

\bibitem[\citeproctext]{ref-Gomez2013}
Gomez, Pablo, Manuel Perea, and Roger Ratcliff. 2013. {``A Diffusion
Model Account of Masked Versus Unmasked Priming: Are They Qualitatively
Different?''} \emph{Journal of Experimental Psychology: Human Perception
and Performance} 39 (6): 1731.

\bibitem[\citeproctext]{ref-GraingerJacobs1996}
Grainger, Jonathan, and Arthur M. Jacobs. 1996. {``Orthographic
Processing in Visual Word Recognition: A Multiple Read-Out Model.''}
\emph{Psychological Review} 103 (3): 518.

\bibitem[\citeproctext]{ref-Herdaugdelen2017}
Herdağdelen, Amaç, and Marco Marelli. 2017. {``Social Media and Language
Processing: How Facebook and Twitter Provide the Best Frequency
Estimates for Studying Word Recognition.''} \emph{Cognitive Science} 41
(4): 976--95.

\bibitem[\citeproctext]{ref-Jacoby1983}
Jacoby, Larry L. 1983. {``Remembering the Data: Analyzing Interactive
Processes in Reading.''} \emph{Journal of Verbal Learning and Verbal
Behavior} 22 (5): 485--508.

\bibitem[\citeproctext]{ref-Jacoby1981}
Jacoby, Larry L, and Mark Dallas. 1981. {``On the Relationship Between
Autobiographical Memory and Perceptual Learning.''} \emph{Journal of
Experimental Psychology: General} 110 (3): 306.

\bibitem[\citeproctext]{ref-Kinoshita2006}
Kinoshita, Sachiko. 2006. {``Additive and Interactive Effects of Word
Frequency and Masked Repetition in the Lexical Decision Task.''}
\emph{Psychonomic Bulletin \& Review} 13 (4): 668--73.
\url{https://doi.org/10.3758/bf03193979}.

\bibitem[\citeproctext]{ref-KuceraFrancis1967}
Kučera, J., and W. N. Francis. 1967. \emph{Computational Analysis of
Present Day American English}. Providence, RI: Brown University Press.

\bibitem[\citeproctext]{ref-LoAndrews2015}
Lo, Steson, and Sally Andrews. 2015. {``To Transform or Not to
Transform: Using Generalized Linear Mixed Models to Analyse Reaction
Time Data.''} \emph{Frontiers in Psychology} 6: 1171.

\bibitem[\citeproctext]{ref-LukacsGaspar2023}
Lukács, Gáspár, and Andreas Gartus. 2023. {``Precise Display Time
Measurement in JavaScript for Web-Based Experiments.''} \emph{Behavior
Research Methods} 55 (3): 1079--93.
\url{https://doi.org/10.3758/s13428-022-01835-2}.

\bibitem[\citeproctext]{ref-LundBurgess1996}
Lund, Kevin, and Curt Burgess. 1996. {``Producing High-Dimensional
Semantic Spaces from Lexical Co-Occurrence.''} \emph{Behavior Research
Methods, Instruments, {\&} Computers} 28 (2): 203--8.

\bibitem[\citeproctext]{ref-MassonBodner2003}
Masson, Michael E. J., and Glen E. Bodner. 2003. {``A Retrospective View
of Masked Priming: Toward a Unified Account of Masked and Long-Term
Repetition Priming.''} \emph{Masked Priming: The State of the Art},
57--94.

\bibitem[\citeproctext]{ref-McClellandRumelhart1981}
McClelland, James L., and David E. Rumelhart. 1981. {``An Interactive
Activation Model of Context Effects in Letter Perception: Part i. An
Account of Basic Findings.''} \emph{Psychological Review} 88 (5):
375--407.

\bibitem[\citeproctext]{ref-Nievas2010}
Nievas, Francisco. 2010. {``The Frequency Attenuation Effect in Identity
and Associative Priming.''} \emph{The Spanish Journal of Psychology} 13
(1): 30--62. \url{https://doi.org/10.1017/s1138741600003668}.

\bibitem[\citeproctext]{ref-NorrisKinoshita2008}
Norris, Dennis, and Sachiko Kinoshita. 2008. {``Perception as Evidence
Accumulation and Bayesian Inference: Insights from Masked Priming.''}
\emph{Journal of Experimental Psychology: General} 137 (3): 434--55.
\url{https://doi.org/10.1037/a0012799}.

\bibitem[\citeproctext]{ref-PeirceEtal2019}
Peirce, Jonathan, Jeremy R. Gray, Sol Simpson, Michael MacAskill,
Richard Höchenberger, Hiroyuki Sogo, Erik Kastman, and Jonas Kristoffer
Lindeløv. 2019. {``PsychoPy2: Experiments in Behavior Made Easy.''}
\emph{Behavior Research Methods} 51 (1): 195--203.

\bibitem[\citeproctext]{ref-Petrosino2020}
Petrosino, Roberto. 2020. {``More Than Islands of Regularity: An
Investigation of the Sensitivity of Morphological Decomposition to
Higher-Level Linguistic Properties.''} PhD thesis, University of
Connecticut.

\bibitem[\citeproctext]{ref-PetrosinoEtal2023}
Petrosino, Roberto, Jon Sprouse, and Diogo Almeida. 2023. {``Asymmetries
in the Stem and Suffix Masked Priming Response in a Large-Scale Online
Study.''} \emph{Quaderni Di Linguistica e Studi Orientali}, no. 49:
177--94. \url{https://doi.org/10.13128/QUL-SO-2421-7220-15154}.

\bibitem[\citeproctext]{ref-PotvinSchtuz2000}
Potvin, Patrick J., and Robert W. Schutz. 2000. {``Statistical Power for
the Two-Factor Repeated Measures ANOVA.''} \emph{Behavior Research
Methods, Instruments, \& Computers} 32 (2): 347--56.
\url{https://doi.org/10.3758/bf03207805}.

\bibitem[\citeproctext]{ref-RajaramNeely1992}
Rajaram, Suparna, and James H Neely. 1992. {``Dissociative Masked
Repetition Priming and Word Frequency Effects in Lexical Decision and
Episodic Recognition Tasks.''} \emph{Journal of Memory and Language} 31
(2): 152--82. \url{https://doi.org/10.1016/0749-596x(92)90009-m}.

\bibitem[\citeproctext]{ref-ScarboroughEtal1977}
Scarborough, Don L., Charles Cortese, and Hollis S. Scarborough. 1977.
{``Frequency and Repetition Effects in Lexical Memory.''} \emph{Journal
of Experimental Psychology: Human Perception and Performance} 3 (1):
1--17. \url{https://doi.org/10.1037/0096-1523.3.1.1}.

\bibitem[\citeproctext]{ref-SeguiGrainger1990}
Segui, Juan, and Jonathan Grainger. 1990. {``Priming Word Recognition
with Orthographic Neighbors: Effects of Relative Prime-Target
Frequency.''} \emph{Journal of Experimental Psychology: Human Perception
and Performance} 16 (1): 65--76.
\url{https://doi.org/10.1037/0096-1523.16.1.65}.

\bibitem[\citeproctext]{ref-Sereno1991}
Sereno, Joan A. 1991. {``Graphemic, Associative, and Syntactic Priming
Effects at a Brief Stimulus Onset Asynchrony in Lexical Decision and
Naming.''} \emph{Journal of Experimental Psychology: Learning, Memory,
and Cognition} 17 (3): 459--77.
\url{https://doi.org/10.1037/0278-7393.17.3.459}.

\bibitem[\citeproctext]{ref-SpenceStanley2016}
Spence, Jeffrey R, and David J Stanley. 2016. {``Prediction Interval:
What to Expect When You're Expecting\ldots{} a Replication.''}
\emph{PloS One} 11 (9): 1--22.
\url{https://doi.org/10.1371/journal.pone.0162874}.

\bibitem[\citeproctext]{ref-StockallMarantz2006}
Stockall, Linnaea, and Alec Marantz. 2006. {``A Single Route, Full
Decomposition Model of Morphological Complexity: MEG Evidence.''}
\emph{The Mental Lexicon} 1 (1): 85--123.
https://doi.org/\url{https://doi.org/10.1075/ml.1.1.07sto}.

\bibitem[\citeproctext]{ref-TaftForster1975}
Taft, Marcus, and Kenneth I. Forster. 1975. {``Lexical Storage and
Retrieval of Prefixed Words.''} \emph{Journal of Verbal Learning and
Verbal Behavior} 14 (6): 638--47.

\bibitem[\citeproctext]{ref-w3counterGlobalStats}
W3 Counter. 2023. {``Browser \& Platform Market Share - November
2023.''}
\url{https://www.w3counter.com/globalstats.php?year=2023&month=11}.

\bibitem[\citeproctext]{ref-Yap2009}
Yap, Melvin J., and David A. Balota. 2009. {``Visual Word Recognition of
Multisyllabic Words.''} \emph{Journal of Memory and Language} 60 (4):
502--29.

\bibitem[\citeproctext]{ref-Zevin2002}
Zevin, Jason D, and Mark S Seidenberg. 2002. {``Age of Acquisition
Effects in Word Reading and Other Tasks.''} \emph{Journal of Memory and
Language} 47 (1): 1--29.

\bibitem[\citeproctext]{ref-Zimmerman2012}
Zimmerman, Robert, and Pablo Gomez. 2012. {``Drawing Attention to Primes
Increases Inhibitory Word Priming Effects.''} \emph{The Mental Lexicon}
7 (2): 119--46.

\end{CSLReferences}

\newpage{}

\section*{Wordlists}\label{wordlists}
\addcontentsline{toc}{section}{Wordlists}

\subsection*{Experiment 1}\label{experiment-1}
\addcontentsline{toc}{subsection}{Experiment 1}

\begin{longtable*}{lllrrrr}
\toprule
 &  &  & \multicolumn{2}{c}{RT (to repetition)} & \multicolumn{2}{c}{RT (to unrelated)} \\ 
\cmidrule(lr){4-5} \cmidrule(lr){6-7}
related & unrelated prime & word & mean & SD & mean & SD \\ 
\midrule\addlinespace[2.5pt]
\multicolumn{7}{l}{\emph{low frequency condition}} \\ 
\midrule\addlinespace[2.5pt]
arrow & hunch & arrow & 590 & 130 & 587 & 124 \\ 
pitch & sneak & pitch & 576 & 126 & 612 & 122 \\ 
hatch & widow & hatch & 621 & 151 & 639 & 148 \\ 
shark & brief & shark & 573 & 125 & 590 & 138 \\ 
tooth & sharp & tooth & 536 & 125 & 565 & 116 \\ 
booth & grief & booth & 572 & 136 & 627 & 157 \\ 
pound & sting & pound & 551 & 127 & 572 & 127 \\ 
weigh & thief & weigh & 593 & 167 & 636 & 164 \\ 
blank & avoid & blank & 571 & 139 & 596 & 124 \\ 
crush & award & crush & 554 & 128 & 592 & 136 \\ 
bench & smack & bench & 573 & 132 & 601 & 129 \\ 
fetch & brand & fetch & 622 & 156 & 658 & 146 \\ 
cheek & salad & cheek & 561 & 141 & 602 & 142 \\ 
brush & swamp & brush & 564 & 130 & 600 & 128 \\ 
march & depth & march & 559 & 125 & 580 & 123 \\ 
bleed & flesh & bleed & 560 & 148 & 577 & 146 \\ 
cliff & harsh & cliff & 602 & 130 & 645 & 137 \\ 
fraud & creep & fraud & 621 & 147 & 628 & 132 \\ 
cloud & plead & cloud & 536 & 115 & 551 & 101 \\ 
fluid & thumb & fluid & 605 & 140 & 678 & 162 \\ 
trash & creek & trash & 554 & 127 & 560 & 128 \\ 
flush & blond & flush & 576 & 123 & 617 & 140 \\ 
porch & stink & porch & 587 & 136 & 620 & 160 \\ 
stiff & patch & stiff & 626 & 154 & 678 & 156 \\ 
cough & sweep & cough & 564 & 142 & 601 & 141 \\ 
smash & squad & smash & 570 & 129 & 587 & 126 \\ 
\midrule\addlinespace[2.5pt]
\multicolumn{7}{l}{\emph{high frequency condition}} \\ 
\midrule\addlinespace[2.5pt]
blood & chief & blood & 541 & 130 & 551 & 104 \\ 
bunch & child & bunch & 585 & 148 & 617 & 145 \\ 
catch & board & catch & 545 & 116 & 562 & 130 \\ 
stuff & tough & stuff & 555 & 119 & 585 & 137 \\ 
break & stand & break & 545 & 107 & 561 & 124 \\ 
speak & beach & speak & 545 & 131 & 573 & 129 \\ 
stick & hotel & stick & 562 & 128 & 598 & 138 \\ 
sleep & angel & sleep & 538 & 113 & 559 & 119 \\ 
wrong & truth & wrong & 563 & 143 & 565 & 132 \\ 
grand & quick & grand & 571 & 127 & 582 & 143 \\ 
mouth & world & mouth & 543 & 125 & 556 & 119 \\ 
knock & extra & knock & 560 & 134 & 631 & 136 \\ 
guard & think & guard & 580 & 132 & 590 & 134 \\ 
small & thing & small & 557 & 130 & 577 & 125 \\ 
check & round & check & 558 & 135 & 562 & 121 \\ 
watch & proud & watch & 541 & 128 & 546 & 110 \\ 
group & smell & group & 559 & 127 & 576 & 142 \\ 
month & earth & month & 555 & 120 & 572 & 123 \\ 
south & relax & south & 575 & 139 & 611 & 133 \\ 
lunch & truck & lunch & 547 & 119 & 557 & 125 \\ 
clock & throw & clock & 548 & 132 & 574 & 124 \\ 
sound & death & sound & 538 & 127 & 552 & 103 \\ 
drink & north & drink & 559 & 129 & 556 & 122 \\ 
touch & young & touch & 541 & 122 & 573 & 121 \\ 
laugh & weird & laugh & 546 & 119 & 568 & 121 \\ 
black & reach & black & 553 & 131 & 563 & 114 \\ 
\midrule\addlinespace[2.5pt]
\multicolumn{7}{l}{non-word} \\ 
\midrule\addlinespace[2.5pt]
alkew & grack & alkew & 599 & 153 & 591 & 140 \\ 
agink & furob & agink & 626 & 148 & 614 & 141 \\ 
ruzak & begro & ruzak & 577 & 130 & 584 & 142 \\ 
sondo & labok & sondo & 625 & 142 & 612 & 149 \\ 
guesh & gazzo & guesh & 702 & 184 & 721 & 194 \\ 
fadio & criam & fadio & 618 & 149 & 604 & 146 \\ 
plich & coreb & plich & 650 & 162 & 640 & 159 \\ 
sgrew & docab & sgrew & 626 & 182 & 638 & 182 \\ 
sceak & colob & sceak & 675 & 154 & 683 & 171 \\ 
ghisk & isloo & ghisk & 588 & 139 & 593 & 139 \\ 
deirm & ahuck & deirm & 589 & 142 & 596 & 139 \\ 
villo & flurb & villo & 632 & 182 & 615 & 181 \\ 
tidow & pikto & tidow & 648 & 167 & 624 & 160 \\ 
drick & aliom & drick & 684 & 168 & 681 & 172 \\ 
phick & purso & phick & 643 & 160 & 637 & 165 \\ 
nello & borno & nello & 625 & 156 & 612 & 151 \\ 
feach & pacaw & feach & 730 & 201 & 720 & 192 \\ 
tello & rilth & tello & 651 & 175 & 644 & 171 \\ 
dolio & caveb & dolio & 602 & 148 & 610 & 165 \\ 
gorgo & swysh & gorgo & 643 & 164 & 619 & 170 \\ 
whilo & lanjo & whilo & 612 & 137 & 604 & 150 \\ 
stanf & drief & stanf & 611 & 134 & 617 & 133 \\ 
crulk & ocheb & crulk & 671 & 162 & 665 & 169 \\ 
phumb & tunch & phumb & 645 & 160 & 633 & 148 \\ 
sirth & steaf & sirth & 612 & 141 & 618 & 145 \\ 
slerk & nohew & slerk & 640 & 153 & 634 & 163 \\ 
vitbo & nualm & vitbo & 593 & 151 & 596 & 154 \\ 
sunch & ofium & sunch & 665 & 165 & 665 & 161 \\ 
soeth & croik & soeth & 589 & 141 & 589 & 130 \\ 
eltow & valuo & eltow & 628 & 171 & 606 & 158 \\ 
framo & sorgo & framo & 617 & 146 & 618 & 146 \\ 
lumpo & shavo & lumpo & 630 & 162 & 635 & 172 \\ 
spuff & oceab & spuff & 672 & 169 & 667 & 183 \\ 
gatob & tolio & gatob & 599 & 139 & 606 & 155 \\ 
nosom & theck & nosom & 598 & 155 & 604 & 139 \\ 
gezzo & tooch & gezzo & 592 & 136 & 586 & 131 \\ 
afoub & slonk & afoub & 582 & 133 & 589 & 128 \\ 
wateb & salch & wateb & 633 & 151 & 619 & 133 \\ 
nelch & raceb & nelch & 601 & 144 & 594 & 145 \\ 
dahoo & ahack & dahoo & 598 & 132 & 595 & 146 \\ 
driek & fideo & driek & 606 & 145 & 606 & 143 \\ 
gnask & fluko & gnask & 612 & 171 & 604 & 153 \\ 
brosk & cyrrh & brosk & 629 & 159 & 647 & 175 \\ 
duvez & revuo & duvez & 580 & 152 & 580 & 155 \\ 
fielm & cempo & fielm & 609 & 146 & 611 & 151 \\ 
pumph & exulk & pumph & 669 & 162 & 685 & 176 \\ 
gerif & kleck & gerif & 584 & 137 & 588 & 149 \\ 
racef & bonth & racef & 618 & 151 & 622 & 156 \\ 
pheek & scook & pheek & 640 & 155 & 644 & 176 \\ 
pruaw & slork & pruaw & 593 & 133 & 592 & 135 \\ 
guilm & whilf & guilm & 603 & 142 & 598 & 142 \\ 
lairf & drosh & lairf & 587 & 144 & 600 & 150 \\ 
\bottomrule
\end{longtable*}

\subsection*{Experiment 2}\label{experiment-2}
\addcontentsline{toc}{subsection}{Experiment 2}

\begin{longtable*}{lllrrrr}
\toprule
 &  &  & \multicolumn{2}{c}{RT (to repetition)} & \multicolumn{2}{c}{RT (to unrelated)} \\ 
\cmidrule(lr){4-5} \cmidrule(lr){6-7}
related & unrelated prime & word & mean & SD & mean & SD \\ 
\midrule\addlinespace[2.5pt]
\multicolumn{7}{l}{\emph{low frequency condition}} \\ 
\midrule\addlinespace[2.5pt]
arrow & hunch & arrow & 573 & 130 & 591 & 128 \\ 
pitch & sneak & pitch & 565 & 132 & 608 & 146 \\ 
hatch & widow & hatch & 606 & 163 & 644 & 164 \\ 
shark & brief & shark & 560 & 132 & 587 & 124 \\ 
tooth & sharp & tooth & 533 & 126 & 560 & 111 \\ 
booth & grief & booth & 566 & 147 & 618 & 163 \\ 
pound & sting & pound & 552 & 139 & 570 & 134 \\ 
weigh & thief & weigh & 583 & 158 & 649 & 186 \\ 
blank & avoid & blank & 556 & 132 & 583 & 132 \\ 
crush & award & crush & 542 & 126 & 583 & 142 \\ 
bench & smack & bench & 565 & 143 & 601 & 138 \\ 
fetch & brand & fetch & 612 & 149 & 663 & 146 \\ 
cheek & salad & cheek & 556 & 138 & 597 & 148 \\ 
brush & swamp & brush & 559 & 136 & 593 & 125 \\ 
march & depth & march & 560 & 149 & 575 & 122 \\ 
bleed & flesh & bleed & 552 & 137 & 568 & 135 \\ 
cliff & harsh & cliff & 595 & 149 & 662 & 157 \\ 
fraud & creep & fraud & 596 & 131 & 634 & 113 \\ 
cloud & plead & cloud & 539 & 132 & 552 & 108 \\ 
fluid & thumb & fluid & 601 & 167 & 690 & 173 \\ 
trash & creek & trash & 564 & 157 & 566 & 126 \\ 
flush & blond & flush & 570 & 133 & 624 & 155 \\ 
porch & stink & porch & 584 & 155 & 617 & 153 \\ 
stiff & patch & stiff & 617 & 163 & 677 & 161 \\ 
cough & sweep & cough & 552 & 136 & 591 & 145 \\ 
smash & squad & smash & 553 & 133 & 590 & 142 \\ 
\midrule\addlinespace[2.5pt]
\multicolumn{7}{l}{\emph{high frequency condition}} \\ 
\midrule\addlinespace[2.5pt]
blood & chief & blood & 534 & 131 & 558 & 112 \\ 
bunch & child & bunch & 584 & 156 & 619 & 146 \\ 
catch & board & catch & 529 & 111 & 568 & 141 \\ 
stuff & tough & stuff & 556 & 128 & 576 & 133 \\ 
break & stand & break & 542 & 128 & 564 & 123 \\ 
speak & beach & speak & 543 & 131 & 570 & 132 \\ 
stick & hotel & stick & 556 & 136 & 599 & 130 \\ 
sleep & angel & sleep & 525 & 111 & 566 & 137 \\ 
wrong & truth & wrong & 555 & 142 & 569 & 137 \\ 
grand & quick & grand & 564 & 129 & 588 & 147 \\ 
mouth & world & mouth & 534 & 123 & 552 & 116 \\ 
knock & extra & knock & 549 & 131 & 615 & 139 \\ 
guard & think & guard & 565 & 118 & 590 & 118 \\ 
small & thing & small & 549 & 137 & 567 & 130 \\ 
check & round & check & 546 & 125 & 563 & 120 \\ 
watch & proud & watch & 531 & 115 & 545 & 133 \\ 
group & smell & group & 553 & 126 & 579 & 142 \\ 
month & earth & month & 549 & 126 & 576 & 128 \\ 
south & relax & south & 563 & 138 & 603 & 144 \\ 
lunch & truck & lunch & 549 & 141 & 563 & 124 \\ 
clock & throw & clock & 547 & 133 & 574 & 130 \\ 
sound & death & sound & 534 & 127 & 559 & 147 \\ 
drink & north & drink & 550 & 124 & 563 & 116 \\ 
touch & young & touch & 545 & 145 & 567 & 120 \\ 
laugh & weird & laugh & 535 & 113 & 556 & 114 \\ 
black & reach & black & 547 & 129 & 578 & 129 \\ 
\midrule\addlinespace[2.5pt]
\multicolumn{7}{l}{non-word} \\ 
\midrule\addlinespace[2.5pt]
alkew & grack & alkew & 603 & 150 & 606 & 160 \\ 
agink & furob & agink & 632 & 161 & 615 & 155 \\ 
ruzak & begro & ruzak & 580 & 149 & 582 & 155 \\ 
sondo & labok & sondo & 619 & 168 & 621 & 178 \\ 
guesh & gazzo & guesh & 707 & 189 & 748 & 195 \\ 
fadio & criam & fadio & 632 & 169 & 602 & 171 \\ 
plich & coreb & plich & 645 & 148 & 646 & 180 \\ 
sgrew & docab & sgrew & 637 & 197 & 648 & 184 \\ 
sceak & colob & sceak & 692 & 179 & 692 & 165 \\ 
ghisk & isloo & ghisk & 595 & 149 & 594 & 152 \\ 
deirm & ahuck & deirm & 599 & 147 & 595 & 147 \\ 
villo & flurb & villo & 637 & 210 & 615 & 178 \\ 
tidow & pikto & tidow & 651 & 175 & 641 & 177 \\ 
drick & aliom & drick & 696 & 194 & 685 & 186 \\ 
phick & purso & phick & 648 & 176 & 639 & 168 \\ 
nello & borno & nello & 631 & 178 & 637 & 189 \\ 
feach & pacaw & feach & 748 & 201 & 745 & 208 \\ 
tello & rilth & tello & 665 & 185 & 662 & 174 \\ 
dolio & caveb & dolio & 618 & 175 & 609 & 157 \\ 
gorgo & swysh & gorgo & 651 & 179 & 631 & 178 \\ 
whilo & lanjo & whilo & 619 & 150 & 615 & 164 \\ 
stanf & drief & stanf & 613 & 147 & 627 & 134 \\ 
crulk & ocheb & crulk & 678 & 173 & 679 & 183 \\ 
phumb & tunch & phumb & 655 & 175 & 648 & 180 \\ 
sirth & steaf & sirth & 623 & 146 & 616 & 136 \\ 
slerk & nohew & slerk & 644 & 167 & 638 & 172 \\ 
vitbo & nualm & vitbo & 590 & 143 & 596 & 154 \\ 
sunch & ofium & sunch & 664 & 162 & 676 & 172 \\ 
soeth & croik & soeth & 590 & 131 & 589 & 146 \\ 
eltow & valuo & eltow & 638 & 170 & 603 & 153 \\ 
framo & sorgo & framo & 626 & 165 & 631 & 162 \\ 
lumpo & shavo & lumpo & 634 & 172 & 635 & 164 \\ 
spuff & oceab & spuff & 683 & 182 & 665 & 204 \\ 
gatob & tolio & gatob & 607 & 167 & 594 & 154 \\ 
nosom & theck & nosom & 603 & 156 & 608 & 160 \\ 
gezzo & tooch & gezzo & 601 & 160 & 600 & 164 \\ 
afoub & slonk & afoub & 583 & 136 & 587 & 133 \\ 
wateb & salch & wateb & 640 & 159 & 641 & 160 \\ 
nelch & raceb & nelch & 609 & 163 & 595 & 145 \\ 
dahoo & ahack & dahoo & 611 & 157 & 597 & 154 \\ 
driek & fideo & driek & 613 & 152 & 610 & 151 \\ 
gnask & fluko & gnask & 606 & 154 & 606 & 159 \\ 
brosk & cyrrh & brosk & 637 & 165 & 649 & 180 \\ 
duvez & revuo & duvez & 590 & 168 & 576 & 154 \\ 
fielm & cempo & fielm & 621 & 181 & 614 & 160 \\ 
pumph & exulk & pumph & 687 & 184 & 697 & 202 \\ 
gerif & kleck & gerif & 587 & 154 & 586 & 147 \\ 
racef & bonth & racef & 619 & 164 & 621 & 151 \\ 
pheek & scook & pheek & 641 & 157 & 635 & 169 \\ 
pruaw & slork & pruaw & 599 & 157 & 598 & 143 \\ 
guilm & whilf & guilm & 620 & 168 & 603 & 163 \\ 
lairf & drosh & lairf & 590 & 138 & 597 & 146 \\ 
\bottomrule
\end{longtable*}




\end{document}
